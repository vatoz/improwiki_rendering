\documentclass[main.tex]{subfiles}
 
\begin{document}
\needspace{5cm} \section{Improknihovnička} \label{improknihovnička} Ahoj, 
zdraví tě malá \textbf{Improknihovnička}{}. 
Mám v sobě několik knih v jazyce anglickém, které si můžeš půjčit, přečíst a vrátit. 
  
Aktuálně obsahuji tyto knihy (řazeno dle data získání): 
\begin{itemize}
\item Keith Johnstone: Impro for storytellers
\item John Cremer: Improv: Enjoy life and success with the power of yes
\item Patricia Ryan Madson: Improv wisdom : Don't prepare, Just show up
\item Viola Spolin: Improvisation for the Theater
\item Mick Napier: Improvise: Scene from the Inside Out
\item Charna Halpern, Del Close, Kim "Howard"{} Johnson: Truth in Comedy: Manual of Improvisation
\item Ben Hauck: Long-Form Improv: The Complete Guide to Creating Characters, Sustaining Scenes, and Performing Extraordinary Harolds
\item Tom Salinsky and Deborah Francis-White: The Improv Handbook: The Ultimate Guide to Improvising in Theatre, Comedy, and Beyond
\item Jimmy Carrane: Improvising Better: A Guide for the Working Improviser
\item Nancy Howland Walker: Instant Songwriting: Musical Improv from Dunce to Diva
\item Matt Walsh a kol.: Upright Citizens Brigade Comedy Improvisation Manual
\item Joe Samuel & Heather Urquhart: Sing it! The essential guide to musical improvised comedy
\item Jo Salas: Improvising real life
\item Jason Chin: Long-form improvisation & the Art of Zen
\item William Hall: Improv games for performers
\item Mick Napier: Behind the scenes (Improvising Longforms)
\item T.J.Jagodowski, David Pasquesi with Pam Victor: Improvisationat the speed of life
\end{itemize}
 
 
Jak postupovat? 
 
\begin{itemize}
\item  Oslovíš \odkaz{Vaška}{uživatel:vatoz}, který ti knihu půjčí.
\end{itemize}
 
Knihovnička přivítá i tvá doporučení na další rozšířování, či sponzorské dary. 
\needspace{5cm} \section{Literatura} \label{literatura} Pár tipů na knížky ze světa dramatické výchovy a improvizace. Kvalita není zaručena. Hodnoťte, pokud můžete nebo přidávejte odkazy, pokud je kniha dostupná online. 
 
Česká literatura 
\begin{itemize}
\item  Budínská, H.: Hry pro šest smyslů
\item  Propp, V. J.: Morfologie pohádky a jiné studie
\item  Ulrychová, I.: Drama a příběh
\item  Svozilová, L.: Tvořivá dramatika
\item  Štembergová-Kratochvílová, Š.: Loutkářova řeč
\item  Štembergová-Kratochvílová, Š.: Metodika mluvní výchovy
\end{itemize}
 
 
Zahraniční literatura  
 
\begin{itemize}
\item  Piers, L.: Theatresports
\item  Spolin, V.: Theatre Games
\item  Spolin, V.: Improvisation for the Theater
\item  Keith Johnstone: Impro for storytellers  (vyšlo v českém překladu)
\end{itemize}
 
Pár knih v jazyce anglickém je možno si \odkaz{půjčit}{improknihovnička} 
 
Cenné můžou být také knihy přímo o improvizaci od českých autorů 
\begin{itemize}
\item  Martin Vasquez: Staňte se mistrem improvizace
\item  Jana Machalíková: Improvizace ve škole
\item  Martin Vasquez & Mája Škvorová: Cvičení mistrů improvizace
\end{itemize}
 
či bakalářské a disertační práce 
\begin{itemize}
\item  Martin Vasquez: Principy  divadelní  improvizace  (v  prostředí  zápasů  v  divadelní improvizaci)
\item  Vít Piskala: Česká improvizační liga v roce 2013
\item  Tomáš Kováč: Koncepce a pravidla improvizační ligy jako klíčový prvek určující charakter divadelní komunikace mezi herci a diváky
\end{itemize}
 
 
 
\todo{isbn českých knih} 
\end{document}