\needspace{5cm} \section{Asociace (cvičení)} \label{asociace (cvičení)} Základní a často používané improvizační cvičení na rozvíjení rychlé reakce a představivosti. 
 
\subsection{Průběh} Hráči stojí v kruhu. Ten který je na řadě, řekne slovo, které mu nějakým, libovolným způsobem navazuje (asociuje) na slovo předchozí. Na toto nové slovo pak pokračuje následující hráč. 
 
\subsection{Varianty} \begin{itemize}
\item říká se druhá asociace ještě přes jedno slovo (sériové)
\item říká se druhá asociace na příchozí slovo (paralelní)
\item jdou dvě (tři) slova najednou, vymýšlí se nová slova pro každou z těchto řad samostatně.
\end{itemize}
 
 
 
 
 
 
 
\needspace{5cm} \section{Bunny bunny} \label{bunny bunny} Jedná se o velice bláznivé cvičení, které má za úkol rozehřát hráč a nabít energií. Hráči stojí v kruhu. Všichni společně drží rytmus poklepáváním dlaněmi o stehna (a občas skandováním "umpa"{} či "umča"{} ). Jeden z hráčů vysílá signál předvedením hlavičky králíka pomocí prstů. Palcem, ukazováčkem a prostředníčkem klepne o sebe 2x směrem k sobě, 2x směrem hráči, kterému signál posílá. Pohyb doprovází slovem "bunny". Sousedící hráči podporují králíka pohybem rozevřených dlaní doleva a doprava na úrovni hrudníků a pohyb doprovází slovem "tyky", "taky". Oba sousedé předvádí stříhání králičích uších, dlaně mají nad hlavou a opět doprovází pohyb dohodnutým zvukem ("flappy"). 
 
Pokud chceme hru zamotat ještě více, může se ten, kdo byl králíčkem jako poslední, otočit jednou kolem osy s hláškou "I am not bunny any more!"{} či "Já už nejsem králíček", tento pohyb se však vykonává až v okamžiku, kdy hráč nemá žádnou jinou povinnost ("bunny", "tyky,taky", "flappy") . 
 
Běžné je také (podobně jako u dalších podobných her) zrychlování až do závratného tempa. 
 
 
 
 
\needspace{5cm} \section{Dvě židle} \label{dvě židle} Dva herci vyjdou na jeviště, každý si nese svoji židli a náhodně zaplňují prostor. Trenér zaplňování ukončí tlesknutím nebo písknutím a oba hráči okamžitě položí židli a zaujmou k ní náhodnou pozici. Poté se rozehrává scéna, která vychází ze vzájemné pozice židlí a herců. 
 
\subsection{Může se plést s} \begin{itemize}
\item  \odkaz{Židle}{židle}
\item  \odkaz{Tři židle}{tři židle}
\end{itemize}
 
Cvičení je vhodné na \odkaz{vnímání prostoru}{vnímání prostoru} a na \odkaz{dávání si na čas}{dávání si na čas}. 
 
 
 
\needspace{5cm} \section{Gordický uzel} \label{gordický uzel} Kontaktní rozcvička, která začíná v kruhu. Postup je následující: 
 
\begin{enumerate}
\item  Skupina vytvoří kruh
\item  Všichni zavřou oči
\item  Každý natáhne do kruhu pravou ruku nahoru, levou ruku dolů a snaží se chytit ruku někoho jiného. Čeká se, dokud se všichni někoho nedrží
\item  Všichni otevřou oči a snaží se rozmotat zpátky do kruhu \textbf{aniž by se rozpojily ruce}{}
\end{enumerate}
 
Ne všechny možnosti propojení se dají rozmotat. Pokud se stane, že skupina utvoří oddělené n-tice, je cílem rozmotat je odděleně. 
 
 
 
 
 
\needspace{5cm} \section{Heja hají} \label{heja hají} Skupina stojí v kruhu a posílá se signál. 
 
Možné signály zhruba seřazené dle četnosti výskytu a znalostí mezi improvizátory: 
\begin{itemize}
\item \textbf{Heja}{} - pokračuje ve směru
\item \textbf{Hold down}{} - obrací směr
\item \textbf{Hají}{} - jednoho přeskočí
\item \textbf{Hold hají}{} - otočí a přeskočí
\item \textbf{Beng}{} - na kohokoli ve kruhu
\item \textbf{Pinguin party}{} - hráči si vyměňují místa
\item \textbf{Blechy}{} -  dtto s chytáním blech
\item \textbf{Klarinet}{} -  přepíná jen na gesta, bez zvuků
\item \textbf{Hoboj}{} - přepíná zpět
\end{itemize}
 
Dbejme na čistotu gest, rytmičnost a pozornost. 
 
\subsection{Varianty} \begin{itemize}
\item Několik signálů v kruhu
\item Místo zvuků říkáme \odkaz{asociace}{asociace}
\end{itemize}
 
\subsection{Viz také} \begin{itemize}
\item  \odkaz{Posílání signálu po kruhu}{posílání signálu po kruhu}
\end{itemize}
 
 
 
\needspace{5cm} \section{Hot spot} \label{hot spot} Toto cvičení je zaměřeno na nalazení hráčů a vzájemnou podporu. Hráči stojí v kruhu. Jeden z nich jde do prostřed a začne zpívat jakoukoliv známou písničku, ostatní se k němu přidávají. A i když neznají slova, snaží se ho jakkoliv podpořit. Ve chvíli kdy někdo z dalších hráčů má asociaci na novou písničku, vstupuje do kruhu, jemně se dotkne hráče uprostřed, a vymění si s ním místo. Střídání probíhá velice rychle a často zaznívají jen první slova z písničky. 
 
 
 
\needspace{5cm} \section{Kdo ukradl sušenky a ze stolu je vzal} \label{kdo ukradl sušenky a ze stolu je vzal} Skupina stojí v kruhu. Přípravou na cvičení je opakování věty "Kdo ukradl sušenky a ze stolu je vzal"{} (dále jen věta), dokud všichni nedrží rytmus a neznají to nazpaměť. Pro toto cvičení je ideální čtyřčtvrťový rytmus. 
 
Určí se, kdo začíná. Poté celá skupina jednou zopakuje \textif{větu}{}. Začínají jedinec si náhodně někoho vybere a použije v rytmu jeho jméno. Vypadá to následovně: 
 
\begin{enumerate}
\item  Skupina: Kdo ukradl sušenky a ze stolu je vzal
\item  A: Kdo ukradl sušenky a ze stolu je vzal
\item  A: Martin
\item  B: Kdo, já?
\item  A: Jó ty!
\item  B: Já ne!
\item  A: Tak kdo?
\end{enumerate}
 
Poté skupina opět zopakuje větu a pokračuje hráč, kterého byl naposledy označen za "zloděje sušenek". 
 
Cvičení se opakuje, dokud cvičící uzná za vhodné. 
 
\subsection{ Pomocné techniky } \begin{itemize}
\item  Dupání, klepání, luskání do rytmu
\end{itemize}
 
 
 
 
 
\needspace{5cm} \section{Kung-fu} \label{kung-fu} Hráči stojí ve dvojici naproti sobě, na větší vzdálenost než je jejich natažená paže. Hráči se střídají v útocích, švihnutím paže. Ve chvíli, kdy útočí hráč A, hráč B uhýbá. V obou případech hráči doprovází svůj pohyb slabikou "ha". 
 
\subsection{ Možnosti útoku a úhybu } \begin{itemize}
\item  Švihnutím paže nad hlavu protihráče, druhý hráč uhýbá skrčením se.
\item  Švihnutím paže přímo před sebe, druhý hráč uhýbá uhnutím v boku.
\item  Švihnutím paže k nohám protihráče, druhý hráč uhýbá vyskočením do vzduchu.
\end{itemize}
 
 
 
\needspace{5cm} \section{Ninjové} \label{ninjové} Hráči stojí v kruhu, v určeném pořadí (po směru hodinových ručiček) hráč, který je na tahu, útočí jedním plynulým pohybem na jiného hráče, cílem je zasáhnout dlaní či prsty předloktí druhého hráče a tím mu danou končetinu vyřadit. Hráč, na kterého je útočeno, smí jedním plynulým pohybem zkusit uniknout útoku. 
Hráč, který nemá ani jednu ruku, vypadává, stejně tak jako hráč, který útočí, když není na řadě. 
 
\subsection{Varianta (pokračování) } Ve chvíli kdy hráči ovládají hru v kruhu, můžeme přejít na vyhazování v reálném čase (bez určených tahů) s tím, že se člověk do hry vrací, když je vyhozen ten, kdo mu sebral poslední končetinu.  
 
 
\needspace{5cm} \section{Otázky (cvičení)} \label{otázky (cvičení)} Cílem této [[:Kategorie:Warm-up|warm-up]] hry je, aby účastníci na sebe prozradili něco nového a zároveň se nebáli akce a vystupování z řady. Moderátor předloží nahlas uzavřenou otázku, tedy takovou, aby na ní bylo možné odpovědět ano-ne. Ve větším prostoru mohou účastníci stát na jedné straně a v případě jejich kladné odpovědi, přejdou na druhou stranu místnosti. Moderátor následně menší skupinu vyzve také k přechodu, aby získal znovu 1 skupinu a pokládá další otázku. V menším prostoru se mohou účastníci přesunovat z židle na židli. Důležité je, aby bylo dostatečně viditelné, kdo odpovídá ano a akce byla výrazná. 
 
\subsubsection{Příprava hry:} Důležitá je příprava otázek, které mohou být od úplně jednoduchých po výrazně osobní. Správně vymyšlené otázky budou v účastnících vyvolávat také smích a tím dojde k uvolnění atmosféry. Tedy přesně toho, co od hry potřebujeme. 
 
\subsubsection{Délka:} 5 - 10 minut 
 
\subsubsection{Tipy otázek:} Máte radši kočky než psy? 
Věříte na horoskop? 
Zamilovali jste se někdy do cizince? 
 
 
 
 
\needspace{5cm} \section{Pomluva} \label{pomluva} \textbf{Pomluva}{} též známá jako fáma. Hráči v tomto cvičení imitují šíření pomluvy, tak jak to známe z běžného šíření "zaručených zpráv". Cvičení se podobá tiché poště, ale hráči si předávají informaci nahlas. Hráči stojí vedle sebe a prvnímu z nich je zadána věta s jednoduchou informací. Hráč předává tuto informaci druhému hráči s drobnou obměnou. Druhý hráč předává informaci třetímu zase trošku obměněnou, atd. až k poslednímu hráči. Je nutné, aby hráči reagovali jen na informaci, kterou obdrželi a ne na informaci, kterou obdrželi hráči před ním. Také je dobré, aby už v polovině skupiny nehrozila globální katastrofa či zničení vesmíru, života a vůbec. 
 
 
\subsection{Příklad}  
Základní informace: V samoobsluze zdražili rohlíky. 
 
1. hráč: V potravinách zase podražilo pečivo. 
 
2. hráč: Letos se urodilo málo obilí, a tak je vše dražší. 
 
3. hráč: Zemědělci si zase stěžují na malou úrodu. 
 
4. hráč: Zemědělci chtějí dotace od EU. 
 
atd. 
 
 
 
\needspace{5cm} \section{Popcorn} \label{popcorn} Cvičení pro alespoň pět hráčů. Všichni si stoupnou do kruhu a na místě vyskakují. Během výskoku hráč musí tlesknout - tedy dřív než dopadne. Pokud se dva hráči stlesknou, tedy tlesknou oba současně, musí si na krátký interval, třeba 5 vteřin, dřepnout. Zvuk tohoto cvičení pro vnějšího pozorovatele zní, jako když v mikrovlnné troubě praská popcorn, proto ten název. 
 
 
 
 
 
\needspace{5cm} \section{Posílání signálu po kruhu} \label{posílání signálu po kruhu} Cílem cvičení je nechat hráče se na sebe naladit, uvědomovat si ostatní a přijmout vlastní zodpovědnost v rámci skupiny. Zadání pro skupinu zní, aby posílání signálu bylo držené v jednom rytmu, a vždy, když signál předáváme dál, měli s příjemcem oční kontakt a věděli, že příjemce je připravený signál přijmout. 
 
\subsection{  Verze } \begin{itemize}
\item  V1: V lehké verzi hráči posílají signál po kruhu
\item  V2: Hráči posílají signál libovolně na přeskáčku
\item  V3: Hráči posílají signál na přeskáčku a mění varianty signálů
\item  V4: Více signálů najednou
\end{itemize}
 
\subsection{ Varianty signálů } \begin{itemize}
\item  formou tleskání
\item  formou přejetí dlaní o dlaní
\item  házení míčku
\item  mrknutí
\item  říkání jmen apod
\item  \odkaz{Heja hají}{heja hají}
\end{itemize}
 
\subsection{ Těžkosti } \begin{itemize}
\item  V případě, když skupina neudrží rytmus posílání signálu, je vhodné nastavit ve skupině pomalé tempo a postupně zrychlovat.
\item  Pokud nejsou hráči při předávání signálu v očním kontaktu, je dobré zkusit nejdříve variantu s mrknutím, které nutí hráče být pozorní.
\end{itemize}
 
\subsection{Varianty} \begin{itemize}
\item \odkaz{Heja hají}{heja hají}
\item \odkaz{Koncepty}{koncepty}
\end{itemize}
 
 
 
\needspace{5cm} \section{Ready Go} \label{ready go} Cvičení \textbf{Ready Go}{} je dobré pro procvičování poslouchání svých spoluhráčů a současně na pohyb. 
 
\subsection{ Průběh cvičení } Hráči stojí v řadě vedle sebe. První hráč začne vyprávět nějaký příběh na téma, nebo může třeba vyprávět svůj denní zážitek a snaží se u toho výrazně gestikulovat a pohybovat. Ostatní hráči poslouchají a sledují co dělá. Hlavně druhý hráč, který musí být připraven. Třetí hráč totiž může kdykoli během vyprávění 1. hráče říci \textbf{Ready!}{}, což je povel pro 2. hráče a ten musí co nejvěrněji kopírovat pohyby prvního hráče. Takto se chvíli pokračuje, aby se druhý hráč naladil na prvního, současně musí dávat pozor na to, co první hráč vypráví, protože třetí hráč může opět kdykoli říci \textbf{Go!}{} V tuto chvíli přebírá vyprávění druhý hráč a současně své vyprávění doprovází svými pohyby. První hráč odchází a může zbytek cvičení sledovat. Roli třetího hráče, který dával povely se posouvá na dalšího v řadě. Takto se vystřídají všichni hráči. Poslední hráč příběh uzavírá a povely mu udílí první hráč nebo trenér. 
 
Příběh by měl být kontinuální a zachovává se pohlaví prvního hráče. Pokud tedy začala vyprávět žena i mužští hráči pokračují ve vyprávění v ženském rodě. 
 
 
\needspace{5cm} \section{Sloupce} \label{sloupce} Cvičení ve kterém si herci pomocí minimalizace výrazových prostředků procvičí vnímání ostatních herců na jevišti, vnímání sebe, vnímání diváků a skupinovou responsivitu. Vhodný počet herců je 4 - 7. 
 
\subsection{ Průběh }  
Herci se postaví do řady tak aby mezi sebou měli místo alespoň na délku ruky. Prostor pro cvičení není omezen. Trenér zadá výrazové prostředky, které herci smějí používat. Pokud herec použije výrazový prostředek jiný, než bylo zadáno je dobré ho upozornit, aby se držel zadání. 
 
Cvičení nemá specificky daný čas a skupina si ho ukončuje sama. 
 
\subsubsection{ Výrazové prostředky }  
\begin{itemize}
\item  Krok dopředu
\item  Otočení o 180°
\item  Dřep dolů
\item  Dřep nahoru
\item  Výskok
\end{itemize}
 
\subsection{ Varianty } V zadání může být, že herci mají vyprávět příběh. Cvičení potom skončí s příběhem. 
 
 
 
 
 
\needspace{5cm} \section{Zabírání kopce} \label{zabírání kopce} Cvičení na rozhýbání skupiny a rozproudění myšlenek. V cvičení půjde hodně o přebíjení, ale také o spolupráci. 
 
\subsection{Průběh} Stojí se v kruhu. První člověk začíná se větou typu "Vylezu na kopec a ten je můj". Následuje slovní souboj o kopec, kdy každý mírně upraví co se stane například "Vylezu na kopec, shodím tě dolů a kopec je můj". Skupina by měla dělat co nejmenší změny, aby bylo kam gradovat. Může se jet po kruhu a nebo náhodně podle toho kdo má nápad.  
 
\subsection{Varianta beze slov} Hra je čistě pantomimická. Důraz na co nejmenší změny. 
 
 
 
 
 
\needspace{5cm} \section{Zaplňování prostoru} \label{zaplňování prostoru} Pramáti všech improvizačních cvičení, využívaná v nespočetném množství variant. \odkaz{Hráči}{hráč} se pohybují po prostoru a snaží se jej rovnoměrně zaplňovat. 
 
\subsection{Varianty} \begin{itemize}
\item Rychlosti chůze  1-5 , úkoly - výskok, dřep
\item Zdravení lidí pohledem
\item Chůze v emoci/charakteru, potkávání ostatních a sehrávání krátkých scén
\item Čtení dopisu/ básně ve více lidech
\item Počítání do 20
\item \odkaz{Štronzo}{štronzo}, zavřít oči a pak ukázat na nějakého hráče (jména, popis)
\end{itemize}
 
 
 
\needspace{5cm} \section{Zip, zep, zop} \label{zip, zep, zop} Hráči stojí v kruhu a posilají si mezi sebou jeden signal ukázáním na někoho prstem s hláškou zip, zep nebo zop v tomto pořadí. 
 
Kolikrát padne slovo “zip”, tolikrát se musejí opakovat i další dvě hlášky. 
 
 
\textbf{Příklad:}{}

 
''Hráč A vysílá “zip”, pokud hráč B řekne “zep”, hráč C musí říct “zop”. 
Pokud hráč B zopakuje “zip”, musí i další hlášky padnout 2x a tak dále.''  
 
 
Další “zip” už restartuje počet a opět záleží kolikrát “zip” od hráčů padne. 
 
Ve chvíli, kdy hráč udělá chybu nebo zaváha, říká místo “zip, zep a zop”, “bip, bep a bop” a s každou další chybou dostává nové písmeno na začátek, třeba až do “g”. Pak může následovat ještě obměna a může říkat například místo “gop” “gong” apod. 
 
Hra zpravidla pokračuje do té doby dokud i poslední hráč neudělá chybu. 
 
\subsubsection{Návrh otázek k reflexi:} \begin{itemize}
\item  Co vám na hře přišlo složité?
\item  Jakou jste měli strategii, abyste hru vyhráli?
\item  Co vám pomáhalo neudělat chybu?
\end{itemize}
 
 
 
 
 
 
 
 
 
\needspace{5cm} \section{Živé obrazy} \label{živé obrazy} \katabox{Prostředí 
}{neomezený 
}{neomezený, obvykle 2-3 minuty} 
 
Je ve své prapodstatě \odkaz{štronzo}{štronzo} několika improvizátorů které vytváří situaci, či prostředí. V živém obraze není vhodné se hýbat, ani vydávat zvuky. 
 
{{Todo| 
\begin{itemize}
\item  Doplnit průběh
\item  Doplnit video
\end{itemize}
}} 
 
 
 
 
