\needspace{5cm} \section{Bránění ve hře} \label{bránění ve hře} \faulbox{\obrazekmaly{Braneni.png}}{Rozhodčí zkříží ruce ve výši prsou}{1 } 
 
Bránění ve hře též blokování hry je faul pískán v případě, pokud hráč neguje nabídku svého spoluhráče případně pokud jej výrazně manipuluje. Dalším případem je, pokud spoluhráči zamezí možnost se projevovat nebo pohybovat. 
 
 
\needspace{5cm} \section{Hrubá chyba} \label{hrubá chyba} \faulbox{\obrazekmaly{Hruba-chyba.png}}{pohyb zalomenou paží nahoru a dolů}{2} 
Pokud hráč opakuje již faulované chování, může rozhodčí pískat hrubou chybu. V případě dále stupňovaného opakovaného chybování může rozhodčí hráče vyloučit z následující kategorie/kategorií. 
 
\needspace{5cm} \section{Klišé} \label{klišé} \faulbox{\obrazekmaly{Klise.png}}{Rozhodčí klepne dlaní o kotník}{1 } 
 
Pokud je v improvizaci použito jméno širokou veřejností známé značky, osobnosti, knižní nebo filmové postavy, citace z filmu apod., měl by být písknut \odkaz{faul}{faul} \textbf{Klišé}{}. Obecně se do tohoto výčtu počítají jakékoliv reference na známé věci. V zápasových kategorií je používání klišé zakázáno a pískáno jako faul, protože ve hře odkazuje na věci mimo hru a stává se vtipem, který hru rozbíjí, místo, aby ji rozvíjel. 
  
 
Výjimkou z tohoto pravidla jsou názvy \odkaz{improligových}{čili} týmů a názvy zemí. 
 
 
\needspace{5cm} \section{Mimo rámec} \label{mimo rámec} \faulbox{\obrazekmaly{Mimo-ramec.png}}{Rozhodčí do vzduchu kreslí rámeček}{1 } 
 
\odkaz{Hráč}{hráč} porušil stanovená pravidla \odkaz{kategorie}{kategorie} a nebo došlo k úplnému opuštění tématu improvizace. Tento \odkaz{faul}{faul} se nejčastěji objevuje v kategorii \odkaz{Smrt v 1 minutě}{smrt v 1 minutě}, když jedna postava nezemře po 60 vteřinách, případně ve \odkaz{Stíhačce}{stíhačka}, začnou li hráči hrát ještě před písknutím. 
 
 
\needspace{5cm} \section{Nepovolená rekvizita} \label{nepovolená rekvizita} \faulbox{\obrazekmaly{Nepovolena-rekvizita.png}}{Rozhodčí se tahá za cíp dresu}{1 } 
 
Pokud není dohodnuto jinak a nebo nemá \odkaz{kategorie}{kategorie}, ve které se scénka hraje, vlastní pravidla o \odkaz{rekvizitách}{rekvizita}, jsou na \odkaz{zápasech}{zápas} povoleny pouze následující věci pro použití v improvizaci: 
 
\begin{itemize}
\item  vlastní tělo
\item  tričko nebo týmový dres
\item  ostatní \odkaz{hráči}{hráč}, kteří nejsou v \odkaz{postavě}{postava}
\end{itemize}
 
V momentě, kdy \odkaz{hráč}{hráč} použije jakýkoliv jiný objekt jako \odkaz{rekvizitu}{rekvizita}, je na místě tento \odkaz{faul}{faul} 
 
 
\needspace{5cm} \section{Nepovolený postup} \label{nepovolený postup} \faulbox{\obrazekmaly{Nepovoleny-postup.png}}{poklepávání na paži}{1 } 
 
 
Hráč či tým nedodržují pravidla kategorie či ducha improvizace, a nelze to postihnout jinak. 
\needspace{5cm} \section{Nepozornost} \label{nepozornost} \faulbox{\obrazekmaly{Nepozornost.png}}{ruka držící druhou v zápěstí}{1} 
 
O nepozornost se jedná, pokud hráč ignoruje vstup nových postav na pódium, 
nehraje s informacemi, které padly (včetně jmen postav), nebo pokud na scéně  
vládne obecné neporozumění. 
 
 
\needspace{5cm} \section{Nepřehlednost hry} \label{nepřehlednost hry} \faulbox{\obrazekmaly{Neprehlednost-hry.png}}{Rozhodčí dělá rukama mlýnek}{1} 
 
Pokud situace v rozehrané improvizaci není jasná, je na místě \odkaz{faul}{faul} \textbf{Nepřehlednost hry}{}. Zpravidla takové situace vznikají tím, že \odkaz{hráči}{hráč} vytvoří zbytečně komplikovaný příběh nebo prostředí a nebo dojde k vzájemnému nepochopení. 
 
 
\needspace{5cm} \section{Nerespektování postavy} \label{nerespektování postavy} \faulbox{\obrazekmaly{Nerespektovani-postavy.png}}{Rozevřenou rukou (dlaní) si zakrejeme obličej}{1} 
 
Faul nerespektování postavy píská \odkaz{rozhodčí}{rozhodčí}, pokud některý hráč  
jedná s jinou postavou v rozporu s tím, co o jeho postavě víme. 
Nejčastěji se jedná o zapomenutí jména či rodinného stavu. 
Dále v případě, že muž hraje ženu a naopak. 
\needspace{5cm} \section{Nevhodné chování} \label{nevhodné chování} \faulbox{\obrazekmaly{Nevhodne-chovani.png}}{Ruce v bok}{2 } 
 
 
Pod nevhodné chování se dají zahrnout rozličné chyby proti dobrému duchu 
improvizace, zejména pak používání sprostých a vulgárních termínů, 
nadbytečné urážení týmu protivníka, neustálé odmlouvání hlavnímu rozhodčímu,  
apod. Dále sem patří i opakované stupňované \odkaz{šaškování}{šaškování}. 
\needspace{5cm} \section{Nátlak (faul)} \label{nátlak (faul)} \faulbox{\obrazekmaly{Natlak.png}}{Rozhodčí bouchá pěstí do dlaně}{1 } 
\odkaz{Hráč}{hráč} blokuje ostatní spoluhráče tím, že se snaží protlačit si vlastní způsob rozvoje příběhu a nebo je omezuje akcemi svojí postavy např. rozkazováním a nebo odmítáním. 
 
 
\needspace{5cm} \section{Plynulost hry} \label{plynulost hry} \faulbox{\obrazekmaly{Plynulost-hry.png}}{Rozhodčí krouží prstem nad hlavou}{1 } 
 
Pokud scénka vázne, je na místě písknout faul \textbf{plynulost hry}{}. Taková situace může nastat, pokud se příběh nikam dlouho neposune, což může být zapříčiněno více faktory. 
 
\begin{itemize}
\item  \odkaz{hráči}{hráč} nenastoupí včas do \odkaz{kategorie}{kategorie}
\item  postavy se pouze dohadují
\item  není definován \odkaz{improvizační trojzubec}{improvizační trojzubec}
\end{itemize}
 
 
\needspace{5cm} \section{Přečíslení} \label{přečíslení} \faulbox{\obrazekmaly{Precisleni.png}}{Rozhodčí se plácá dlaní do hlavy}{1 } 
 
Pokud vznikne situace, kdy je na scéně o dva a více hráčů z jednoho týmu víc než z druhého jedná se o \textbf{přečíslení}{}. Jeden tým totiž ubírá prostor tomu druhému a vytváří tak nerovné podmínky. Tento faul se nepíská v \odkaz{kategoriích porovnávacích}{:kategorie:porovnávací kategorie} 
 
 
\needspace{5cm} \section{Vypadnutí z role} \label{vypadnutí z role} \faulbox{\obrazekmaly{Vypadnuti-z-role.png}}{Ruka zaťatá s pěstí, zapumpujeme dvakrát nahoru a dolu   }{1} 
 
Faul vypadnutí z role píská \odkaz{rozhodčí}{rozhodčí} zejména ve chvíli, kdy se hráč odbourá - rozesměje,  
nebo pokud změní svoji postavu příliš razantním a pro děj narušujícím způsobem. 
 
\needspace{5cm} \section{Výstraha} \label{výstraha} \faulbox{\obrazekmaly{Vystraha.png}}{poklepávající ruce v bok }{2} 
 
Hráč může být v zápase varován před opakováním nějaké chyby.  
Při dalším opakování se pak může pískat výstraha následovaná např. vyloučením z další kategorie. 
 
 
 
 
\needspace{5cm} \section{Šaškování} \label{šaškování} \faulbox{\obrazekmaly{Saskovani.png}}{Rozhodčí ukazuje dlouhý nos}{1 } 
 
\odkaz{Hráč}{hráč} se snaží být vtipnější na úkor ostatních a sklouzává k trapným klišé nebo gagům, které neposouvají děj, případně nabourává děj jednorázovými fórky. 
 
 
