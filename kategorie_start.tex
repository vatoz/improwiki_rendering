\documentclass[main.tex]{subfiles}\begin{document}
\needspace{5cm} \label{kategorie} Slovo kategorie je nešikovný překlad anglického slovíčka "play"{} (hra). Tímto slovem se rozumí ohraničení začátku a konce nějakého určitého formátu improvizace. Neplést se slovem \odkaz{forma}{forma}.  
 
 
\needspace{5cm} \label{kategorie:zápasové kategorie} Zde se nacházejí kategorie, které se objevují na \odkaz{Improligových}{čili} \odkaz{zápasech}{zápas}. Některé kategorie se hrají častěji, některé jsou již trochu zapomenuté. Před zápasem je vhodné seznam kategorií projít a některé si upřesnit, protože pravidla kategorií se často liší. Zejména se jedná o kategorie: \odkaz{Zpívaná}{zpívaná}, \odkaz{Pyramida}{pyramida}, \odkaz{S rekvizitou}{s rekvizitou}. 
 
U zápasových kategorií je důležité, aby nabízely stejné možnosti oběma týmům. V případě, že se v kategorii vyskytuje speciální jedna role, doporučuje se v polovině času vyměnit hráče. Příkladem může být např. \odkaz{Překlad do znakové řeči}{překlad do znakové řeči}.  
 
 
Některé kategorie najdete také v  \odkaz{kategoriích na improshow}{:kategorie:kategorie na improshow}, či v \odkaz{rozcvičkách}{:kategorie:rozcvičky} 
 
 
 

\end{document}