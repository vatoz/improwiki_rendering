\documentclass[main.tex]{subfiles}
 
\begin{document}
\needspace{5cm} \section{1000 způsobů jak} \label{1000 způsobů jak} \katabox{Činnosti (3-5) 
}{libovolný, obvykle se zapojí všichni 
}{volný, dokud nedojdou nápady} 
 
 
Nebo taky \textbf{Tisíc způsobů jak}{}. Hráči hledají různé způsoby, jak vykonat činnost, například \odkaz{výměna žárovky}{výměna žárovky}. 
 
 
\subsection{Průběh}  
\odkaz{MC}{mc} vybere od publika činnosti. Poté se hráči postaví do půlkruhu a střídají se. Každý hráč, který vyběhne, předvede volně nějaký svůj alternativní způsob, jak vykonat zadanou činnost. Obvykle MC po chvíli činnost vymění, zadá jinou. 
 
\subsection{ Varianty } \begin{itemize}
\item  \odkaz{Věty, které by neměly zaznít}{věty, které by neměly zaznít}
\item  \odkaz{Metafory}{metafory}
\item  \odkaz{1000 hitů o...}{1000 hitů o...}
\end{itemize}
 
\subsection{ Běžné chyby } \begin{itemize}
\item Sklouzávání k záchodovému humoru
\item Přehnané použití dynamitu
\end{itemize}
 
 
 
 
 
\needspace{5cm} \section{Abeceda} \label{abeceda} \katabox{Téma 
}{2 
}{2 min} 
 
V kategorii \textbf{Abeceda}{} vám 2 hráči zahrají na Vaše téma, ale tak, že jejich repliky musí začínat písmeny tak, jak jsou seřazeny v abecedě. 
 
\subsection{Průběh} Připraví se dva hráči a je jim zadáno téma či prostředí. Hráči poté rozehrají improvizaci, ale s omezením, že jejich repliky musí začínat písmeny podle pořadí v abecedě. V případě, že se hráč splete může být nahrazen jiným hráčem ze svého týmu. Pokud hráči dojdou až na konec abecedy, začíná se opět od písmene A. 
 
\subsection{ Varianty } \begin{itemize}
\item  Hraje se i na písmena s háčky a čárkami.
\item  Mohou se zapojit další hráči.
\item  Kategorie končí písmenem Z.
\end{itemize}
 
\subsection{ Běžné chyby } \begin{itemize}
\item Hráč se ztratí v abecedě.
\end{itemize}
 
 
 
 
 
\needspace{5cm} \section{Break-up song} \label{break-up song} \katabox{Dobrovolník z publika, židle 
}{2 
}{neomezený, 3 - 5 minut (podle počtu hráčů)} 
 
Hráči zazpívají o rozchodu na motivy informací od jednoho dobrovolníka. 
 
\subsection{ Průběh } Průběh je téměř stejný jako u kategorie \odkaz{Love song}{love song}. \odkaz{MC}{mc} ale z publika požádá diváka, který v nedávné době prošel rozchodem. Píseň je mířená na jeho protějšek. 
 
\subsection{ Varianty } \begin{itemize}
\item \odkaz{Love song}{love song}
\end{itemize}
 
 
 
 
 
\needspace{5cm} \section{Báseň (cvičení)} \label{báseň (cvičení)} Na zadané téma diváků, které může být buď inspirací nebo přímo názvem \odkaz{básně}{báseň}, hráči postupně chodí dopředu a každý přednáší  
jeden verš. Verše se rýmují \odkaz{ABAB}{abab}. 
 
V případě více slok, hráči, kteří složili 1. sloku opouští hrací prostor a vrací se k ostatním hráčům a kdokoliv z nich může začít tvořit 
další sloku. 
 
\subsection{Varianty} \begin{itemize}
\item  báseň končí po po první sloce.
\end{itemize}
  
\subsection{Viz také} \begin{itemize}
\item \odkaz{Poetická}{poetická}
\item \odkaz{Divadlo poezie}{divadlo poezie}
\item \odkaz{Báseň od publika}{báseň od publika}
\end{itemize}
 
\subsection{Běžné chyby} \begin{itemize}
\item Důležitým prvkem u básně je zvýraznit poslední řádek, dát mu důraz, aby posluchač cítil, že touto větou báseň končí.
\item Tím, že verše vznikají na jevišti a nejistotou hráčů, mohou básně působit občas plitce. Je tedy třeba dbát v tomto cvičení na přednes a sebevědomí hráčů.
\end{itemize}
 
 
 
 
\needspace{5cm} \section{Divadlo poezie} \label{divadlo poezie} \katabox{Téma 
}{4 a více 
}{4 minuty} 
 
 
\textbf{Divadlo poezie}{} poetickým způsobem ztvární vaše téma. 
 
 
\subsection{Průběh} Hráči společně tvoří pomocí zvuků, rýmu a pohybu zvukomalebný příběh. 
Využívají přitom opakování slov, zvuků, pohybů, celá skupina se organicky přelévá a využívá synonyma, asociace a přirovnání. 
 
{{todo|rozšířit text, přidat videa.}} 
 
 
 
 
 
\needspace{5cm} \section{Dopředu, dozadu} \label{dopředu, dozadu} \katabox{Téma 
}{2 a naskakující 
}{neomezený} 
 
V této kategorii se podíváme na příběh několikrát a zajímavé části si přetočíme dopředu nebo zpátky. 
 
\subsection{ Průběh } Hraje se volně příběh. \odkaz{MC}{mc} může kdykoliv přepnout směr děje příběhu dopředu nebo dozadu. Pokud \odkaz{hráči}{hráč} hrají příběh směrem dopředu a zazní povel "Dozadu!", opakují pozpátku všechny pohyby a repliky v opačném pořadí než zazněly. Řeč se při směru dozadu neobrací, mluví se normálně. Příběh se odehrává až na začátek nebo dokud \odkaz{MC}{mc} opět nepřepne směr. Poté, co \odkaz{MC}{mc} přepne opět dopředu, všechno se zopakuje, dokud se \odkaz{hráči}{hráč} nedostanou k bodu, kde ještě příběh nebyl zahrán a pokračují v ději. 
 
\subsection{ Běžné chyby } \begin{itemize}
\item  Příliš mnoho povídání, statická scénka
\item  Hráči zapomenou, co se v příběhu stalo
\item  \odkaz{MC}{mc} pouští příliš dlouhé úseky dozadu
\end{itemize}
 
 
 
 
 
\needspace{5cm} \section{Dva začátky} \label{dva začátky} \katabox{Téma 
}{libovolný 
}{5 minut} 
 
 
Diváci si ze \textbf{dvou začátků}{} příběhu vyberou, který chtějí slyšet. 
 
 
\subsection{Průběh} Dva hráči krátce shrnou příběh, přednesou anotaci či ukázku - dva rozličné příběhy na stejné téma. 
Následně se \odkaz{hlasováním}{hlasování} určí, který příběh bude pokračovat, ten následně hráči sehrají. 
 
\subsection{ Varianty } \begin{itemize}
\item  \odkaz{Režiséři}{režiséři}
\end{itemize}
 
 
 
\needspace{5cm} \section{Dvě repliky} \label{dvě repliky} \katabox{Dvě věty pro kažého (až na jednoho) hráče 
}{libovolný (3 a více) 
}{3 minuty} 
 
Hráči mají jen \textbf{dvě repliky}{} na to, aby vyjádřili vše, co chtějí říci. 
 
\subsection{Průběh} Jeden hráč není ve svém vyjadřování nijak omezován, další hráči mají přidělen každý dvě jiné obecné věty (např. Nesnáším tě.  Tohle by se za Marie Terezie nestalo.). Jenom \odkaz{replikami}{replika} mohou reagovat na jakékoli promluvy centrálního hráče, který své nabídky volí tak, aby ostatním tyto reakce umožnil. 
 
\subsection{Běžné chyby} \begin{itemize}
\item  Neomezovaný hráč nenahrává těm ostatním.
\item  Příliš obskurní věty pro hráče.
\end{itemize}
 
 
 
\needspace{5cm} \section{Facebookový status} \label{facebookový status} \katabox{Poslední facebookový status některého diváka 
}{libovolný  
}{3  minuty} 
 
Hráči sehrají příběh ze života inspirovaný  \textbf{Facebookovým statusem}{} některého diváka. 
 
\subsection{Průběh} Hraje se příběh statusem inspirovaný, situace která končí tímto statusem.  
 
 
 
\needspace{5cm} \section{Jestli víš, co tím myslím} \label{jestli víš, co tím myslím} \katabox{Prostředí 
}{2+  
}{2  minuty} 
 
Každé slovo může získat zcela nový význam, \textbf{jestli víte, co tím myslím}{}. 
 
\subsection{Průběh} Dva hráči rozehrávají scénu v daném prostředí, ale každou svou \odkaz{repliku}{replika} zakončí  
větou jestli víš, co tím myslím. Všechny věty tak získávají jakousi "dvousmyslnost", která se ještě podpoří volbou slov před tímto zakončovačem repliky.   
 
 
 
\needspace{5cm} \section{Kramářská píseň} \label{kramářská píseň} \katabox{Téma 
}{libovolný (4 a více) 
}{4 minuty} 
 
\textbf{Kramářská píseň}{} sloužila předkům jako bulvární zdroj informací. 
 
\subsection{Průběh} Jeden hráč se stává zpěvákem, na jeho tlesknutí se velmi rychle mění \odkaz{živý obraz}{živý obraz} tvořený ostatními hráči. 
Hráč zpěvem popisuje děj, a obrazy, mezi slokami obvykle přechází do vyvolavačské komunikace s diváky ( Ale to ještě není vše, přijďte blíž a uvidíte, přijďte blíž a uslyšíte.) 
 
Pro zjednodušení této obtížné kategorie se občas používá melodie z "Pes jitrničku sežral" 
 
\subsection{Varianty} \begin{itemize}
\item  Na zápase mohou být dva zpěváci, kteří se střídají po sloce a promluvě.
\end{itemize}
 
\subsection{Viz také} Podobná je kategorie \odkaz{Diapozitivy}{diapozitivy}. 
 
 
 
 
 
\needspace{5cm} \section{Muzikál} \label{muzikál} \katabox{Slovní téma 
}{2 - 5 
}{5-15 minut} 
 
Muzikál je improvizační kategorie založena na příběhu (pro lepší přehlednost zpravidla jednoho hlavního hrdiny) do kterého jsou vkládány písně, které posunují děj vpřed. Důležitá a hojně opomíjena je složka taneční, choreografická. Není ovšem nutné stále jen zpívat (narozdíl od kategorie \odkaz{opera}{opera}) v muzikálu jsou stejně nutné i dialogy. 
 
\subsection{Jak na to}  
Důležité je mít příběh (nejlépe hlavní hrdina někam musí jít a něco udělat aby něco se stalo). Často se hledí hlavně na zpěv, ten však je pouze prostředek ke sdělení nějaké informace (proto když vám prostě nejde rým tak tam nerýmujte, ale veďte příběh dál). Tip na tvoření choreografií: postavě se tak, aby jeden byl kousek před ostatními ten pak udává pohyby které ti za ním zrcadlí (vytváří tak efekt nacvičených choreografií) 
 
{{Todo| 
\begin{itemize}
\item  Doplnit průběh
\item  Doplnit video
\end{itemize}
}} 
 
 
 
 
 
\needspace{5cm} \section{Oprsklá} \label{oprsklá} \katabox{téma/prostředí/vztah/situace 
}{2 
}{neomezen 
} 
 
V kategorie \textbf{Oprsklá}{} jsou hráči opravdu překvapeni každou informací, kterou se dozví. 
 
\subsection{Příprava} Každý hráč obdrží naplněnou sklenici vody.  
 
\subsection{Průběh} Kategorie spočívá v tom, že si navzájem sdělují informace, které druhého překvapí. Výrazem překvapení je následné vyprsknutí vody na spoluhráče. Hraje se, dokud je v půllitru nějaká voda.  
 
\subsection{Úklid} Po kategorii je scéna i hráči mokrá. Doporučujeme uklidit. 
 
\subsection{Běžné chyby} \begin{itemize}
\item Hráč ve své větě, větách nepřinese žádnou informaci.
\item Hráč nereaguje udiveně.
\item Hráč neumí vodu vyprsknout (doporučujeme natrénovat)
\end{itemize}
 
\subsection{Viz také} Kategorie je částečně podobná \odkaz{Co tím chceš jako říct}{co tím chceš jako říct}, hodí se zejména na horké letní festivaly. 
 
 
 
\needspace{5cm} \section{Orákulum} \label{orákulum} \katabox{otázka 
}{všichni 
}{volný} 
 
V této kategorii se z hráčů stává mýtické všeznalé stvoření, kterého se diváci ptají na záludné otázky. 
 
\subsection{ Průběh } Hráči se k sobě přitisknou a dohromady odpovídají na otázky z publika. Diváci se ptají na cokoliv, často i řečnické otázky. Orákulum by mělo být schopné na ně originálně a jednohlasně odpovědět. 
 
\subsection{ Běžné chyby } \begin{itemize}
\item Dlouhé odmlky
\item Všichni mluví přes sebe
\item Orákulum má viditelně jednoho hlavního mluvčího, po kterém všichni opakují
\end{itemize}
 
 
 
 
\needspace{5cm} \section{Režiséři} \label{režiséři} \katabox{Téma pro každého z režisérů}{neomezen}{neomezen} 
 
V kategorii \textbf{Režiséři}{} též známé jako \textbf{Super scene}{} několik hráčů v roli režiséra představí divákům své filmy. Diváci mají poté možnost rozhodovat o tom, který film chtějí vidět. vyberte si, který chcete vidět. 
 
\subsection{Průběh} Vyčlení se hráči, kteří se stávají režiséry. Každému z režisérů je zadán název jeho filmu (téma od diváků) a jeho žánr. Každý režisér má poté max. 30s na krátkou slovní upoutávku tohoto filmu, kterým se snaží diváky navnadit. Po představení všech filmů diváci hlasují o tom, který film vidět vůbec nechtějí. Z ostatních filmů se poté odehraje jejich první část, během níž může režisér komentovat děj filmu zákulisními informacemi nebo děj přímo ovlivňovat. Každý režisér první část svého filmu ukončí. Ke slovu se opět dostává \odkaz{MC}{mc} a diváci. Opět je zde hlasování po kterém opět jeden z filmů vypadne. Takto kategorie pokračuje až do konce jediného celého filmu. 
 
 
\subsection{ Varianty } \begin{itemize}
\item  \odkaz{Dva začátky}{dva začátky}
\end{itemize}
 
 
 
\needspace{5cm} \section{Scénář / Souboj s knihou} \label{scénář / souboj s knihou} \katabox{Připravená kniha či scénář 
}{2 
}{3 minuty} 
 
 
V této  kategorii  vám hráči sehrají příběh, který bude částečně shodný s  některým známým dílem. 
 
\subsection{Varianty začátku} \begin{itemize}
\item \textbf{Scénář}{} - Jeden hráč dostane od \odkaz{MC}{mc} přidělený kus scénáře - dialogu, ve kterém je vše co říká jedna postava vyškrtáno. Hráč musí používat pouze věty z dialogu té druhé postavy a to v původním pořadí.
\item \textbf{Souboj s knihou}{} - Jeden hráč dostane přidělenou knihu.  Všechny věty, které pronese, musí být \odkaz{repliky}{replika} z této knihy.
\end{itemize}
 
\subsection{Průběh} Sehraje se scénka, kdy druhý hráč musí citlivě reagovat a pečlivě poslouchat prvního hráče, v případě Scénáře navíc odhadovat, co se asi v scénáři může stát, aby reakce druhého hráče vypadaly logicky. 
 
\subsection{Viz také} \begin{itemize}
\item  \odkaz{Dvě repliky}{dvě repliky}
\end{itemize}
 
 
 
\needspace{5cm} \section{Shakespeare pozpátku} \label{shakespeare pozpátku} \katabox{Téma 
}{neomezený 
}{cca 10 minut} 
 
Uvidíme pravé a hutné drama jak od Shakespeara, ale celé \textbf{pozpátku}{}. 
\subsection{Průběh} Hraje se příběh rozdělený do 5 dějství pojmenovatelných dle modifikovaných částí \odkaz{příběhu}{příběh}.  
Tyto dějství se ale hrají v opačném pořadí - začínáme tedy od konce.  
\begin{itemize}
\item Tragedie
\item Konflikt/Intriky
\item Zápletka
\item Láska/Vztahy
\item Úvod
\end{itemize}
 
V rámci dějství jsou scény řazeny standardně, chronologicky.  
 
 
 
\needspace{5cm} \section{Stojí, leží, sedí} \label{stojí, leží, sedí} \katabox{Téma}{3}{2 min} 
 
V Kategorii \textbf{stojí, leží, sedí}{} hrají 3 hráči a vždy musí jeden z hráčů stát, jeden sedět a poslední ležet. 
 
\subsection{ Průběh } Jsou vybrání tři hráči, je jim zadáno téma a dána k dispozici jedna židle. Hráči zaujmou dané pozice - jeden zůstane stát, druhý si sedne a třetí lehne. Poté rozehrají improvizaci na dané téma a přitom musí zachovávat, aby byly stále zastoupeny všechny tři pozice. Hráči mohou své pozice během improvizace měnit. Není určeno pořadí v jakém musí pozice jít po sobě. Ležící hráč si může rovnou stoupnout a stejně tak i obráceně.  
 
\subsection{ Časté chyby } \begin{itemize}
\item Hráči nedávají pozor a opomenou jednu pozici
\end{itemize}
 
 
 
 
\needspace{5cm} \section{Trojitá stíhačka} \label{trojitá stíhačka} \katabox{Téma 
}{Dva týmy nebo dvě dvojice}{6x40s 
} 
Dva týmy se střídají v hraní jednoho příběhu. 
 
\subsection{ Průběh } Kategorie se hraje stejně jako \odkaz{Stíhačka}{stíhačka} nebo \odkaz{Dvojitá stíhačka}{dvojitá stíhačka}. Dojde k pěti výměnám, každá dvojice tak hraje 3x. 
 
\subsection{ Rozšíření } Narozdíl od \odkaz{Stíhačky}{stíhačka} nebo \odkaz{Dvojité stíhačky}{dvojitá stíhačka} mají hráči více času pohrát si s příběhem v určité fázi. Pokud je třeba nastolen velice hysterický problém, mohou se v něm pořádně pohrabat. 
 
 
 
 
\needspace{5cm} \section{Tři židle} \label{tři židle} \katabox{Dvouslovné téma, které inspiruje vzniklé příběhy 
}{ve výchozím stavu tři, 5 je zvládnutelné maximum 
}{5-15 minut} 
 
Tato kategorie je škálovatelná. Výchozí počet židlí je tři, proto se jmenuje \textbf{Tři židle}{} a vypráví se v ní 7 příběhů. V této kategorii divák uslyší ''2\^{}n-1'' různých příběhů.  Dá se předvádět pro diváky, ale dá se také pojmout jako cvičení na mnoho aspektů vyprávění. 
 
\subsection{Průběh}  
Připraví se \textbf{n} židlí, na které se neutrálně usadí stejný počet herců. Hráč, který sedí nedělá nic. Poté, co kategorie začne se mění příběh podle kombinace herců, kteří zrovna stojí. Herci si stoupají a sedají podobně jako v kategorii \odkaz{Toaster}{toaster}, nikoliv ale však na signál, ale pocitově. Pokud hráč stojí, tak vypráví příběh, který náleží kombinaci herců, kteří stojí s ním - každá kombinace herců tak má vlastní příběh, proto ''2\^{}n-1'' příběhů. Kategorie skončí v momentě, kdy jsou všechny příběhy dovyprávěny. 
 
\subsection{Princip}  
Kategorie je divácky zábavná díky změnám dynamice vyprávění. Herci nemají žádné \odkaz{askfor}{askfor}, mají proto úplnou svobodu ve výběru postav, žánru a délce příběhu. 
 
\subsubsection{Tipy}  
\begin{itemize}
\item  Příběhy se mohou propojovat
\item  Mohou se použít dvě židle, pokud je zadání tří příliš složité - vzniknou tři příběhy
\end{itemize}
 
\subsection{Varianty} \begin{itemize}
\item  Čtyři židle (15 příběhů), Pět židlí (31 příběhů)
\end{itemize}
 
\subsection{Podobné}  
\begin{itemize}
\item  \odkaz{Spoon river}{spoon river}
\item  \odkaz{Toaster}{toaster}
\end{itemize}
 
\subsection{ Může se plést s } \begin{itemize}
\item  \odkaz{Židle}{židle}
\item  \odkaz{Dvě židle}{dvě židle}
\end{itemize}
 
 
 
 
 
 
 
\needspace{5cm} \section{Tříhlavý song} \label{tříhlavý song} \katabox{první verš básně}{3}{cca 3 min} 
V této takřka nehrané kategorii hráči zazpívají píseň po jednom slově. 
\subsection{Průběh} Hráči se k sobě přitisknou a dohromady zazpívají píseň s tím, že se střídají po jednom slově. První verš, vyžádaný od publika, na začátku zopakují, čímž se i dostanou do nějakého tempa.  
 
\subsection{Běžné chyby} \begin{itemize}
\item Dlouhé odmlky
\item Běžné veršovací chyby
\end{itemize}
 
\subsection{Viz také} \begin{itemize}
\item Kategorie \odkaz{Opilecká píseň}{opilecká píseň} - zpěv se strřídáním po jednom verši.
\item Cvičení \odkaz{Příběh po jednom slově}{příběh po jednom slově}
\item Rozcvička \odkaz{Vtip}{vtip}
\end{itemize}
 
 
 
 
\needspace{5cm} \section{Vtip} \label{vtip} \katabox{nemusí být 
}{alespoň dva}{cca 1 min 
} 
Hráči vám po jednom slově řeknou zbrusu nový vtip. 
\subsection{Průběh} Hráči se seřadí vedle sebe, jeden začne mluvit s tím, že každý říká jen jedno slovo, drží se pořadí a odříká se vtip.  Vtip končí pointou a úklonou. 
\subsection{Varianty} \begin{itemize}
\item Předložky a spojky se nepočítají jako slovo.
\end{itemize}
 
\subsection{Viz také} \begin{itemize}
\item \odkaz{Tříhlavý song}{tříhlavý song}
\item \odkaz{Příběh po jednom slově}{příběh po jednom slově}
\end{itemize}
 
 
 
\needspace{5cm} \section{Věty, které by neměly zaznít} \label{věty, které by neměly zaznít} \katabox{Situace/činnosti (3 - 5) 
}{libovolný 
}{neomezený} 
 
 
Hráči říkají věty, které by v určitých situacích nebo na určitých místech neměly zaznít. 
 
 
\subsection{ Průběh } Podobně jako v kategorii \odkaz{1000 způsobů jak}{1000 způsobů jak} vybere \odkaz{MC}{mc} od publika několik situací např. první kroky dítěte, přijímací pohovor, na operačním sále, atd. \odkaz{Hráči}{hráč} se postaví do půlkruhu a poté vždy ten s nápadem na větu vyběhne na forbínu, kde tuto \odkaz{repliku}{replika} prezentuje a opět se zařadí do půlkruhu. Situace či prostředí se nechává do vyčerpání nápadů nebo na pokyn \odkaz{MC}{mc}. 
 
\subsection{ Varianty } \begin{itemize}
\item  \odkaz{1000 způsobů jak}{1000 způsobů jak}
\item  \odkaz{Metafory}{metafory}
\item  Některé věty se hodí do více situací a hráči toho hojně využívají.
\end{itemize}
 
\todo{doplnit video} 
 
 
 
\needspace{5cm} \section{Zvuky} \label{zvuky} \katabox{Téma 
}{2 
}{2-3 minuty} 
 
 
V kategorii \textbf{Zvuky}{} vám hráči zahrají situaci, kterou dotvoří vybraní diváci svými zvuky. 
 
 
\subsection{Průběh} \odkaz{MC}{mc} na začátku kategorie vybere z diváků dva dobrovolníky. Každý z nich dostane mikrofon a je mu přidělen hráč, kterému bude divák dotvářet zvuky podle jeho činností. Je dobré ještě před začátkem kategorie si jeden zvuk vyzkoušet nanečisto např. hráč otevírá dveře a divák doplní vrzavý zvuk pantů. V samotné kategorii hráči hrají volně na téma a snaží se svým "zvukařům"{} nabízet co nejvíce akcí, které se dají ozvučit. Divák má možnost být pouze pasivním a doplňovat zvuky akcí, které se dějí ale pokud má nápad může také zvuky nabídnout hráčům další podnět (mluvící papoušek, hlášení z rádia, atd). 
 
\subsection{ Varianty } \begin{itemize}
\item Zvuky vydávají všichni diváci najednou, bez mikrofonů
\end{itemize}
 
\subsection{ Běžné chyby } \begin{itemize}
\item Diváci zapomenou, koho mají zvučit.
\item Divákovi je dán vypnutý mikrofon.
\end{itemize}
 
 
 
 
\end{document}