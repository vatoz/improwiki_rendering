\documentclass[main.tex]{subfiles}
 
\begin{document}

\needspace{5cm}

\section{Barman song} 
\label{barman song} 



V kategorii \textbf{Barmanský song}{} se hráči vyzpívají ze zadaných problémů a barman jim přívětivě též písní poradí, jak tento problém řešit.  
 
\subsection{ Průběh } \odkaz{MC}{mc} si do diváků vyžádá několik problémů či fóbií a rozdá je hráčům. Improvizace začíná příchodem barmana do baru, který rozehrává zejména pantomimou. Po chvíli přichází do baru první hráč a může i ve svém projevu či jednání zohlednit zadaný problém či fóbii. Proběhne krátká konverzaci na jejímž konci barman podává hráči mikrofon s tím, ať se ze svého problému vyzpívá. Poté, co hráč dozpívá ihned reaguje barman svou radou, kterou též zpívá. Hudba u rady by měla být veselejšího rázu než u problému. Hráč děkuje barmanovi za radu, zaplatí za drink a odchází. Chvilku na to přichází další hráč a situace se opakuje. 
Dá se říci, že \textbf{Barman Song}{} je specializovanou variantou \odkaz{muzikálu}{muzikál}. 
 
\subsection{ Varianty } \begin{itemize}
\item Hráči si problém vymýšlí sami.
\end{itemize}
 
 
\subsection{ Běžné chyby} \begin{itemize}
\item  Hráči je podán vypnutý mikrofon
\item  Hráč zapomene zaplatit útratu
\item  Rada: utop svůj žal v alkoholu.
\end{itemize}
 
 
 
  
  
 
\needspace{5cm}
\section{Báseň od publika}
\label{báseň od publika}
\katabox{Několik(4) slov pro každého hráče}{2}{cca 6 min} 
 
 
Uslyšíte báseň, na které se sami budete částečně podílet. 
Známé též jako \textbf{Valounská nedělní}{}. 
 
\subsection{Průběh} Dva hráči jdou za dveře či si zacpou uši a \odkaz{MC}{mc} si vyžádá od publika několik slov, která napíše na papíry a položí textem dolů. 
Hráč přijde, oznámí svou báseň (např. Sbírka:Autor:Báseň) a následně jsou mu papíry otočeny a on musí slova co nejrychleji a přitom logicky zapracovat do své básně , 
která by měla být v nějaké logické veršovací struktuře (např \odkaz{ABAB}{abab} či AABB), a zároveň by měla vyhovovat názvu, který hráč anoncoval. 
 
 
\subsection{ Varianty } \begin{itemize}
\item Píseň od publika
\item Karty se otáčí průběžně
\end{itemize}
 
 
 
 
\needspace{5cm} \section{Co tím chceš jako říct} \label{co tím chceš jako říct} \katabox{Téma, prostředí 
}{2+ 
}{volně, více jak 3 minuty} 
 
V kategorii \textbf{Co tím chceš jako říct}{} uvidíte spoustu nedorozumění. 
 
 
\subsection{ Průběh } Volně se rozehrává příběh. \odkaz{Hráči}{hráč} jsou ale v silně výbušných postavách. V momentě, kdy zazní nějaká narážka na citlivé téma od jiné postavy, začínají se slovy "Co tím chceš jako říct?"{} svůj agresivní výlev, ve kterém by na sebe měli, podobně jako při zpěvu v kategorii \odkaz{Mikrofon}{mikrofon}, něco prozradit. Po výlevu scénka pokračuje dál. 
 
\subsubsection{ Příklad výlevu } A: Ta barva zdi se mi moc líbí. 
B: Co tím chceš jako říct? 
B: Že je to špatně natřený? 
B: Tím chceš říct, že bys to natřel líp? 
B: Že jsem jako komunista, když mám červenej pokoj? 
B: Tím chceš říct, že si nevážíš práce soudruhů? 
... 
A: Ne, tím jsem chtěl říct jenom, že je fakt pěkná. 
 
\subsection{ Časté chyby } \begin{itemize}
\item  Výlev není neutrálně ukončen
\end{itemize}
 
 
 
 
 
\needspace{5cm} \section{Dabing ABC} \label{dabing abc} \katabox{Téma 
}{3 
}{neomezený} 
 
V kategorii \textbf{Dabing ABC}{} vám 3 hráči zahrají na Vaše téma, ale tak, že za každého hráče mluví jeho kolega. 
 
\subsection{Průběh} Připraví se tři hráči. Improvizaci rozehrává jeden hráč (A), ale pokud mluví, pouze otevírá pusu a jeho promluvy dabuje druhý hráč (B), který je zatím mimo scénu. Po nějaké chvíli se do děje zapojí hráč B, který je ale dabován třetím hráčem (C). Hráč B stále dabuje hráče A. Nakonec se na scéně objeví i třetí hráč (C), kterého dabuje hráč A. Pro lepší iluzi je dobré, aby hráč, který je právě dabován otočil hlavu do diváků, zatímco jeho dabér mluví do jeviště. Mnohdy vznikají zmatené situace, kdy hráč začne mluvit a jeho kolega si neuvědomí, že by měl otevírat pusu, ale také naopak. Dabovaný hráč může otevřít pusu a jeho dabér musí začít mluvit. Vtipnost této kategorie také spočívá v tom, že dabér může svého svěřence tím, co řekne do jisté míry ovládat. 
 
\subsection{ Varianty } \begin{itemize}
\item Dabing ABCD - hrají 4 hráči.
\item Ostatní hráči mohou ztvárňovat předměty.
\end{itemize}
 
\subsection{ Běžné chyby } \begin{itemize}
\item Hráči nedávají prostor svým kolegům a stále mluví.
\item Hráči neotvírají pusu když jejich dabér mluví.
\end{itemize}
 
 
 
 
 
\needspace{5cm} \section{Dabing filmu} \label{dabing filmu} \katabox{Téma 
}{2-3 
}{omezen ukázkou} 
 
V kategorii \textbf{Dabing filmu}{} je promítána filmová ukázka bez zvuku, kterou hráči předtím neviděli a naživo ji před Vámi dabují. 
 
\subsection{ Průběh } Scéna se připraví na promítání (příprava plátna, projektoru). \odkaz{MC}{mc} určí počet hráčů a jejich pohlaví, podle vybrané filmové ukázky. Hráči dostanou mikrofon, v sále se zhasne a je spuštěna projekce ukázky. Hráči se musí co nejpřesněji strefovat do úst své postavy na plátně a současně vytvářet smysluplný dialog. Kategorie končí s koncem ukázky. Ne ve všech záběrech některá z postav mluví, pak může mluvit libovolná z nich. 
 
Po skončení kategorie je vhodné nechat před hlasováním hráče, aby některou ze svých \odkaz{replik}{replika} představili své postavy. 
 
\subsection{ Varianty } \begin{itemize}
\item Dabing filmu může být pokračováním předešlé kategorie, pokud hráči dojdou do prostředí, o němž \odkaz{MC}{mc} ví, že se v něm odehrává připravená filmová ukázka.
\end{itemize}
 
\subsection{ Běžné chyby } \begin{itemize}
\item Postava na plátně mluví a dabující hráč nemluví a naopak.
\item Hráči dostanou vypnutý mikrofon.
\item Hráči dabují oba stejnou postavu.
\item Promítaná ukázka obsahuje záběry, na nichž většinou není vidět postava, která právě mluví.
\end{itemize}
 
 
 
 
 
\needspace{5cm} \section{Dabovaná} \label{dabovaná} \katabox{Téma 
}{4 
}{2 minuty} 
 
 
V kategorii \textbf{Dabovaná}{} vám hráči na scéně zahrají na vaše téma, ale pouze pantomimou. Slova jim do úst vkládají dabéři, kteří jsou mimo scénu. 
 
 
\subsection{Průběh} Připraví se sudý počet hráčů, ideálně dva z každého týmu. Jeden se stává dabovaným, který poté hraje a druhý se stává dabérem, který za hrajícího kolegu mluví (oba jsou ze stejného týmu).  
Dabovaní poté na scéně rozehrají improvizaci na zadané téma, ale pouze pantomimou. Jejich promluvy pronášejí dabéři mimo scénu zpravidla před forbínou, aby na své kolegy dobře viděli. Hráči na scéně musí dávat pozor, co bylo řečeno a podle toho také jednat a naopak dabéři musí zohledňovat to, co jejich kolegové na scéně hrají. Pro správné fungování kategorie je důležité, aby dabovaným hráčům bylo vždy dobře vidět na obličej. Pokud se hráč otočí zády, jeho dabér ho nemůže dobře dabovat. 
 
Na konci je vhodné, když se jednotlivé hlasy ještě jednou představí. 
 
\subsection{ Varianty } \begin{itemize}
\item Hraje 6 nebo 8 hráčů (3, 4 dabovaní a 3, 4 dabéři).
\item Dabéři mluví do mikrofonů.
\end{itemize}
 
\subsection{ Běžné chyby } \begin{itemize}
\item Dabovanému hráči není vidět do obličeje.
\item Dabér nepostřehne, že jeho kolega otevírá pusu a nemluví za něj.
\item Jeden dabér stále mluví a nedá tak prostor svému kolegovi.
\item Dabéři mluví přes sebe a není jim rozumět.
\end{itemize}
 
 
 
 
 
\needspace{5cm} \section{Diapozitivy} \label{diapozitivy} \katabox{Téma 
}{6+ 
}{3 minuty} 
 
\odkaz{Hráči}{hráč} vám předvedou prezentaci s diapozitivy na dané téma. 
 
\subsection{ Průběh } Jeden \odkaz{hráč}{hráč} z každého týmu se připraví dopředu na prezentaci, kterou na začátku krátce uvedou. Scéna je rozdělena na přední část, kde stojí moderátoři prezentace a zadní část, kde se objevují diapozitivy. Slovem "Cvak"{} prezentující přepínají obraz. Při přepnutí obrazu naběhne neurčený počet hráčů do zadní části a zastaví se ve \odkaz{štronzu}{štronzo}. Moderátoři poté vysvětlují, co je na obrazu a inspirují se pozicí, ve které jsou diapozitivy. 
 
\subsection{ Technické výrazivo } \begin{itemize}
\item \textbf{cvak}{} - pro přepnutí snímku  - hráči změní pozici, vyskočí ze snímku či do něj vskočí a strnou ve \odkaz{štronzu}{štronzo}
\item \textbf{nepatřící snímek}{}
\item \textbf{dva snímky přes sebe}{} - část hráčů na následující cvaknutí odbíhá, je možné že se snímek s jejch pozicí ještě vrátí.
\item \textbf{smítko/prach/moucha}{} - moderátor odfoukne jednoho z hráčů ze snímku pryč
\item \textbf{stranové převrácení}{} - hráči se přeorganizují dle svislé osy
\item \textbf{převracení vzhůru nohama}{} - hráči se převrátí dle vodorovné osy, obvykle je nutný nějaký stoj na rukách apod.
\item \textbf{zaostření}{} - hráči se přiblíží k sobě i k divákům, zvýrazní mimiku obličeje
\end{itemize}
 
\subsection{ Varianty } \begin{itemize}
\item Scénka začíná s  "úvodním slajdem"{} spuštěným
\item PowerPointová prezentace - v prezentaci se mohou objevit i videa a ozvučené obrázky
\item \odkaz{Kramářská píseň}{kramářská píseň} - "zpívaná verze"
\end{itemize}
 
\subsection{ Běžné chyby } \begin{itemize}
\item Úvod trvá zbytečně dlouho
\item \odkaz{Hráči}{hráč} chodí v obrazu diapozitivů nebo do něj šahají
\item Popletené technické výrazivo (např. hráči \textbf{zoomují}{}, což diaprojektor neumožňuje)
\end{itemize}
 
 
 
 
 
\needspace{5cm} \section{Dopis} \label{dopis} \katabox{Obvykle dvě postavy(kdo a komu píše) 
}{2, následně neomezeno 
}{30 vteřin, 3 minuty 
} 
Dva hráči na střídačku přečtou \textbf{Dopis}{}, dle kterého je následně sehrána celá scéna. 
\subsection{Průběh} Před začátkem kategorie, vyzve \odkaz{MC}{mc} jednoho hráče z každého týmu ať předstoupí a společně přečtou dopis na téma od diváků (či s vyžádaným adresátem a pisatelem). 
Střídají se po cca 1/2 věty.  Většinou na to bývá 20-30 vteřin.  
 
Po přečtení dopisu MC zahájí normální kategorii s časovým omezením, která se může jakkoli inspirovat dopisem (děj se může odehrávat před dopisem, po dopisu, týkat jednoho charakteru z dopisu, týkat se události z dopisu, týkat se zmíněné věci z dopisu atd. ). 
\subsection{Běžné chyby} \begin{itemize}
\item Ignorování dopisu v druhé části
\end{itemize}
 
 
 
 
\needspace{5cm} \section{Duet} \label{duet} \katabox{jednoslovné téma nebo předmět pro každou dvojici 
}{dvojice 
}{volný, do 30 s 
} 
 
V duetech spolu každá dvojice zahraje krátkou scénku na dané téma. Ukončuje se pohledem do diváků. 
 
\subsection{ Průběh } Dvojice stojí na okrajích pódia zády k sobě. \odkaz{MC}{mc} si vyžádá téma/předmět pro každou dvojici. Kreativita není omezena, ale nejčastěji se hraje \odkaz{pantomima}{pantomima} doplněná zvuky. Scénka končí v momentě, kdy se oba podívají nejdříve na sebe a poté do publika. 
Jeden \odkaz{hráč}{hráč} může hrát \odkaz{předmět}{předmět} zadaný od diváků a druhý s ním interaguje, ale možnosti jsou samozřejmě širší. 
 
\subsection{ Varianty } \begin{itemize}
\item \odkaz{Kvartet}{kvartet}
\end{itemize}
 
\subsection{ Běžné chyby } \begin{itemize}
\item Zanedbání očního kontaktu
\end{itemize}
 
 
 
 
\needspace{5cm} \section{Duál} \label{duál} \katabox{Téma}{2 a více 
}{2 min. 
} 
 
V kategorii \textbf{Duál}{} hráči hrají volnou improvizaci ve dvou jazycích, mezi kterými přepínají diváci. 
 
\subsection{Průběh} Hráči rozehrají volnou improvizaci inspirovanou tématem a mluví přitom v češtině. Diváci mají možnost kdykoli během improvizace nahlas vykřiknout slovo "Duál!", což je pro hráče znamení, že musí okamžitě přejít do druhého jazyka - šumlované či zpátky do češtiny. Šumlovaná je smyšlený jazyk, též známý jako švojština či \odkaz{gibberish}{gibberish}. Hráči ani diváci tak přímo nerozumí tomu, co říká právě mluvící hráč. Hráči by se tak měli soustředit také na pantomimický doprovod své mluvy. Diváci mohou vykřikovat dle své libosti, počet přepnutí není omezen.  
 
\subsection{Běžné chyby} \begin{itemize}
\item Divák křičí, ale hráč ho neslyší. \odkaz{MC}{mc} může povel hráčům zopakovat.
\item Diváci nechávají hráče příliš dlouho v jednom jazyce.
\item Opožděná reakce na přepnutí (doříkávání věty)
\item Nerealistické začínání nové věty od začátku po přepnutí
\end{itemize}
 
\subsection{Varianty} \begin{itemize}
\item Přepínání provádí \odkaz{MC}{mc}
\end{itemize}
 
\subsection{Podobné kategorie} \begin{itemize}
\item  \odkaz{Šumlovaná}{šumlovaná}
\item  \odkaz{Šumlovaná se simultánním překladem}{šumlovaná se simultánním překladem}
\end{itemize}
 
 
 
 
\needspace{5cm} \section{Dva na pět} \label{dva na pět} \katabox{prostředí 
}{dva 
}{neomezený} 
 
V kategorii \textbf{Dva na pět}{} dva hráči ztvární alespoň sedm postav. 
 
\subsection{Průběh} \odkaz{MC}{mc} si od diváků vyžádá prostředí, kde se bude improvizace odehrávat. V improvizaci hrají pouze dva hráči, kteří se připraví a poté rozehrají za první dvě postavy. Kdykoli poté je jedním hráčem pojmenována či zavolána nová postava, druhý hráč musí postavu nadále ztvárňovat, ale současně si zachovává i postavy, které hrál předtím. Jeden hráč tak v improvizaci hraje za 1-6 postav, ideální samozřejmě je, když jsou počty postav na hráče vyrovnané. 
 
\subsection{ Běžné chyby } \begin{itemize}
\item  Hráč zapomene na některou ze svých postav.
\item  Hráč neurčí druhému další postavy.
\item  Hráč využívá k posunu děje jen jednu ze svých postav.
\end{itemize}
 
 
 
 
 
 
 
 
\needspace{5cm} \section{Dvojitá stíhačka} \label{dvojitá stíhačka} \katabox{Téma 
}{Dva týmy}{4x40s 
} 
 
Příběh začnou hrát hráči pouze jednoho týmu, jsou pak přesně vystřídáni týmem druhým  
 
\subsection{ Průběh } Průběh je podobný jako v kategorii \odkaz{Stíhačka}{stíhačka}, pouze rozšířen o dvě výměny hráčů. 
 
\subsection{ Rozšíření } Narozdíl od obyčejné \odkaz{stíhačky}{stíhačka}, nemusí \odkaz{hráči}{hráč} prvního týmu hledat problém. Čas navíc by měli využít k \odkaz{expozici}{expozice}. Po první výměně hráči druhého týmu nachází problém. Po druhé výměně \odkaz{hráči}{hráč} prvního týmu problém vyvrcholí a po poslední výměně \odkaz{hráči}{hráč} druhého týmu problém vyřeší. Obvykle téma zazní až po poslední výměně. 
 
\subsection{ Ukazování času } \odkaz{Pomocný rozhodčí}{pomocný rozhodčí} \odkaz{ukazuje}{ukazování času} v prvních třech blocích pouze konec, jen v posledním bloku všechny časové intervaly. 
 
 
 
 
 
\needspace{5cm} \section{Dům a zahrada} \label{dům a zahrada} \katabox{Téma, prostředí 
}{libovolný (ideálně 3 na každé straně) 
}{neomezený} 
 
 
V kategorii \textbf{Dům a Zahrada}{} vám hráči zahrají improvizaci v dvou těsně sousedících prostředích. 
 
 
\subsection{Průběh} Jeviště se ve středu pomysleně rozdělí na dvě poloviny a každé polovině je zadáno prostředí (dům, zahrada). Co je hlavní – prostředí spolu nesousedí středem jeviště, ale jeho portály, kraji. Každý tým hraje jen na své polovině. Na obou stranách hraje stejný počet hráčů, a to tak, že dva soupeřící hráči hrají jednu postavu. Na začátku kategorie nastoupí hráči, kteří hrají libovolně do domu nebo do zahrady, tak, aby byly všechny postavy zastoupené. Rozehraje se improvizace ve které se hráči snaží co nejvíce pracovat s logickým přecházením z domu do zahrady a naopak. Hráči si musí hlídat svého partnera, se kterým hrají stejnou postavu a přebírat jeho akce. 
 
\subsection{ Varianty } \begin{itemize}
\item Vyberou se jiná dvě prostředí, která spolu sousedí (bazén, sauna/restaurace, kuchyně)
\end{itemize}
 
\subsection{ Běžné chyby } \begin{itemize}
\item Hráč zapomene sledovat svého kolegu a zapomene nastoupit nebo nepřevezme akci svého kolegy.
\item Hráči na sebe do sousedních prostředí volají a ohlížejí se přes střed jeviště.
\end{itemize}
\subsection{Podobné kategorie} \begin{itemize}
\item \odkaz{Schizofrenie}{schizofrenie} kde také dva hráči hrají stejnou postavu
\end{itemize}
 
 
 
 
\needspace{5cm} \section{En face} \label{en face} \katabox{Téma 
}{4-6 
}{3 min 
} 
 
Hráči Vám přednesou příběh, budou však zachovávat naprosto strnulé tváře. 
 
\subsection{ Průběh } Hráči se postaví do řady (podobně jako v \odkaz{Spoon river}{spoon river}) střídavě aby hráči stejného týmu nebyli vedle sebe. Během improvizace se hráči nesmějí hýbat a nesmějí mimikou vyjadřovat emoce. Jediným výrazovým prostředkem zůstává slovo a zvuky. Kategorie běžně začíná nejprve zvuky prostředí, kde se bude improvizace odehrávat poté se objeví i postavy. Hráči si po celou dobu zachovávají jednu danou postavu. 
 
\subsection{ Běžné chyby } \begin{itemize}
\item  Hráči se sami ztratí v ději
\item  Hráči mění výraz tváře (smějí se)
\item  Hráč se vůbec nezapojí do příběhu
\item  Hráč nepustí ke slovu ostatní hráče
\end{itemize}
 
\subsection{ Viz také } Má blízko ke kategoriím \odkaz{Spoon river}{spoon river}, \odkaz{Rozhlasová hra}{rozhlasová hra} či \odkaz{Vysílání FM}{vysílání fm}. 
 
 
 
 
\needspace{5cm} \section{Filmové žánry} \label{filmové žánry} \katabox{Téma, několik existujících filmových a divadelních žánrů 
}{2 + 2n 
}{neomezený, 5 - 15 minut} 
 
 
\odkaz{Hráči}{hráč} v této kategorii předvedou krátké scénky podle zvoleného \odkaz{žánru}{:kategorie:žánry}. 
 
 
\subsection{Průběh} Začíná se volně - bez \odkaz{žánru}{:kategorie:žánry}, dvojice spolu začne scénku na dané téma. \odkaz{MC}{mc} přepíná žánry ze seznamu, který má zapamatovaný nebo napsaný. V každém žánru se zahraje krátká scénka a vše co se odehrálo se přelévá do dalšího žánru. Ostatní \odkaz{hráči}{hráč} mohou naskakovat jako \odkaz{rekvizity}{rekvizita} a vedlejší postavy. Vedlejší postavy by měly po změně žánru opustit scénu.  
 
Na \odkaz{zápasech}{zápas} se obvykle vystřídají všichni hráči v dvojicích. \odkaz{MC}{mc} dává povel ke střídání. 
 
\subsection{ Varianty } \begin{itemize}
\item Není žádná výchozí dvojice a postavy volně přicházejí a odcházejí
\end{itemize}
 
\subsection{ Běžné chyby } \begin{itemize}
\item Běžné chyby, které se vyskytují v \odkaz{žánrech}{:kategorie:žánry}
\item Vedlejší postavy upozadí hlavní dvojici
\item MC nerespektuje žánry navržené diváky
\end{itemize}
 
 
 
 
 
\needspace{5cm} \section{Kam zmizel ten starý song} \label{kam zmizel ten starý song} \katabox{Téma, povolání/specializace 
}{libovolný (3 a více) 
}{cca 3 minuty na každou píseň} 
 
 
V kategorii \textbf{Kam zmizel ten starý song}{} moderátor s hostem vzpomínají na písničky, které hosta formovaly. 
 
 
\subsection{Průběh} Moderátor a host spolu krátce hovoří o specializaci/životě hosta. 
Vždy je přivedena řeč na nějakou písničku, kterou moderátor "pustí"{} - nechá hráče, aby ji zazpívali. 
Píseň bývá pojmenována stylem, názvem, případně může být i nějaká citace z ní, která by pak měla také zaznít. 
Zpívajícímu hráči/hráčům ostatní dělají \odkaz{chór}{chór}. 
 
\subsection{ Varianty } \begin{itemize}
\item Zpívají dvojice, každý hráč z jiného týmu.
\item Každý hráč zpívá sám, na zápasech je pak vhodné nechat každý tým shodný počet písní.
\end{itemize}
  
 
\subsection{ Běžné chyby } \begin{itemize}
\item Rozkecávání moderátora s hostem nad únosnou mez.
\item Moderátor s hostem určí obskurní žánry
\end{itemize}
 
 
 
 
 
\needspace{5cm} \section{Komiks} \label{komiks} \katabox{Téma 
}{4 
}{cca 1 minuta} 
 
Hráči vám ukáží a přečtou \textbf{Komiksový}{} strip. 
 
\subsection{Průběh} Dva hráči zaujmou pózy prvního obrázku stripu. 
Druzí dva hráči "přečtou"{} text komiksového okénka - ta postava z obrázku, která má otevřená ústa, mluví, pokud má ústa zavřená, jedná se o myšlenkovou bublinu. 
Následně se totéž opakuje pro druhý a třetí obrázek stripu (či další, pokud \odkaz{MC}{mc} určil jinak. 
Je záhodno používat i komiksové výrazivo typu BOOM apod. 
 
 
 
 
\needspace{5cm} \section{Kvartet} \label{kvartet} \katabox{jednoslovné téma nebo předmět 
}{4 
}{volný 
} 
 
 
Čtyři hráči zahrají krátkou improvizaci na zadané téma a zakončí ji společným pohledem do publika. 
 
 
\subsection{Průběh}  
Podobné jako \odkaz{duet}{duet}, avšak ve čtyřech hráčích. Každý hráč začíná zády ke středu v jednom rohu jeviště. Vše platí jako u duetu - na povel \odkaz{MC}{mc} se všichni hráči najednou otočí a rozehrají společně improvizaci, ve které by měla být více zastoupena pantomima než řeč. Kategorie je pro hráče náročná i co se vnímání ostatních hráčů týče. Doporučuje se více klást důraz na kontrast, ať v tělovém napětí, zvucích, statusu, či v čemkoli jiném. Improvizace končí pohledem všech hráčů na sebe a poté společně do diváků. 
 
\subsection{ Běžné chyby } \begin{itemize}
\item Improvizace se příliš protáhne a nedaří se najít její závěr
\end{itemize}
 
 
 
\needspace{5cm} \section{Loutky} \label{loutky} {{todo|zpřehlednit průběh, youtube}} 
 
\katabox{Téma 
}{sudý počet, 4 a více 
}{3 minuty} 
 
V kategorii \textbf{Loutky}{}  uvidíte krátké loutkové představení. 
 
\subsection{Příprava scény} Na scéně se postaví židličky vedle sebe. Na každé židli stojí hráč - loutkovodič, pod každou židli si zaleze klečící shrbený hráč - loutka. 
 
\subsection{Průběh} Loutkovodič zdánlivě vede loutku, veškeré pohyby ale přitom určuje loutka, kterou loutkovodič stínuje, a dabuje. 
 
\textbf{Loutka}{} po celou dobu drží strnulý výraz tváře (škleb) a pozici prstů a dlaně, může hýbat (předstírat že je vedena) pouze rukama, či v případě klidové pozice rukou podél těla otáčet celým trupem. 
 
\textbf{Loutkovodič}{} drží loutku buď  za "drátek", s ním manipuluje, když chce loutkou otočit na stranu či ji položit zpět. 
Jinak manipuluje pouze "provázky"{} připevněnými k zápěstí loutky.   
  
\subsection{ Běžné chyby } \begin{itemize}
\item  Zamotané provázky - loutkovodiči pohybují s loutkami tak, že by se jim provázky bývaly byly zamotaly.
\end{itemize}
 
 
\subsection{Varianty} Loutku může Loutkovodič v průběhu improvizace "zandat"{} a znovu vytáhnout. Pokud mezitím změní obličej a pozici prstů a dlaně, jde o novou postavu. Pokud ji nezmění, pokračuje příběh na novém místě.  
 
 
 
\needspace{5cm} \section{Love song} \label{love song} \katabox{Dobrovolník z publika, židle 
}{2 
}{neomezený, 3 - 5 minut (podle počtu hráčů)} 
 
 
V kategorii \textbf{Love song}{} je dobrovolníkovi z publika na místě složena a zazpívána zamilovaná píseň. Kategorie je \odkaz{porovnávací}{:kategorie:porovnávací kategorie}. 
 
 
\subsection{Průběh} \odkaz{MC}{mc} vybere jednoho člověka z publika a zeptá se ho na jméno. Hráči se střídavě zeptají na určený počet otázek (3-5). Odpovědi na jejich otázky slouží jako inspirace pro text písně. Poté v doprovodu \odkaz{hudebníka}{hudebník} každý z hráčů zazpívá svoji milostnou/zamilovanou píseň. Dobrovolník následně vybírá, která píseň byla lepší. 
 
\subsection{ Varianty } \begin{itemize}
\item \odkaz{Break-up song}{break-up song}
\end{itemize}
 
\subsection{ Běžné chyby } \begin{itemize}
\item Zapomenutí jména, komu se zpívá
\item Nevyužití informací od člověka
\item Divné či ano/ne otázky
\end{itemize}
 
 
 
 
 
 
 
\needspace{5cm} \section{Metafory} \label{metafory} \katabox{Dva názvy předmětů 
}{libovolný, obvykle se zapojí všichni 
}{volný, dokud nedojdou nápady} 
 
Hráči spojují podobnosti dvou různých předmětů nebo činností - vyvářejí metafory. Zadání může být '''"Děti jsou jako řidičák"'''. 
 
\subsection{Průběh}  
\odkaz{MC}{mc} vybere od publika více zadání metafor. Poté se hráči postaví do půlkruhu a střídají se. Každý hráč, který vyběhne, řekne svojí metaforu. Metafory si drží formu ''{Něco} je jako {něco}, {vysvětlení}''. Obvykle MC po chvíli činnost vymění, zadá jinou. 
 
\subsubsection{Příklad metafory}  
''Děti jsou jako řidičák. Tak dlouho se na ně těšíš a když je máš, tak zjistíš, že nemáš na auto.'' 
 
\subsection{ Varianty } \begin{itemize}
\item  \odkaz{Tisíc způsobů jak}{tisíc způsobů jak}
\item  \odkaz{Věty, které by neměly zaznít}{věty, které by neměly zaznít}
\end{itemize}
 
\subsection{ Běžné chyby } \begin{itemize}
\item  Sklouzávání k záchodovému humoru
\item  Sklouzávání k záporům "Něco není jako něco"
\end{itemize}
 
 
 
 
 
\needspace{5cm} \section{Mikrofon} \label{mikrofon} \katabox{Téma 
}{libovolný 
}{neomezený,volný} 
 
 
V kategorii \textbf{Mikrofon}{} hráči rozehrají improvizaci a na divácký pokyn, vykřiknutí slova "Mikrofon!"{} ten který jako poslední promluvil zazpívá písničku, ve které představí své vnitřní touhy, bolesti a záměry. 
 
 
\subsection{Průběh} Hráči rozehrají volnou improvizaci na dané téma. Diváci mají možnost kdykoli během improvizace zvolat slovo "\textbf{Mikrofon}{}". \odkaz{Hráč}{hráč}, který mluvil naposled nebo hráč, kterého určí \odkaz{MC}{mc} dostane mikrofon a zazpívá píseň, ve které by mělo zaznít něco, co o jeho postavě ještě nevíme - její touhy, bolesti, plány, atd. Po krátké písni se vracíme do děje, před písní a příběh pokračuje.  
 
\subsection{ Varianty } \begin{itemize}
\item Zpívají dva hráči současně.
\item Heslo "Mikrofon!"{} mohou diváci vykřiknout kdykoli během večera a daný hráč, zazpívá píseň bez ohledu na právě hranou kategorii.
\end{itemize}
 
\subsection{ Běžné chyby } \begin{itemize}
\item Hráči je dán vypnutý mikrofon.
\item Hráč zpívá o tom, co o jeho postavě už víme.
\item Diváci vykřiknou mikrofon se začátkem improvizace.
\end{itemize}
 
 
\subsection{ Podobné kategorie } \begin{itemize}
\item \odkaz{Monolog}{monolog}
\end{itemize}
 
 
 
 
 
\needspace{5cm} \section{Monolog} \label{monolog} \katabox{Téma 
}{libovolný 
}{neomezený,volný} 
 
 
V kategorii \textbf{Monolog}{} hráči rozehrají improvizaci a na Váš pokyn, vykřiknutí slova "Monolog!"{} Vám jeden z nich v monologu představí své vnitřní pocity. 
 
 
\subsection{Průběh} Hráči rozehrají volnou improvizaci na dané téma. Diváci mají možnost kdykoli během improvizace zvolat slovo "\textbf{Monolog}{}". \odkaz{Hráč}{hráč}, který mluvil naposled nebo hráč, kterého určí \odkaz{MC}{mc} dostane mikrofon a řekne monolog, ve kterém by mělo zaznít něco, co o jeho postavě ještě nevíme - její touhy, bolesti, plány, atd. Po monologu se vracíme do děje a příběh pokračuje.  
 
\subsection{ Běžné chyby } \begin{itemize}
\item \odkaz{Hráči}{hráč} je dán vypnutý mikrofon.
\item \odkaz{Hráč}{hráč} svým monologem nerozvíjí příběh.
\end{itemize}
 
 
\subsection{ Podobné kategorie } \begin{itemize}
\item \odkaz{Mikrofon}{mikrofon}
\end{itemize}
 
 
 
 
\needspace{5cm} \section{Nátlak (kategorie)} \label{nátlak (kategorie)} \katabox{úkol pro jednoho hráče 
}{2 
}{3 minuty, či do splnění úkolu} 
 
V kategorii \textbf{Nátlak}{} má jeden hráč za úkol dotlačit druhého k provedení úkolu. 
 
\subsection{Průběh} \odkaz{MC}{mc} přidělí jednomu hráčovi činnost, ke které  má tento dohnat druhého, který mezitím neposlouchal. Dál se rozjíždí děj stejně jako ve \odkaz{volné}{volná}, ale první hráč má stále na mysli svůj úkol.  
 
\subsection{Příklad} Má-li hráč donutit druhého ke zpěvu, může např. hrát dítě ukládající se ke spánku nebo hudebníka na rockovém koncertě. 
  
\subsection{ Běžné chyby } \begin{itemize}
\item  "tlačící hráč"{} přímo řekne tomu druhému.
\end{itemize}
 
 
 
 
 
 
 
 
\needspace{5cm} \section{Poetická} \label{poetická} \katabox{Téma 
}{2 
}{3 minuty} 
 
V kategorii \textbf{Poetická}{} smí \odkaz{hráči}{hráč} mluvit pouze veršovaně. 
 
\subsection{ Průběh } V podstatě se jedná o kategorii \odkaz{volná}{volná}. Jediným omezením \odkaz{hráčů}{hráč} je povinnost skládat věty do veršů. Hráči by měli průběhu kategorie zahrát krátký příběh. 
Do kategorie mohou nabíhat ostatní hráči jako \odkaz{postavy}{postava} nebo \odkaz{rekvizity}{rekvizita}. 
 
Krásné je, když hráč nerýmuje jen "sám na sebe"{} ale spolupracuje s druhým hráčem a doříkává za něj verše. 
 
\subsection{ Běžné chyby } \begin{itemize}
\item Nikomu se nechce, je opožděný začátek kategorie a \odkaz{rozhodčí}{rozhodčí} je nucen \odkaz{písknout}{pískání} \odkaz{plynulost hry}{plynulost hry}.
\item \odkaz{Hráči}{hráč} se blokují tím, že se přestanou hýbat
\item Hráči zapomenou veršovat
\end{itemize}
 
\subsection{Varianty} \begin{itemize}
\item \odkaz{Divadlo poezie}{divadlo poezie}
\end{itemize}
 
 
 
 
 
\needspace{5cm} \section{Polovina času} \label{polovina času} \katabox{téma}{2 a více 
}{Např 60, 30, 15, 7.5, 3.75, 1.8   
} 
 
V \textbf{Polovině času}{} hráči zahrají krátký příběh, který následně zahrají rychleji -  za polovinu času , následně ještě rychleji… 
 
\subsection{ Průběh } Dva hráči rozehrávají příběh, který by měl být hodně akční. Ostatní hráči se zapojují hlavně jako rekvizity ale mohou i jako další postavy. Po uplynutí časového limitu se celý příběh přehrává znovu, s tím, že by mělo zaznít a proběhnout vše, co bylo v původní, nejdelší verzi. Toto se opakuje až do nejkratšího časového úseku. Proto je nutné zjednodušovat, zkracovat dialogy, odebírat slovní vatu. Dobré je také spíše jednat, než mluvit, akce se zkrátit dají rozvité souvětí už tolik ne. 
 
\subsection{ Varianty } \begin{itemize}
\item Po předvedení na 5 vteřin se dá hlasovat, zda pokračovat se zkracováním, či scénu sehrát v deseti vteřinách pozpátku.
\item Když začnete na 64 vteřinách, je to mnohonásobně jednodušší počítat.
\item Nejkratší časový úsek se nesnaží do sebe vměstnat obsah celého příběhu, ale jen jeho zakončení.
\end{itemize}
 
\subsection{ Běžné chyby } \begin{itemize}
\item V kratších časech se všichni chtějí prosadit a vzniká nevzhledná kakofonie.
\item Hráči v nejdelším časovém limitu jen stojí a mluví.
\end{itemize}
 
 
 
 
 
\needspace{5cm} \section{Poslední věta} \label{poslední věta} \katabox{Téma a prostředí 
}{dvojice (2 a více) 
}{do nalezení pointy, 5-10 minut} 
 
 
Každá dvojice sehraje scénku ve svém \odkaz{prostředí}{prostředí}, \odkaz{MC}{mc} mezi prostředími/dvojicemi přepíná. 
 
 
\subsection{Průběh} \odkaz{MC}{mc} od diváků vybere prostředí pro každou dvojici. Rozehrává se scénka, dokud nedojde k přepnutí. Scénka v dalším prostředí začíná poslední větou z předchozího. Po přepnutí prostředí by poslední věta měla zaznít od \odkaz{MC}{mc} spolu s prostředím. 
 
\subsection{ Varianty } \begin{itemize}
\item Závěr se hledá pouze v jednom prostředí
\item Všechny dvojice si najdou závěr
\end{itemize}
 
\subsection{ Běžné chyby } \begin{itemize}
\item Zapomenutí první věty
\item \odkaz{MC}{mc} přepíná na úkor příběhu ve scénkách
\end{itemize}
 
 
 
 
 
\needspace{5cm} \section{Počet slov} \label{počet slov} \katabox{Téma,počet pro každého hráče (ten může být přidělený) 
}{libovolný (3 a více) 
}{3 minuty} 
 
 
V kategorii \textbf{Počet slov}{} známé též jako \textbf{7 slov}{} hráči  budou přesně dodržovat počet slov ve větě. 
\subsection{Příprava} \odkaz{MC}{mc} určí každému hráči, kolik slov smí používat - typicky 1,3,5,7.  
 
\subsection{Průběh} Sehrává se scéna, každý hráč musí v každé své větě/replice použít přesně tolik slov (včetně předložek a spojek) , jako mu bylo přiděleno. 
Často se některý divák pověří, aby kontroloval zejména ty delší varianty. 
 
 
\subsection{ Běžné chyby } \begin{itemize}
\item  Přepočítání se
\item  Nekompletní věty
\end{itemize}
 
 
 
\needspace{5cm} \section{Prskavka} \label{prskavka} \katabox{jednoslovné téma pro každého hráče 
}{libovolný, každý sám za sebe 
}{30 vteřin 
} 
V kategorii \textbf{Prskavka}{} má každý hráč sólový výstup na 30 vteřin a co nejlépe ztvárňuje Vaše jednoslovné téma. 
 
\subsection{Průběh} Prskavka se zaměřuje na tvorbu jednotlivce. 
Hráči stojí zády k publiku, \odkaz{MC}{mc} si vyžádá jednoslovné téma (nejčastěji se vybírají předměty), poté vybere hráče, který bude na toto téma hrát a zahájí improvizaci. Hráč má pak právě 30 vteřin (doba hoření idealizované prskavky) na ztvárnění tématu, bez omezení - může zpívat, veršovat, hrát více postav nebo (nejčastěji) hrát pantomimu. 
 
\subsection{ Varianty } \begin{itemize}
\item \odkaz{Prskavka ala šanson}{prskavka ala šanson}
\item Během představeních v okolí vánočních svátků se dá jako \odkaz{časomíra}{ukazování času} použít místo stopek reálná prskavka.
\end{itemize}
 
 
 
 
 
 
 
 
 
\needspace{5cm} \section{Pyramida} \label{pyramida} \katabox{Prostředí, Téma nebo činnost 
}{4-6 
}{volný} 
 
V \textbf{pyramidě}{} se hraje serie příběhů s narůstajícím počtem hráčů, pokaždé v novém prostředí. Následně se začne počet hráčů snižovat a prostředí se vrací. 
 
 
\subsection{ Průběh } Kategorie je podobná cvičení \odkaz{Roztočený brankář}{roztočený brankář}. Začíná jeden hráč (špička pyramidy), který volně improvizuje na zadané \odkaz{askfor}{askfor}. \odkaz{MC}{mc} ho zastaví ve \odkaz{štronzu}{štronzo} písknutím a přichází druhý hráč. Spolu rozehrají novou scénu v jiném prostředí, je dobré, když vycházejí z pozice, ve které jsou. Tímto způsobem se pokračuje dokud není na scéně maximum hráčů (obvykle 6). 
\odkaz{MC}{mc} scénu se šesti hráči po chvíli zastaví, \odkaz{ posune}{posun ve štronzu} např. na dvě fáze   a začíná se bourání pyramidy. Začíná se v poslední rozehrané scénce a hráč, který ji vytvořil z ní musí \odkaz{motivovaně odejít}{motivovaný odchod}. \odkaz{MC}{mc} ohlásí odchod písknutím nebo tlesknutím a pokračuje se ve scénce, která byla rozehraná předtím. Tak to jde dál, až je na scéně pouze první hráč. Pyramidu ukončuje \odkaz{MC}{mc} 
 
\subsection{ Varianty } \begin{itemize}
\item Větší autonomie - \odkaz{hráči}{hráč} si tleskají sami, kategorii ukončují sami
\item Před bouráním pyramidy dojde k časovému nebo pohybovému posunu
\end{itemize}
 
\subsection{ Běžné chyby } \begin{itemize}
\item Nikdo neví, do kterého prostředí se vrací
\item \odkaz{Hráč}{hráč} zapomene, že má odejít
\end{itemize}
 
 
 
 
 
\needspace{5cm} \section{Překlad do znakové řeči} \label{překlad do znakové řeči} \katabox{Téma,odbornost experta 
}{libovolný (3 a více) 
}{3 min} 
 
Diskuze s expertem je \textbf{překládána do znakové řeči}{}. 
 
\subsection{Průběh} Moderátor a expert (často s odborností přidělenou od \odkaz{MC}{mc}) spolu diskutují, obvykle v sedě na židličkách. Třetí hráč  - tlumočník  - převádí celou diskuzi do znakové řeči. 
 
\subsection{Varianty} Na zápase se tlumočník v polovině času vymění za hráče druhého týmu. 
 
\subsection{ Běžné chyby } \begin{itemize}
\item  Expert a moderátor svou gestikulací na sebe strhávají pozornost.
\item  Tlumočník neopakuje gestikulaci u opakovaných výrazů.
\end{itemize}
 
 
 
 
\needspace{5cm} \section{Reklamace} \label{reklamace} \katabox{Co se bude reklamovat 
}{4 
}{neomezený, volný} 
 
 
Uvidíte reklamaci očima servisního technika. 
 
 
\subsection{ Průběh } \odkaz{Hráči}{hráč} se rozdělí na dvojice. Jedna bude reklamovat, druhá bude přijímat reklamaci. Dvojice hráčů, která bude reklamovat odejde ze sálu a nebo je jiným způsobem zbavena možnosti postřehnout, co se bude reklamovat. Následně \odkaz{MC}{mc} vybírá od diváků věc, která se bude reklamovat. Scéna začíná příchodem reklamující dvojice k přijímající dvojici a snaží se vysvětlit, jaké různé problémy má s věcí, kterou přišel reklamovat. Podle reakcí pak reklamující dvojice přichází na to, jaký předmět vlastně reklamuje, čímž \odkaz{kategorie}{kategorie} končí. 
 
\subsection{ Varianty } \begin{itemize}
\item  Hráči nejsou po dvojicích
\item  Do obchodu přijde více lidí se stejným problémem
\end{itemize}
 
\subsection{ Běžné chyby } \begin{itemize}
\item  Reklamující nedávají dost nabídek
\end{itemize}
 
 
 
 
 
\needspace{5cm} \section{Rozhlasová hra} \label{rozhlasová hra} \katabox{Téma 
}{všichni 
}{neomezený} 
 
 
V kategorii \textbf{Rozhlasová hra}{} od hráčů uslyšíte část drama, tak jak byste jej mohli slyšet z rádia. 
 
\subsection{Průběh} Hráči se usadí na jevišti do půlkruhu a v sále se zhasne. Hráči pak podobně jako v kategorii \odkaz{Vysílání FM}{vysílání fm} rozehrají vysílání rádia, ale tentokrát jde jen o jeden pořad a ucelený příběh. Zpravidla začíná Vypravěč rekapitulací toho, co jsme mohli slyšet v minulém díle a poté následuje pokračování, ve kterém vždy jeden hráč ztvárňuje jednu postavu. Tato kategorie se dá přirovnat k \odkaz{En face}{en face} potmě. 
 
\subsection{ Varianty } \begin{itemize}
\item \odkaz{Vysílání FM}{vysílání fm}
\end{itemize}
 
\subsection{ Běžné chyby } \begin{itemize}
\item Nezjistí se dopředu, zda jde v sále úplně zhasnout.
\end{itemize}
 
 
 
 
 
 
\needspace{5cm} \section{Ryba (ve čtyřech)} \label{ryba (ve čtyřech)} \katabox{Prostředí 
}{4 
}{3 minuty 30 vteřin} 
 
 
V kategorii \textbf{Ryba (ve čtyřech)}{} vám 4 hráči zahrají situaci, na jejímž konci budou všichni mrtví. 
 
\subsection{Průběh} Připraví se 4 hráči zády k divákům jako na \odkaz{Prskavku}{prskavka} a je jim zadáno prostředí. Na písknutí \odkaz{MC}{mc} se všichni čtyři otočí, rozehrají situaci a současně musí dávat pozor na časomíru. Funguje zde podobný princip jako v kategorii \odkaz{Smrt v 1 minutě}{smrt v 1 minutě}. V 60 vteřině minutě musí jeden z hráčů motivovaně zemřít, v 120 vteřině umírá druhý, ve 180 vteřině třetí a poslední hráč má dalších 30 vteřin, na jejichž konci musí také zemřít. Mrtvoly v této kategorii zůstávají na scéně do konce improvizace a zbylí hráči s nimi mohou pracovat, ale mrtvoly se už aktivně nezapojují. 
 
\subsection{ Běžné chyby } \begin{itemize}
\item Hráči nesledují čas.
\end{itemize}
 
 
 
 
\needspace{5cm} \section{S rekvizitou} \label{s rekvizitou} \katabox{Reálný předmět , Prostředí 
}{2 
}{2 minuty} 
 
V kategorii \textbf{S rekvizitou}{} vám 2 hráči předvedou, kolika různými způsoby se dá daná rekvizita použít jako předmět. 
 
\subsection{Průběh} Hrají dva hráči, kteří do začátku kategorie rekvizitu nesmí vidět. Pomocný rozhodčí připraví na jeviště \odkaz{rekvizitu}{rekvizita}, kterou může být jakýkoli reálný předmět (deštník, židle, ramínko). Během kategorie se hráči snaží předmět použít co nejvíce způsoby a to tak, že si jej během improvizace předávají a hrají s ním jako by to byl ten předmět, který rekvizita právě představuje (dalekohled, pádlo, mobil, hřeben, atd.).  Předmět by přitom neměl být pojmenován. 
 
\subsection{ Varianty } \begin{itemize}
\item Předmět je vybrán přímo od diváků.
\end{itemize}
 
\subsection{ Běžné chyby } \begin{itemize}
\item Hráči vždy pojmenují, co má rekvizita představovat.
\item Rekvizita se během improvizace promění v malý počet předmětů.
\item Hráči nejsou schopní si rekvizitu technicky předat. Doporučuje se ji například upustit na zem a pokračovat následně v činnosti s imaginárním předmětem.
\end{itemize}
 
 
 
 
\needspace{5cm} \section{Schizofrenie} \label{schizofrenie} \katabox{téma + charakter pro každého hráče}{4}{2 min.} 
 
Ve \textbf{Schizofrenie} jsou dvě postavy ztvárněné každá dvěma \odkaz{hráči}{hráč}, každá postava má dvě tváře/charaktery. 
 
\subsection{ Průběh } Často hrají jen dvě postavy. Každou postavu hrají dva hráči. Před začátkem improvizace vyzve \odkaz{MC}{mc} hráče, aby předstoupili a diváci jim zadají charakterové vlastnosti či silné \odkaz{emoce}{emoce} Často velice odlišné, ne však naprosto opačné: např. závistivý/zamilovaný, stydlivá/hašteřivá, flegmatik/cholerik, odvážná/bojácná. 
Při kategorii hrají 2 hráči a druzí dva hráči vyčkávají. V momentě kdy chtějí svého kolegu vystřídat tlesknou, hráči na jevišti \odkaz{štronznou}{štronzo}, hráč co tleskl přiběhne ke svému kolegovi, převezme jeho pozici a kategorie pokračuje. Časté proměny jsou funkční a je zábavné, jak se i během věty může charakter postavy zcela proměnit. Většinou platí pravidlo "Tleskejte hodně!". 
 
\subsection{ Časté chyby } \begin{itemize}
\item Charaktery se vybírají příliš dlouho.
\item Vyčkávající hráči málo střídají svého kolegu.
\item Hráč nejedná ve svém zadaném charakteru.
\end{itemize}
 
 
 
 
 
\needspace{5cm} \section{Sci-fi} \label{sci-fi} \katabox{téma  
}{2+libovolně 
}{cca 3 minuty} 
V prostředí sci-fi, vědeckofantastických filmů a knih se odehrává kategorie \textbf{Sci-fi}{} 
 
\subsection{Průběh} Probíhá stejně jako \odkaz{volná}{volná}, ale s využitím prostředí, reálií sci-fi.  
\subsection{Běžné chyby} \begin{itemize}
\item Stereotypní začínání ve stavu beztíže
\end{itemize}
 
 
 
 
\needspace{5cm} \section{Smrt v 1 minutě} \label{smrt v 1 minutě} \katabox{Prostředí 
}{2 
}{60 vteřin} 
 
 
V kategorii \textbf{Smrt v 1 minutě}{} jeden ze dvou hráčů zemře. 
 
\subsection{Průběh} Hráči v daném prostředí improvizují scénku, která vyustí v umrtí jedné postavy přesně po uplynutí jedné minuty. Kategorie se na \odkaz{zápasech}{zápas} často opakuje, protože hráči nesplní její podmínky. 
 
Průběh je podobný jako v kategorii \odkaz{Ryba (ve čtyřech)}{ryba (ve čtyřech)} 
 
\subsection{ Varianty } \begin{itemize}
\item Hraje se volně a k smrti se nějak dojde
\item Hráči generují co největší množství nechtěných zranění svému kolegovi či sobě
\end{itemize}
 
\subsection{ Běžné chyby } \begin{itemize}
\item Hráči nesledují čas
\item Nikdo nezemře v šedesáté vteřině
\item Smrt je příliš náhlá a očekávatelná
\item \odkaz{Blokování}{bránění ve hře} druhé postavy znemožněním pohybu (svázané končetiny, atd.)
\end{itemize}
 
 
 
 
\needspace{5cm} \section{Spoon river} \label{spoon river} \katabox{Dvojslovné téma, prostředí 
}{4-6 
}{Neomezený, bývá 5 a více minut, typicky alespoň na 3 promluvy za každého hráče. 
} 
V kategorii \textbf{Spoon river}{} uslyšíte příběh několika postav, které zdánlivě nemají nic společného. 
 
\subsection{Průběh} Hráči se postaví na kraj pódia (stejně jako v \odkaz{En face}{en face}), v náhodném pořadí se střídají v monolozích z pohledu jednotlivých postav. 
Při prvním monologu se zdá, že každá postava je na ostatních zcela nezávislá, s každou další promluvou se však čím dál tím víc zjišťuje, že tomu tak není, osudy postav se porůznu protínají. 
 
\subsection{Časté chyby} \begin{itemize}
\item Postavy se nepropojí.
\item Postavy se propojí příliš jednoznačně.
\end{itemize}
 
\subsection{Název} Odkazuje na knihu Spoon River Anthology, pojednávající historii jednoho městečka formou epitafu. 
 
 
 
 
\needspace{5cm} \section{Sportovní komentátor} \label{sportovní komentátor} \katabox{Vrcholový sport 
}{3 
}{3 minuty} 
 
V kategorii \textbf{Sportovní komentátor}{} uvidíte návštěvu vrcholového sportovce v televizním studiu s překladem do znakové řeči. 
 
\subsection{ Průběh } Hráči se rozdělí na moderátora, sportovce a překladatele. Moderátor uvítá a zběžně představí hosta, kterým je vrcholový sportovec sportu, který dali diváci. Celý jejich rozhovor je překládán do \odkaz{znakové řeči}{kvaziznaková řeč} překladatelem. Moderátor a host by měli být při rozhovoru co nejnehybnější a pouze mluvit, aby nestrhávali pozornost na sebe a diváci byli zaměřeni na překladatele. 
 
\subsection{ Varianty } \begin{itemize}
\item  \odkaz{Překlad do znakové řeči}{překlad do znakové řeči}
\item  Střídání překladatelů v polovině (nejčastěji na zápase)
\item  \odkaz{Televizní komentátor}{televizní komentátor}
\end{itemize}
 
\subsection{ Běžné chyby } \begin{itemize}
\item  Moderátor nebo host dělají výrazná gesta strhávají pozornost na sebe
\item  Moderátor nebo host mluví příliš rychle a překladatel nestíhá (pokud to není záměr)
\item  Překladatel se zraní během překladu
\end{itemize}
 
 
 
 
 
 
\needspace{5cm} \section{Stíhačka} \label{stíhačka} \katabox{Téma 
}{2, plus naskakující 
}{1:20, 40s na tým} 
 
V kategorii \textbf{Stíhačka}{} pro Vás hráči jednoho týmu vytvoří příběh a hráči druhého týmu určí, jak dopadne. 
 
 
\subsection{Průběh} \odkaz{MC}{mc} musí \odkaz{náhodně rozhodnout}{náhodné rozhodování}, který tým bude začínat. \odkaz{Hráči}{hráč} z prvního týmu začínají, rozehrají příběh na dané téma od diváků. Po 45 vteřinách se pískne a hráči druhého týmu \odkaz{přeberou pozici}{přebírání pozice} i \odkaz{postavu}{postava} a dohrávají příběh. Obecně funguje velice dobře, když se první dvojice drží vytvoření prostředí, definice postav a nastavení nějakého problému. Druhá dvojice pak už pouze rozvíjí. 
 
Na \odkaz{zápasech}{zápas} se obyčejně hraje \odkaz{dvojitá stíhačka}{dvojitá stíhačka}. 
 
\subsection{ Varianty } \begin{itemize}
\item \odkaz{Dvojitá stíhačka}{dvojitá stíhačka}
\item \odkaz{Trojitá stíhačka}{trojitá stíhačka}
\end{itemize}
 
\subsection{ Běžné chyby } \begin{itemize}
\item První dvojice nezvládne \odkaz{improvizační trojzubec}{improvizační trojzubec}. Druhá dvojice musí doplnit
\item Druhá dvojice nedoplní
\item Nepřejme pozici nebo postavu
\item Hráči špatně rozdělí úlohy ve výměnách
\end{itemize}
 
 
 
 
\needspace{5cm} \section{Tam a zpět} \label{tam a zpět} \katabox{Prostředí, slovo pro každého hráče 
}{libovolný (4 a více) 
}{neomezený,volný , 3 -5 minut (podle počtu hráčů)} 
 
 
V kategorii \textbf{Tam a zpět}{} hráči sehrají příběh, na scénu musí přijít či odejít pouze ve chvíli, kdy padne jejich klíčové slovo. 
 
 
\subsection{Průběh} \odkaz{MC}{mc} každému hráči určí jedno slovo, vetšinou z daného prostředí, a také který hráč má začínat na scéně. 
Hráč začíná v daném prostředí, a ve chvíli, kdy uvede klíčové slovo jiného hráče, ten musí bez meškání naběhnout na scénu a zapojit se do děje až do chvíle, kdy znovu padne jeho klíčové slovo, pak \odkaz{motivovaně odchází}{motivovaný odchod}. Padne li jeho klíčové slovo vícekrát za sebou, musí vícekrát přijít/ odejít. 
 
Na závěr poslední hráč odvolá sám sebe, či, lépe, předposlední při svém odchodu logicky odvolá posledního. 
 
\subsection{ Běžné chyby } \begin{itemize}
\item Hráči si nepamatují klíčová slova.
\item Hráči nereagují na svá klíčová slova.
\end{itemize}
 
 
 
 
 
\needspace{5cm} \section{Televizní komentátor} \label{televizní komentátor} \katabox{Téma, pro každé studio 
}{2x3 
}{3 minuty} 
 
V kategorii \textbf{Televizní komentátor}{} uvidíte náhled dvou televizních studií na různá témata. 
 
\subsection{ Průběh } Průběh je podobný jako v kategorii \odkaz{Sportovní komentátor}{sportovní komentátor}. Hráči se rozdělí na moderátora, hosta a překladatele do \odkaz{znakové řeči}{kvaziznaková řeč}, každý tým má vlastní studio. \odkaz{MC}{mc} přidělí témata studiím a vybere \odkaz{náhodným rozhodováním}{náhodné rozhodování}, kdo bude začínat. V průběhu kategorie \odkaz{MC}{mc} přepíná televizní studia. 
Studio, na které není fokus, zůstává ve \odkaz{štronzu}{štronzo}, ale čas v něm zatím běží, děj se přesouvá dál. 
 
\subsection{ Varianty } \begin{itemize}
\item  \odkaz{Sportovní komentátor}{sportovní komentátor}
\end{itemize}
 
 
 
 
\needspace{5cm} \section{Toaster} \label{toaster} \katabox{Prostředí 
}{4-7}{cca 6 minut  
} 
 
Hráči vyskakují jako topinky z topinkovače a pokračují v epickém příběhu ze zadaného prostředí. 
 
\subsection{ Průběh } Hráči se rozmístí po scéně, čekají na bobku, ve chvíli, kdy \odkaz{MC}{mc} pískne, náhodný počet hráčů se rychle postaví, ti všichni jsou v tu chvíli ve scéně a je nutné s nimi hrát.  Na každé další písknutí jdou stojící hráči k zemi, případně se můžou znovu ihned postavit. Po celou dobu si drží svoji postavu   a rozvíjí celkový příběh. 
 
 
\subsection{ Běžné chyby } \begin{itemize}
\item Uklízení, hledání něčeho.
\item Vyskakování vždycky na každou druhou dobu.
\end{itemize}
 
 
 
\needspace{5cm} \section{Vnitřní hlasy} \label{vnitřní hlasy} \katabox{Téma 
}{4 
}{2-3 minuty} 
 
 
V kategorii \textbf{Vnitřní hlasy}{} vám 2 hráči zahrají situaci na Vaše téma a další dva hráči vám řeknou, co si ti dva při tom myslí. 
 
 
\subsection{Průběh} Hrají 4 hráči, 2 na scéně a 2 na straně, tak že nejsou vidět, kteří dostanou mikrofony.  Dva hráči tvoří v podstatě jednu postavu. Hráči na scéně rozehrají volně situaci a druzí dva jsou jejich Vnitřní hlasy - doplňují akci a promluvu svého kolegy nějakou myšlenkou, kterou si postava myslí a bývá zpravidla protichůdného rázu. Např. situace První rande. Hráč na scéně: Jé, tobě to dneska ale sluší. Vnitřní hlas: Ale v těch šatech jsi docela tlustá. 
 
\subsection{ Varianty } \begin{itemize}
\item Muže se objevit třetí postava na scéně v nějaké malé roli, která nemá vnitřní hlas.
\end{itemize}
 
\subsection{ Běžné chyby } \begin{itemize}
\item Hráči na scéně neustále mluví a nedávají prostor svým Vnitřním hlasům.
\item Hráči dostanou vypnutý mikrofon.
\end{itemize}
 
 
 
 
\needspace{5cm} \section{Volná} \label{volná} \katabox{téma 
}{2+libovolně 
}{cca 10 minut} 
V kategorii \textbf{volná}{}  hráči odehrají krátký příběh. 
 
\subsection{Průběh} V kategorii nejsou žádná omezení, začínají dva hráči a odehraje se nějaký příběh. 
 
\subsection{Varianty} \begin{itemize}
\item \odkaz{Sci-fi}{sci-fi}, \odkaz{Western}{western}, \odkaz{Červená knihovna}{červená knihovna} - je přidáno omezení využívání daného žánru.
\end{itemize}
 
 
 
\needspace{5cm} \section{Vyprávěná} \label{vyprávěná} \katabox{téma 
}{neomezen 
}{2 minuty} 
 
V kategorii \textbf{Vyprávěná}{} se jeden z hráčů stává vypravěčem a ostatní hráči hrají volně na podněty od vypravěče. 
 
\subsection{ Průběh } První hráč, který do kategorie naběhne, je vyprávěč. Vypravěč nás uvede do příběhu, prostředí a představí nám postavy, které vstupují na scénu. Ostatní hráči reagují na to, co řekl vypravěč, ale současně mají možnost sami děj vytvářet. Vypravěč má také možnost posouvat děj v čase a diváci mohou vidět, co bylo příčinou současného jednání té které postavy, když se jí před dvaceti lety něco stalo a poté se vrátíme zpět do současnosti. Děj je také možné posouvat v čase dopředu, zvlášť v případech, kdy se nic neděje, případně když hráč navrhne nějakou zdlouhavou aktivitu. 
 
\subsection{ Běžné chyby } \begin{itemize}
\item Vypravěč stále mluví a nedává tak prostor ostatním hráčům.
\item Vypravěč popisuje to, co hráči dělají = neposouvá děj.
\item Hráč hraje jen to, co řekl vypravěč = neposouvá děj.
\end{itemize}
 
 
 
 
 
\needspace{5cm} \section{Vysílání FM} \label{vysílání fm} \katabox{Téma 
}{5 a více, obvykle se zapojí všichni 
}{volný, obvykle 5 - 10 minut} 
 
 
Diváci  si připomenou, jak mohl vypadat poslech rádia. 
 
 
\subsection{ Průběh } \odkaz{Hráči}{hráč} se usadí do půlkruhu a v sále se \textbf{zhasne}{}. Začíná se zvuky éteru, než se na sebe všichni naladí. Ve studiu sedí moderátor stanice a někdy i host. Každý hráč se zapojuje volně, například vytvořením znělky rádia, zavoláním do studia nebo jen děláním zvuků. 
 
Po ukončení kategorie se hráči akusticky představí nějakou vlastní replikou z vysílání, aby diváci mohli \odkaz{hlasovat}{hlasování}. 
 
\subsection{ Varianty } \begin{itemize}
\item Do studia mohou volat i diváci
\end{itemize}
 
\subsection{ Běžné chyby } \begin{itemize}
\item Po vyhlášení kategorie se zjistí, že v sále nejde udělat tma
\item Nic se neděje a moderátor plká
\item Hudebník se nezapojí
\end{itemize}
 
 
 
 
 
 
\needspace{5cm} \section{Věty z kapsy} \label{věty z kapsy} \katabox{Téma, prostředí 
}{2 
}{neomezený,volný} 
 
 
V kategorii \textbf{Věty z kapsy}{} vám hráči zahrají na Vaše téma a přitom použijí věty, které jste jim napsali na papírky. 
 
 
\subsection{Průběh} Před zápasem mají diváci možnost napsat krom dvouslovného téma také libovolnou větu na jiný papírek. Témata a věty se vybírají zvlášť. Před kategorií si dva hráči přibližně rovnoměrně rozdělí lístečky s větami a ty si připraví do kapes. Rozehrají improvizaci na dané téma a během ní čas od času vytáhnou lísteček s větou, kterou musí okamžitě a co nejobratněji začlenit do improvizace.  
 
 
\subsection{ Varianty } \begin{itemize}
\item Hráči tahají věty z kapsy podle svého uvážení.
\item Hráči tahají věty z kapsy na povel \odkaz{MC}{mc} či na povel od diváků "Věta!"
\end{itemize}
 
 
\subsection{ Běžné chyby } \begin{itemize}
\item Věty jsou vybrány do košíčku na témata.
\item Divák napíše větu a téma na stejný lístek.
\item Hráči vytažení každého lístečku uvádí bezpečnostní větou - např. Maminka mi vždy říkávala ...
\item Hráč otálí před použitím věty po jejím vytažení.
\end{itemize}
 
 
 
 
 
\needspace{5cm} \section{Western} \label{western} \katabox{téma  
}{2+libovolně 
}{cca 3 minuty} 
 
V prostředí divokého západu, Ameriky konce předminulého století, se odehrává kategorie \textbf{Western}{} 
 
\subsection{Průběh} Probíhá stejně jako \odkaz{volná}{volná}, ale s využitím prostředí, reálií divokého západu.  
 
\subsection{Běžné chyby} \begin{itemize}
\item Stereotypní začínání v saloonu
\end{itemize}
 
 
 
 
\needspace{5cm} \section{Změna} \label{změna} \katabox{Téma 
}{2 
}{2 min} 
 
 
V kategorii \textbf{Změna}{}, občas nazývané \textbf{Jinak}{} hrají hráči volnou improvizaci na téma, ale často se v ní mění \odkaz{repliky}{replika} a pohyby na povel \odkaz{MC}{mc}. 
 
 
\subsection{ Průběh } Kdykoli v průběhu improvizace může od \odkaz{MC}{mc} zaznít povel "změna"{} nebo "jinak". \odkaz{Hráč}{hráč}, který mluvil poslední musí změnit svoji repliku včetně pohybu, kterým ji doprovodil. Pokud není \odkaz{MC}{mc} ani s druhou variantou spokojen, může povel opakovat, ale nemělo by to být na úkor improvizace. 
 
\subsection{ Varianty } \begin{itemize}
\item  \odkaz{MC}{mc} má k dispozici dva povely: \textbf{Jinak}{} = změna poslední \odkaz{repliky}{replika}, \textbf{Jiný pohyb}{} = změna posledního pohybu
\item  Může se zapojit více hráčů
\end{itemize}
 
\subsection{ Běžné chyby } \begin{itemize}
\item Hráči se zablokují po změně a přestanou se hýbat
\item Hráči mění příliš málo.  (Rád bych koupil máslo. Změna. Rád bych koupil margarín. Změna. Rád bych koupil chleba.)
\end{itemize}
 
 
 
 
 
\needspace{5cm} \section{Zpívaná} \label{zpívaná} \katabox{Téma, prostředí 
}{libovolný (3 a více) 
}{neomezený,volný , 3 -5 minut (podle počtu hráčů)} 
 
 
V kategorii \textbf{Zpívaná}{} vám hráči dohromady zazpívají písničku na Vaše téma. 
 
 
\subsection{Průběh} Hráči stojí v těsném půlkruhu a zpívají. Začínají refrénem (2x opakovaný), následně jednotlivci zazpívají sloku, vždy následovanou refrénem, na konci se refrén zpívá vícekrát a vygradovává. Když jednotlivec zpívá sloku (sloky), vystoupí z půlkruhu a zpívá na mikrofon.  
 
S ohledem na plynulost je nutné, aby již na závěr refrénu byl u mikrofonu připraven hráč, který bude zpívat jako další. 
 
 
\subsection{ Varianty } \begin{itemize}
\item  Každý přistoupí k tématu z trošku jiného úhlu pohledu.
\item  Příběh navazuje.
\item  "Stará Zpívaná"{} aká muzikál. Hráči jsou implicitně v půlkruhu jako chór a před nimi se normálně odehrává příběh, však zpíván. Hráči mohou libovolně vstupovat a vystupovat jak se jim zamane. Refrén může být i nemusí. Podobné muzikálu při žánrech.
\end{itemize}
 
 
\subsection{ Běžné chyby } \begin{itemize}
\item  Nikomu se nechce dopředu, vzniká mrtvá pauza vyplněná čekáním.
\end{itemize}
 
 
 
 
 
\needspace{5cm} \section{Červená knihovna} \label{červená knihovna} \katabox{Téma 
}{libovolný 
}{2-3 minuty} 
 
 
V kategorii \textbf{Červená knihovna}{} vám hráči zahrají příběh plný klišé z romantické literatury. 
 
 
\subsection{Průběh} Začínají dva hráči, kteří rozehrávají příběh plný klišé, která známe z televizních novel a volně se k nim přidávají další postavy. Co by se mělo objevit: tajná láska, utajené rodinné vztahy, přetrvávající rodové rozepře, souboj, exotická jména, atd. 
 
\subsection{ Varianty } \begin{itemize}
\item V prvním poločasu je rozehrán začátek příběhu a v druhém poločasu jeho dokončení.
\end{itemize}
 
 
\subsection{ Běžné chyby } \begin{itemize}
\item Zapomenutí jména postavy
\end{itemize}
 
 
 
 
 
\needspace{5cm} \section{Čtverec} \label{čtverec} \katabox{4 * prostředí, a 1 téma navíc 
}{4 
}{neomezen} 
 
Ve \textbf{Čtverci}{} dva hráči na každé straně čtverce dostanou jedno prostředí, na které pak hrají, když je jejich strana natočená k divákům. 
 
\subsection{Průběh} Připraví se čtyři hráči a na scéně zaujmou postavení tvaru čtverce - dva dopředu, dva dozadu. \odkaz{MC}{mc} určí každé dvojici prostředí, které vybere od diváků a hráči si je ještě pro jistotu unisono zopakují. Každý hráč tak hraje ve dvou prostředích. Během improvizace \odkaz{MC}{mc} může čtvercem libovolně otáčet po směru či proti směru hodinových ručiček o jedno až tři prostředí. Běžně se při prvním hraní v daném prostředí (strany čtverce) odehraje úvod, v druhém zápletka a v třetím závěr (takže pak má kategorie 4*3 =12 scén), pokud MC neukončí dřív. 
Při "otočení"{} se přesně přejímají pozice hráčů. 
 
\subsection{ Varianty} \begin{itemize}
\item Technicky podobná je kategorie \odkaz{Poslední věta}{poslední věta} a dá se (zejména na \odkaz{improshow}{improshow}) s touto kategorií celkem bezproblémově kombinovat.
\item Hráči přejímají pozice.
\item Hráči mají shodnou postavu v obou prostředích, a tvoří tak větší příběh.
\end{itemize}
 
\subsection{Běžné chyby} \begin{itemize}
\item Shodné charaktery
\item Hráči se ztratí v otáčení čtvercem
\item Hráč zapomene v jakém prostředí se právě nachází
\end{itemize}
 
 
 
 
 
\needspace{5cm} \section{Šumlovaná} \label{šumlovaná} \katabox{Téma}{2 a více 
}{2 min 
} 
 
V kategorii \textbf{Šumlovaná}{} hráči improvizují v jazyce \odkaz{gibberish}{gibberish}. 
 
\subsection{ Průběh } \odkaz{MC}{mc} se zeptá diváků na nějaký jazyk, může být vymyšlený nebo reálný, ale jiný než kterým hráči mluví. Hráči poté rozehrají volnou improvizaci v \odkaz{gibberish}{gibberish} tak, aby připomínala zvolený jazyk. Význam se do vymyšleného jazyka dodává pohybem. 
 
\subsection{ Běžné chyby } \begin{itemize}
\item  Hráči hrají více hlasem než pohybem a z improvizace se vytrácí smysl
\end{itemize}
 
\subsection{ Podobné kategorie } \begin{itemize}
\item  \odkaz{Šumlovaná se simultánním překladem}{šumlovaná se simultánním překladem}
\item  \odkaz{Duál}{duál}
\end{itemize}
 
 
 
 
\needspace{5cm} \section{Šumlovaná se simultánním překladem} \label{šumlovaná se simultánním překladem} \katabox{Téma}{2 a více 
}{2 min 
} 
 
V této kategorii hráči improvizují v jazyce \odkaz{gibberish}{gibberish} a mají k dispozici překladatele. 
 
\subsection{ Průběh } Na začátku je určen překladatel. Průběh je podobný jako v kategorii \odkaz{Šumlovaná}{šumlovaná}. Hráči rozehrají volnou improvizaci v \odkaz{gibberish}{gibberish} a při hraní dávají prostor překladateli (podobně jako vypravěči v kategorii \odkaz{Vyprávěná}{vyprávěná}), který neutrálním hlasem překládá jejich řeč.  
 
\subsubsection{ Varianty } V zápasové formě se v polovině času překladatelé vystřídají. 
 
\subsection{ Podobné kategorie } \begin{itemize}
\item  \odkaz{Šumlovaná}{šumlovaná}
\item  \odkaz{Duál}{duál}
\end{itemize}
 
 
 
 
\needspace{5cm} \section{Ženy vs muži (kategorie)} \label{ženy vs muži (kategorie)} \todo{Video} 
 
\katabox{dvouslovné téma/situace 
}{skupina mužů, skupina žen 
}{2*2minuty} 
 
 
V kategorii \textbf{Ženy vs Muži}{} uvidíte dva podobné příběhy, jednou z mužského a podruhé z ženského pohledu. 
 
 
\subsection{Průběh} Hraje se jako \odkaz{Volná}{volná}, nejprve hrají hráči jen jednoho pohlaví, a následně na stejné zadání hráči druhého pohlaví... 
 
\subsection{ Varianty } \begin{itemize}
\item  Muži hrají ženy a ženy hrají muže
\item  Pohlaví, které hraje jako druhé, jde na první část za dveře.
\item  Druhá skupina hraje takřka doslova stejný příběh jako první skupina, jen výrazně pozmění některé gendrové detaily.
\end{itemize}
 
 
 
 
\needspace{5cm} \section{Živé rekvizity} \label{živé rekvizity} \katabox{Prostředí, dva dobrovolníci 
}{2 
}{2 minuty} 
 
V kategorii \textbf{Živé rekvizity}{} vám 2 hráči předvedou, kolika různými způsoby se dá využít nevinný divák jako rekvizita. 
 
\subsection{ Průběh } \odkaz{MC}{mc} požádá o dva dobrovolníky z publika. Mezitím se připraví dva hráči, kteří budou hrát. \odkaz{MC}{mc} přidělí každému hráči jednoho dobrovolníka, který pak musí být svému hráči plně k dispozici a nechat  sebou jakkoli manipulovat neboť se stává všemi rekvizitami, které hráč v improvizaci použije. Hráči pak rozehrávají improvizaci na zadané téma a přitom hojně využívají rekvizit, které si modelují z dobrovolníků. Divák se tak může stát dalekohledem, batehem, dveřmi, stolem, šálkem, činkou či jinými předměty. 
 
\subsection{Viz také } \begin{itemize}
\item \odkaz{S rekvizitou}{s rekvizitou}
\item \odkaz{Zvuky}{zvuky}
\end{itemize}
 
 
\subsection{ Běžné chyby } \begin{itemize}
\item Dlouhé hledání dobrovolníků
\end{itemize}
 
 
 
 
 
 
\needspace{5cm} \section{Žánrová vyprávěná} \label{žánrová vyprávěná} \katabox{Téma, 4 žánry 
}{4 
}{3 minuty} 
 
V kategorii \textbf{Žánrová vyprávěná}{}, známé i jako \textbf{Vyprávěná napřeskáčku}{}, či \textbf{Společné vyprávění}{}  vám 4 hráči vám poví společně jeden příběh v různých žánrech. 
 
\subsection{ Průběh } Připraví se 4 hráči, stoupnou si do středu jeviště vedle sebe s pohledem upřeným do diváků. \odkaz{MC}{mc} vybere pro každého z \odkaz{hráčů}{hráč} žánr, ve kterém bude vyprávět. Téma je pro všechny stejné. \odkaz{MC}{mc} podle vlastního uvážení přepíná (ukazuje), vypravěče a tím i žánr příběhu. 
 
\subsection{ Varianty } \begin{itemize}
\item  Neurčují se žánry, ale emoce. Hráči, kteří zrovna nevypráví dělají křoví v emoci, ve které je příběh.
\item  Nepřepíná MC ale divák, který je usazen před hráče zády k ostatním divákům.
\item  Hráčům nejsou dány žánry ani emoce, ale jsou jim zadávány instrukce které musí plnit - nepoužívat spojku a, mluvit ve slovech neobsahující určité písmeno. Pokud chybují vypadávají až do posledního hráče.
\end{itemize}
 
\subsection{ Běžné chyby } \begin{itemize}
\item  Nepozornost a ztracení se v příběhu
\item  Prodleva při přepínání/dlouhé pasáže v jednom žánru
\end{itemize}
 
 
 
 
 
 
\end{document}