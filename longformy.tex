\needspace{5cm} \section{Armando} \label{armando} Jedná se o kategorii, která je řazena mezi \odkaz{longformy}{longforma}, zpravidla se hraje dvacet až čtyřicet minut. 
 
Jeden z hráčů se stává ve hře Armandem. Hráč je na jevišti sám za sebe, nehraje žádnou roli. Přichází do prostřed scény a žádá od diváků nějaké téma, větu, otázku, cokoliv o co požádá. 
Na základě námětu od diváků vypráví monolog ze svého života. 
 
Pro ostatní hráče je tento monolog inspirací pro rozehrání 3 různých scének - příběhů. Scénku zpravidla rozehrávají 2 hráči, ale může se zapojit více lidí. 
 
Hráči mohou zastávat v různých scénkách různé role, mohou zahrát jednu postavu ve více scénkách, důležité je udržet logiku rozhraných příběhů. 
 
Po 3. přehraném příběhu zajištuje edit znovu Armando a inspirován tím, co viděl, vypráví 2. monolog. Opět následují 3 scénky. 
 
Edity mezi scénky zpravidla zajištuje jeden hráč, který i rozehrává scénku následující. Edit vznikne tím, že hráč viditelně přeběhne jeviště bez očního kontaktu s ostatním hráči. 
 
 
\needspace{5cm} \section{Bar (forma)} \label{bar (forma)} \textbf{Bar}{}  (též \textbf{Věci zapomenuté v baru}{}) je druh \odkaz{longformy}{longforma}. Hráči na začátku nejprve představují předměty zapomenuté v baru, které si berou jako inspiraci pro vytvoření postavy. Poté už jako postavy hrají scénky tvořící jeden příběh. Mezi scénkami si hráči přepínají sami pomocí \odkaz{střihů}{střih}. \textbf{Bar}{} probíhá v třech fázích, kterými celá forma plynule prochází. 
 
\subsection{ Potřeby } \begin{itemize}
\item  \odkaz{MC}{mc}
\item  Hudba nebo hudebník
\item  5-7 hráčů
\end{itemize}
 
\subsection{ Průběh } \odkaz{Hráči}{hráč} začínají \odkaz{zaplňováním prostoru}{zaplňování prostoru} a nalaďují se na sebe, dokud nepřejdou ve \odkaz{štronzo}{štronzo}. To jim může určit \odkaz{MC}{mc} nebo se hráči sami zastaví.  
Funkční je sekání příbhu na části a střídání scén s použitím hudby. 
 
\subsubsection{ Fáze předmětů } \odkaz{Hráči}{hráč} jsou stále ve \odkaz{štronzu}{štronzo} a pojmenují se, co jsou za předmět (zapomenutý v baru) podle polohy, ve které skončili. Podobně jako v \odkaz{Jsem a beru}{jsem, beru}. Po představení všech předmětů opět zaplňují prostor. 
 
\subsubsection{ Fáze monologů } Při zaplňování prostoru postupně vyjdou hráči do popředí a mají monolog na dané téma ve třech kolech. Během chození po prostoru si hráč v hlavě utváří postavu. Dohromady každá postava řekne tři monology. Každý hráč musí svojí postavu nějak popsat, ale obecně platí, že čím méně toho řekne, tím lépe. Příliš mnoho konkrétních informací, může zablokovat nápady spoluhráčů. V ideálním stavu si hráč vybere jednu věc, která je pro tu postavu nejdůležitější a o té něco poví. 
 
\subsubsection{ Vztah k předmětu } V prvním monologu postava mluví o vztahu k předmětu, který "nechala v baru".  
 
\subsubsection{ Něco o sobě } V druhém monologu mluví postava krátce o sobě. Velice dobrý nápad je říct jméno své postavy. 
 
\subsubsection{ Vztah k ostatním } Ve třetím monologu postava mluví o vztahu k ostatním postavám. Pokud předtím někdo zmínil vztah k nějaké postavě, je dobré se toho vztahu okamžitě ujmout, aby se na něj nezapomnělo, ale není potřeba aby všechny postavy měly už od začátku jasný vztah k jiné. Opět to může být svazující pro spoluhráče a může být i oříšek se pak ve vztazích vyznat. 
 
\subsubsection{ Fáze příběhu } Po skončení monologů se \odkaz{hráči}{hráč} rozejdou na strany scény. Příběh se pak sestává ze série scének, které vždy rozehraje několik hráčů, kteří vyjdou na scénu. Podoba scénky není nijak omezena, ale každý hráč si po celou dobu drží svojí postavu a také se pracuje s tím, co už zaznělo. Scénku může uzavřít \odkaz{MC}{mc}, hráč, který zrovna nehraje a čeká na straně, ale mohou ji uzavřít i hráči, kteří právě hrají. Příběh by měl být logicky ukončen, což hlídá \odkaz{MC}{mc}. 
 
 
 
\needspace{5cm} \section{Motel} \label{motel} V malém motelu uprostřed Ameriky je pokoj, který má zvláštní genius loci. 
\subsubsection{Scéna} Na scéně stojí např. židle, noční stolek, věšák a postel, ideálně stylově odpovídající ošuntělému hotýlku kdesi v Americe. 
\subsubsection{AskFor} Od diváků se vyžádá jedno slovo.  
Všech 6 hráčů začnou "uklízet"{} scénu, urovnávat, oprašovat apod., s opakováním tohoto klíčového slova, nakonec na scéně zůstávají jen dva hráči, kteří začínají prvním výstupem prvního dějství. 
Každý hráč si drží postavu a dvojici po celé představení. 
\subsubsection{Dějství} Každé dějství se sestává ze tří výstupů, výstup končí hráči na stranách novým "uklízením"{} scény - jejím návratem do původního stavu. 
\subsubsection{Mezihra} Po skončení prvního a druhého dějství postupně předstoupí z každé dvojice jedna postava a sdělí divákům v \odkaz{monologu}{monolog} nějaké tajemství, které ten druhý nezná. 
\subsubsection{Konec} Po skončení 3. dějství a úklidu scény hráči scénu neopouští, předstupují dopředu, na klaněčku. 
 
 
---- 
''Tento formát vycházející z \odkaz{Harolda}{harold} vyvinul \textbf{ Jstar Atlanta }'' 
 
\needspace{5cm} \section{Věřte, nevěřte} \label{věřte, nevěřte}  
 
\subsection{ Původ }  
Forma je inspirována americkým pořadem Věřte, nevěřte, který v každé epizodě představoval několik příběhů, které se zdály být velice nepravděpodobné, někdy obsahovaly reference na paranormální jevy. Na konci pořadu byl divák postaven před kvíz, kde se příběhy rekapitulovaly a mohl si sebereflektivně zopakovat zda-li byl příběh pravdivý nebo ne. 
 
\subsection{ Typický žánr }  
Mysteriózní/Komedie 
 
\subsection{ Lokace }  
Celá scéna je rozdělena ať už fyzicky nebo světelným přechodem na Lobby a Jeviště. Příběhů se obvykle hraje více. 
 
\subsection{ Lobby }  
V lobby se nachází moderátor pořadu a zároveň vypravěč následujícího příběhu. Lobby je možno vylepšit rekvizitami od diváků, naaranžovat pro téma daného večera a nebo jinak upravit, aby bylo pořadu dodána atmosféra. 
 
\subsection{ Jeviště}  
Zde herci hrají příběh. Jeviště by mělo být prázdné. Pokud je potřeba v příběhu nějaká rekvizita, dodá ji vypravěč před spuštěním příběhu, jinak se pracuje s imaginárními rekvizitami. 
 
\subsection{ Průběh }  
Pořad se nijak neuvádí, není potřeba konferenciér. Na se jevišti se objeví obraz postavený z herců, kteří budou hrát. Po krátké pauze, do které je vhodná znělka, se v Lobby objeví moderátor a začne představovat první záhadu podobně jako vypravěč otevírá příběh v kategorii \odkaz{vyprávěná}{vyprávěná}. Jako hlavní inspirace pro moderátora je určený obraz z herců a rekvizity které najde v Lobby. Je vhodné použít obraz jako "snímek z příběhu"{} který uvidíte - všechny postavy se tam tak objeví. Není to ale nutnost. Po krátkém úvodu se příběh spustí - focus se předá z Lobby na Jeviště.  
 
Průběh příběhu je volný - nejsou daná žádná pravidla. Je dobré používat scénické přechody, ale vše záleží na nastavení skupiny. 
 
Po skončení příběhu se vrací focus a případně všechny rekvizity do Lobby. Moderátor příběh dokončuje pár větami tak, aby se každý divák zamyslel nad tím co je a není možné v našem světě.  
 
Po skončení příběhu se dělá pauza, kde se mohou vystřídat moderátoři, opět je vhodná znělka a průběh jede od začátku. 
 
\subsection{ Zakončení }  
Všechny příběhy jsou improvizované, proto by kvíz o pravdivosti příběhů dopadl asi jednoznačně. Je dobré příběhy zrekapitulovat, pokud se nevymyslí nic lepšího, tak nechat diváka ať si o tom povídá v hospodě ... 
 
\subsection{ Představení záhady }  
Funkční variantou představení záhady pro tuto formu je model popsaný níže. Nenechte se jím však svazovat. 
 
\begin{enumerate}
\item  Zvol si obyčejný fyzický předmět který se dá najít doma
\item  Pojmenuj ho
\item  Řekni nahlas jednu věc, kterou ten předmět dělá
\item  Vymysli si co v tomhle příběhu ten předmět dělá jinak
\item  Představ hlavní postavu
\item  Uveď prostředí ve kterém se příběh odehrává
\item  Předej rekvizity
\end{enumerate}
 
\subsection{ Poznámky } \begin{itemize}
\item  Na moderátora jsou kladeny vysoké nároky na schopnost prezentovat před divákem, zachování postavy, uvěřitelnost a udržení focusu.
\item  Vnímavý osvětlovač představení hodně pomůže
\item  Příběhy by v rámci této formy měli končit těsně před uzavřením
\end{itemize}
 
 
