\documentclass[a4paper,10pt,openany]{book}
\usepackage[utf8]{inputenc}
\usepackage[czech]{babel}
\usepackage{subfiles}
\usepackage{graphicx}
\usepackage{wrapfig}
\usepackage{longtable}
\usepackage{needspace}
\usepackage{titlesec}
\title{ImproWiki}
\author{wiki.improliga.cz pod licencí Creative Commons Uveďte autora}
\makeatletter
\renewcommand\thesection{}
\renewcommand\thesubsection{}
\makeatother
\usepackage[margin=1.5cm]{geometry}
\usepackage{tgpagella}
\usepackage{multicol}

\usepackage[T1]{fontenc}
\usepackage{fancyhdr}
\pagestyle{fancy}
\fancyhf{}
\fancyhead[LO,LE]{\today}
\fancyhead[CE,CO]{wiki.improliga.cz}
\fancyhead[RE,RO]{\thepage}



\newcommand{\todo}[1]{

TODO: \textbf{#1} 

}

\setlength{\intextsep}{0pt}



\newcommand{\faulbox}[3]{

\begin{wrapfigure}{L}{0}
#1
\end{wrapfigure}

\textbf{Gesto:}\p{#2} 


\textbf{Trestné body:}\p{#3} 


}


\newcommand{\obrazek}[2]{
\includegraphics[width=4cm]{#1}

\textbf{#2}}


\newcommand{\obrazekmaly}[1]{
\includegraphics[width=1.8cm]{#1}}


\newcommand{\odkaz}[2]{ \textbf{#1}\textsuperscript{s.~\pageref{#2}}}


\newcommand{\katabox}[3]
{tady má být katabox


\begin{wrapfigure}{R}{5.3cm}

\textbf{Téma} {#1} \\ 
\textbf{Hráči} {#2} \\
\textbf{Čas} {#3} \\
\end{wrapfigure}

}

\newcommand{\btbinfo}[6]{

\ifx Z#6 \textbf{#1} \else #1 \small{  (i)} \fi   & \pageref{#2} &  \small{#3} & \small{#4} & \small{#5}\hline
}

\newcommand{\faulinfo}[5]{
#4 & #1 & #3 & #5 \\

}


%\footnote{\textbf{#2}, strana \pageref{#2} }

\clubpenalty 10000
\widowpenalty 10000


\titlespacing*{\section}{-5mm}{4mm}{3mm}

\begin{document}


\hyphenation{pros-tře-dí-mi dvo-ji-ce-mi im-pro-li-ga}
\begin{titlepage}
\begin{center}

% Upper part of the page. The '~' is needed because \\
% only works if a paragraph has started.
\includegraphics[width=0.15\textwidth]{./logo}~\\[5cm]

\textsc{\Large wiki.improliga.cz pod licencí Creative Commons Uveďte autora }\\[0.5cm]

% Title
%%%%\HRule %[0.4cm]
{ \huge \bfseries ImproWiki \\[0.4cm] }

%%%%\HRule %\\[1.5cm]


\vfill

% Bottom of the page
{\large Vygenerováno \today}

\end{center}




\end{titlepage}
 
\chapter{Úvod}\label{úvod}

\subfile{uvod} 

Oproti verzi dostupné na internetu zde chybí odkazy na videa a další internetové zdroje.

Aktuálnost a užitečnost tištěné verze (ani wiki) není zaručena, v případě nejasností konzultujte se zkušenějším hráčem či s improwiki.

Tučně vysázená \odkaz{hesla}{úvod} říkají, na které straně se o daném termínu dočtete víc.

Některé termíny ještě nejsou popsané, přihlašte se na http://wiki.improliga.cz/ a dopište je.

\textbf{Varování:} Nikdo se ještě nenaučil improvizovat z knížky. Popis kategorí je určen spíše zkušeným hráčům pro připomenutí, než nováčkům na naučení se.  

\LaTeX{} sazbu připravuje  \odkaz{Václav Černý}{uživatel:vatoz}.

Chyby či náměty na vylepšení konverze hlaste na https://github.com/vatoz/improwikibook/issues

Případné faktické chyby a překlepy můžete opravit přímo na improwiki.
\chapter{Zápas}
\subfile{zapas}






\chapter{Postavy}
\subfile{postavy}


\chapter{Zápasové kategorie}
\label{zápasové kategorie}
\label{:kategorie:zápasové kategorie}
\subfile{kategorie_start}
\subfile{kategoriez}

\chapter{Kategorie na improshow}\label{další kategorie}
\label{:kategorie:kategorie na improshow}
Tyto kategorie bývají hrány na improshow.
\subfile{kategoriei}

\chapter{Rozcvičky}\label{rozcvičky}
\label{:kategorie:rozcvičky}

\subfile{rozcvicky_start}
\subfile{rozcvicky}


\chapter{Příprava zápasu}
\subfile{priprava}

\chapter{Manuály}
\subfile{manual}

\chapter{Předzápasový trénink}
\subfile{predzapasovy}

\chapter{Cvičení}
\subfile{cviceni}

\chapter{Fauly}\label{fauly}

\label{:kategorie:fauly}
\subfile{fauly_start} 
\subfile{fauly} 

\chapter{Terminologie}\label{terminologie}
\label{:kategorie:terminologie}
\subfile{terminologie}

\chapter{Příběh}
\subfile{pribeh_start}
\subfile{pribeh}


\chapter{Longformy}\label{longformy}
\subfile{longforma}
\subfile{longformy}


\chapter{Knihy}\label{knihy}
\subfile{books}


\chapter{Autoři}\label{autori}
\subfile{authors}

\chapter{Co se jinam nevešlo}\label{co se jinam nevešlo}
\subfile{zbytek}

\chapter{Bonusy}
\pagebreak

\begin{longtable}{|p{4cm}|p{.4cm}|p{3.2cm}|p{3cm}|p{4.2cm}|}
\hline 
Název&str.&čas&hráči&téma \hline 
\documentclass[main.tex]{subfiles}\begin{document}
\btbinfo{Abeceda}{abeceda}{2 min}{2 
}{Téma 
}{I}
\btbinfo{Barman song}{barman song}{neomezeně}{neomezeně 
}{Problém pro každého hráče 
}{Z}
\btbinfo{Báseň od publika}{báseň od publika}{cca 6 min}{2 
}{Několik(4) slov pro každého hráče 
}{Z}
\btbinfo{Break-up song}{break-up song}{neomezený, 3 - 5 minut (podle počtu hráčů)}{2 
}{Dobrovolník z publika, židle 
}{I}
\btbinfo{Co tím chceš jako říct}{co tím chceš jako říct}{volně, více jak 3 minuty}{2+ 
}{Téma, prostředí 
}{Z}
\btbinfo{Červená knihovna}{červená knihovna}{2-3 minuty}{libovolný 
}{Téma 
}{Z}
\btbinfo{Čtverec}{čtverec}{neomezen}{4 
}{4 * prostředí, a 1 téma navíc 
}{Z}
\btbinfo{Dabing ABC}{dabing abc}{neomezený}{3 
}{Téma 
}{Z}
\btbinfo{Dabing filmu}{dabing filmu}{omezen ukázkou}{2-3 
}{Téma 
}{Z}
\btbinfo{Dabovaná}{dabovaná}{2 minuty}{4 
}{Téma 
}{Z}
\btbinfo{Diapozitivy}{diapozitivy}{3 minuty}{6+ 
}{Téma 
}{Z}
\btbinfo{Divadlo poezie}{divadlo poezie}{4 minuty}{4 a více 
}{Téma 
}{I}
\btbinfo{Dopis}{dopis}{30 vteřin, 3 minuty 
}{2, následně neomezeno 
}{Obvykle dvě postavy(kdo a komu píše) 
}{Z}
\btbinfo{Dopředu, dozadu}{dopředu, dozadu}{neomezený}{2 a naskakující 
}{Téma 
}{I}
\btbinfo{Duál}{duál}{2 min. 
}{2 a více 
}{Téma}{Z}
\btbinfo{Duet}{duet}{volný, do 30 s 
}{dvojice 
}{jednoslovné téma nebo předmět pro každou dvojici 
}{Z}
\btbinfo{Dům a zahrada}{dům a zahrada}{neomezený}{libovolný (ideálně 3 na každé straně) 
}{Téma, prostředí 
}{Z}
\btbinfo{Dva na pět}{dva na pět}{neomezený}{dva 
}{prostředí 
}{Z}
\btbinfo{Dva začátky}{dva začátky}{5 minut}{libovolný 
}{Téma 
}{I}
\btbinfo{Dvě repliky}{dvě repliky}{3 minuty}{libovolný (3 a více) 
}{Dvě věty pro kažého (až na jednoho) hráče 
}{I}
\btbinfo{Dvojitá stíhačka}{dvojitá stíhačka}{4x40s 
}{Dva týmy}{Téma 
}{Z}
\btbinfo{Emocionální autobus}{emocionální autobus}{neomezen}{1+3-6 
}{nic 
}{I}
\btbinfo{En face}{en face}{3 min 
}{4-6 
}{Téma 
}{Z}
\btbinfo{Facebookový status}{facebookový status}{3  minuty}{libovolný  
}{Poslední facebookový status některého diváka 
}{I}
\btbinfo{Filmové žánry}{filmové žánry}{neomezený, 5 - 15 minut}{2 + 2n 
}{Téma, několik existujících filmových a divadelních žánrů 
}{Z}
\btbinfo{Handicapovaná pohádka}{handicapovaná pohádka}{podle vybrané pohádky}{5 
}{Známá pohádka a handicap pro každého hráče 
}{I}
\btbinfo{Jestli víš, co tím myslím}{jestli víš, co tím myslím}{2  minuty}{2+  
}{Prostředí 
}{I}
\btbinfo{Kam zmizel ten starý song}{kam zmizel ten starý song}{cca 3 minuty na každou píseň}{libovolný (3 a více) 
}{Téma, povolání/specializace 
}{Z}
\btbinfo{Kaskadéři}{kaskadéři}{orientačně 3 min}{4 (dva páry) 
}{Téma 
}{I}
\btbinfo{Komiks}{komiks}{cca 1 minuta}{4 
}{Téma 
}{Z}
\btbinfo{Kramářská píseň}{kramářská píseň}{4 minuty}{libovolný (4 a více) 
}{Téma 
}{I}
\btbinfo{Krátká báseň}{krátká báseň}{volný, max minuta}{4 
}{téma či název básně 
}{I}
\btbinfo{Kvartet}{kvartet}{volný 
}{4 
}{jednoslovné téma nebo předmět 
}{Z}
\btbinfo{Loutky}{loutky}{3 minuty}{sudý počet, 4 a více 
}{Téma 
}{Z}
\btbinfo{Love song}{love song}{neomezený, 3 - 5 minut (podle počtu hráčů)}{2 
}{Dobrovolník z publika, židle 
}{Z}
\btbinfo{Metafory}{metafory}{volný, dokud nedojdou nápady}{libovolný, obvykle se zapojí všichni 
}{Dva názvy předmětů 
}{Z}
\btbinfo{Mikrofon}{mikrofon}{neomezený,volný}{libovolný 
}{Téma 
}{Z}
\btbinfo{Monolog}{monolog}{neomezený,volný}{libovolný 
}{Téma 
}{Z}
\btbinfo{Muzikál}{muzikál}{5-15 minut}{2 - 5 
}{Slovní téma 
}{I}
\btbinfo{Nátlak (kategorie)}{nátlak (kategorie)}{3 minuty, či do splnění úkolu}{2 
}{úkol pro jednoho hráče 
}{Z}
\btbinfo{Občanky}{občanky}{volný, dokud nedojdou nápady}{3-4 
}{Občanské průkazy od diváků podle počtu hráčů 
}{I}
\btbinfo{Opilecká píseň}{opilecká píseň}{volný}{4 
}{Téma 
}{Z}
\btbinfo{Oprsklá}{oprsklá}{neomezen 
}{2 
}{téma/prostředí/vztah/situace 
}{I}
\btbinfo{Orákulum}{orákulum}{volný}{všichni 
}{otázka 
}{I}
\btbinfo{Počet slov}{počet slov}{3 minuty}{libovolný (3 a více) 
}{Téma,počet pro každého hráče (ten může být přidělený) 
}{Z}
\btbinfo{Poetická}{poetická}{3 minuty}{2 
}{Téma 
}{Z}
\btbinfo{Polovina času}{polovina času}{Např 60, 30, 15, 7.5, 3.75, 1.8   
}{2 a více 
}{téma}{Z}
\btbinfo{Poslední věta}{poslední věta}{do nalezení pointy, 5-10 minut}{dvojice (2 a více) 
}{Téma a prostředí 
}{Z}
\btbinfo{Prskavka}{prskavka}{30 vteřin 
}{libovolný, každý sám za sebe 
}{jednoslovné téma pro každého hráče 
}{Z}
\btbinfo{Prskavka ala šanson}{prskavka ala šanson}{cca 40 - 90 vteřin na každého hráče 
}{libovolný, každý jednotlivě 
}{Reálné předměty z publika 
}{Z}
\btbinfo{Překlad do znakové řeči}{překlad do znakové řeči}{3 min}{libovolný (3 a více) 
}{Téma,odbornost experta 
}{Z}
\btbinfo{Příběh po slovech}{příběh po slovech}{volný}{libovolný,minimálně 2 
}{Téma 
}{I}
\btbinfo{Pyramida}{pyramida}{volný}{4-6 
}{Prostředí, Téma nebo činnost 
}{Z}
\btbinfo{Reklamace}{reklamace}{neomezený, volný}{4 
}{Co se bude reklamovat 
}{Z}
\btbinfo{Režiséři}{režiséři}{neomezen}{neomezen}{Téma pro každého z režisérů}{I}
\btbinfo{Rozhlasová hra}{rozhlasová hra}{neomezený}{všichni 
}{Téma 
}{Z}
\btbinfo{Ruce}{ruce}{cca 8 minut}{4+ (nebo 4 a MC) 
}{Bláznivý vynález 
}{Z}
\btbinfo{Ryba (ve čtyřech)}{ryba (ve čtyřech)}{3 minuty 30 vteřin}{4 
}{Prostředí 
}{Z}
\btbinfo{Scénář / Souboj s knihou}{scénář / souboj s knihou}{3 minuty}{2 
}{Připravená kniha či scénář 
}{I}
\btbinfo{Sci-fi}{sci-fi}{cca 3 minuty}{2+libovolně 
}{téma  
}{Z}
\btbinfo{Shakespeare pozpátku}{shakespeare pozpátku}{cca 10 minut}{neomezený 
}{Téma 
}{I}
\btbinfo{Schizofrenie}{schizofrenie}{2 min.}{4}{téma + charakter pro každého hráče}{Z}
\btbinfo{Smrt v 1 minutě}{smrt v 1 minutě}{60 vteřin}{2 
}{Prostředí 
}{Z}
\btbinfo{Spoon river}{spoon river}{Neomezený, bývá 5 a více minut, typicky alespoň na 3 promluvy za každého hráče. 
}{4-6 
}{Dvojslovné téma, prostředí 
}{Z}
\btbinfo{Sportovní komentátor}{sportovní komentátor}{3 minuty}{3 
}{Vrcholový sport 
}{Z}
\btbinfo{S rekvizitou}{s rekvizitou}{2 minuty}{2 
}{Reálný předmět , Prostředí 
}{Z}
\btbinfo{Statusy}{statusy}{volný}{4 
}{Téma a lístky s čísly 
}{I}
\btbinfo{Stíhačka}{stíhačka}{1:20, 40s na tým}{2, plus naskakující 
}{Téma 
}{Z}
\btbinfo{Stojí, leží, sedí}{stojí, leží, sedí}{2 min}{3}{Téma}{I}
\btbinfo{Šumlovaná}{šumlovaná}{2 min 
}{2 a více 
}{Téma}{Z}
\btbinfo{Šumlovaná se simultánním překladem}{šumlovaná se simultánním překladem}{2 min 
}{2 a více 
}{Téma}{Z}
\btbinfo{Tam a zpět}{tam a zpět}{neomezený,volný , 3 -5 minut (podle počtu hráčů)}{libovolný (4 a více) 
}{Prostředí, slovo pro každého hráče 
}{Z}
\btbinfo{Televizní komentátor}{televizní komentátor}{3 minuty}{2x3 
}{Téma, pro každé studio 
}{Z}
\btbinfo{Toaster}{toaster}{cca 6 minut  
}{4-7}{Prostředí 
}{Z}
\btbinfo{Trojitá stíhačka}{trojitá stíhačka}{6x40s 
}{Dva týmy nebo dvě dvojice}{Téma 
}{Z}
\btbinfo{Tříhlavý song}{tříhlavý song}{cca 3 min}{3}{první verš básně}{I}
\btbinfo{Tři židle}{tři židle}{5-15 minut}{3 
}{není nutné 
}{I}
\btbinfo{Věty, které by neměly zaznít}{věty, které by neměly zaznít}{neomezený}{libovolný 
}{Situace/činnosti (3 - 5) 
}{I}
\btbinfo{Věty z kapsy}{věty z kapsy}{neomezený,volný}{2 
}{Téma, prostředí 
}{Z}
\btbinfo{Vnitřní hlasy}{vnitřní hlasy}{2-3 minuty}{4 
}{Téma 
}{Z}
\btbinfo{Volná}{volná}{cca 10 minut}{2+libovolně 
}{téma 
}{Z}
\btbinfo{Vtip}{vtip}{cca 1 min 
}{alespoň dva}{nemusí být 
}{I}
\btbinfo{Vyprávěná}{vyprávěná}{2 minuty}{neomezen 
}{téma 
}{Z}
\btbinfo{Vysílání FM}{vysílání fm}{volný, obvykle 5 - 10 minut}{5 a více, obvykle se zapojí všichni 
}{Téma 
}{Z}
\btbinfo{Western}{western}{cca 3 minuty}{2+libovolně 
}{téma  
}{Z}
\btbinfo{Zemři}{zemři}{do vyčerpání}{3 - 20 
}{Téma 
}{I}
\btbinfo{Změna}{změna}{2 min}{2 
}{Téma 
}{Z}
\btbinfo{Zpívaná}{zpívaná}{neomezený,volný , 3 -5 minut (podle počtu hráčů)}{libovolný (3 a více) 
}{Téma, prostředí 
}{Z}
\btbinfo{Zvuky}{zvuky}{2-3 minuty}{2 
}{Téma 
}{I}
\btbinfo{Žánrová vyprávěná}{žánrová vyprávěná}{3 minuty}{4 
}{Téma, 4 žánry 
}{Z}
\btbinfo{Ženy vs muži (kategorie)}{ženy vs muži (kategorie)}{2*2minuty}{skupina mužů, skupina žen 
}{dvouslovné téma/situace 
}{Z}
\btbinfo{Živé obrazy}{živé obrazy}{neomezený, obvykle 2-3 minuty}{neomezený 
}{Prostředí 
}{I}
\btbinfo{Živé rekvizity}{živé rekvizity}{2 minuty}{2 
}{Prostředí, dva dobrovolníci 
}{Z}
\btbinfo{1000 způsobů jak}{1000 způsobů jak}{volný, dokud nedojdou nápady}{libovolný, obvykle se zapojí všichni 
}{Činnosti (3-5) 
}{I}
\btbinfo{4, 2, 1}{4, 2, 1}{1 min}{4 
}{Téma nebo prostředí 
}{I}

\end{document}

\end{longtable}
\pagebreak
\renewcommand{\btbinfo}[6]{
\ifx Z#6 #1   \fi

} 
\begin{multicols}{3}
 \begin{tabular}[t]{|p{25mm}|p{25mm}    |}
 \hline
 \multicolumn{2}{|c|}{ } \\
 \multicolumn{2}{|c|}{\large{Zápas}} \\
 \multicolumn{2}{|c|}{ } \\
\hline
\multicolumn{2}{|l|}{ Rozhodčí: }   \\ \hline
\multicolumn{2}{|l|}{ Konferenciér: }   \\ \hline
\multicolumn{2}{|l|}{ Hudba: }   \\ \hline
\multicolumn{2}{|l|}{ Světla: }   \\ \hline
\multicolumn{2}{|l|}{ Organizace: }   \\ \hline
\multicolumn{2}{|l|}{ Grafika: }   \\ \hline
Tým: &  Tým:\\ \hline
1. \small{(cpt)}  & 1.\small{(cpt)} \\ \hline
2.  & 2. \\ \hline
3. & 3. \\ \hline
4.  & 4. \\ \hline
Body:  & Body: \\ \hline

\end{tabular}
\vspace{3mm}
\documentclass[main.tex]{subfiles}\begin{document}
\btbinfo{Abeceda}{abeceda}{2 min}{2 
}{Téma 
}{I}
\btbinfo{Barman song}{barman song}{neomezeně}{neomezeně 
}{Problém pro každého hráče 
}{Z}
\btbinfo{Báseň od publika}{báseň od publika}{cca 6 min}{2 
}{Několik(4) slov pro každého hráče 
}{Z}
\btbinfo{Break-up song}{break-up song}{neomezený, 3 - 5 minut (podle počtu hráčů)}{2 
}{Dobrovolník z publika, židle 
}{I}
\btbinfo{Co tím chceš jako říct}{co tím chceš jako říct}{volně, více jak 3 minuty}{2+ 
}{Téma, prostředí 
}{Z}
\btbinfo{Červená knihovna}{červená knihovna}{2-3 minuty}{libovolný 
}{Téma 
}{Z}
\btbinfo{Čtverec}{čtverec}{neomezen}{4 
}{4 * prostředí, a 1 téma navíc 
}{Z}
\btbinfo{Dabing ABC}{dabing abc}{neomezený}{3 
}{Téma 
}{Z}
\btbinfo{Dabing filmu}{dabing filmu}{omezen ukázkou}{2-3 
}{Téma 
}{Z}
\btbinfo{Dabovaná}{dabovaná}{2 minuty}{4 
}{Téma 
}{Z}
\btbinfo{Diapozitivy}{diapozitivy}{3 minuty}{6+ 
}{Téma 
}{Z}
\btbinfo{Divadlo poezie}{divadlo poezie}{4 minuty}{4 a více 
}{Téma 
}{I}
\btbinfo{Dopis}{dopis}{30 vteřin, 3 minuty 
}{2, následně neomezeno 
}{Obvykle dvě postavy(kdo a komu píše) 
}{Z}
\btbinfo{Dopředu, dozadu}{dopředu, dozadu}{neomezený}{2 a naskakující 
}{Téma 
}{I}
\btbinfo{Duál}{duál}{2 min. 
}{2 a více 
}{Téma}{Z}
\btbinfo{Duet}{duet}{volný, do 30 s 
}{dvojice 
}{jednoslovné téma nebo předmět pro každou dvojici 
}{Z}
\btbinfo{Dům a zahrada}{dům a zahrada}{neomezený}{libovolný (ideálně 3 na každé straně) 
}{Téma, prostředí 
}{Z}
\btbinfo{Dva na pět}{dva na pět}{neomezený}{dva 
}{prostředí 
}{Z}
\btbinfo{Dva začátky}{dva začátky}{5 minut}{libovolný 
}{Téma 
}{I}
\btbinfo{Dvě repliky}{dvě repliky}{3 minuty}{libovolný (3 a více) 
}{Dvě věty pro kažého (až na jednoho) hráče 
}{I}
\btbinfo{Dvojitá stíhačka}{dvojitá stíhačka}{4x40s 
}{Dva týmy}{Téma 
}{Z}
\btbinfo{Emocionální autobus}{emocionální autobus}{neomezen}{1+3-6 
}{nic 
}{I}
\btbinfo{En face}{en face}{3 min 
}{4-6 
}{Téma 
}{Z}
\btbinfo{Facebookový status}{facebookový status}{3  minuty}{libovolný  
}{Poslední facebookový status některého diváka 
}{I}
\btbinfo{Filmové žánry}{filmové žánry}{neomezený, 5 - 15 minut}{2 + 2n 
}{Téma, několik existujících filmových a divadelních žánrů 
}{Z}
\btbinfo{Handicapovaná pohádka}{handicapovaná pohádka}{podle vybrané pohádky}{5 
}{Známá pohádka a handicap pro každého hráče 
}{I}
\btbinfo{Jestli víš, co tím myslím}{jestli víš, co tím myslím}{2  minuty}{2+  
}{Prostředí 
}{I}
\btbinfo{Kam zmizel ten starý song}{kam zmizel ten starý song}{cca 3 minuty na každou píseň}{libovolný (3 a více) 
}{Téma, povolání/specializace 
}{Z}
\btbinfo{Kaskadéři}{kaskadéři}{orientačně 3 min}{4 (dva páry) 
}{Téma 
}{I}
\btbinfo{Komiks}{komiks}{cca 1 minuta}{4 
}{Téma 
}{Z}
\btbinfo{Kramářská píseň}{kramářská píseň}{4 minuty}{libovolný (4 a více) 
}{Téma 
}{I}
\btbinfo{Krátká báseň}{krátká báseň}{volný, max minuta}{4 
}{téma či název básně 
}{I}
\btbinfo{Kvartet}{kvartet}{volný 
}{4 
}{jednoslovné téma nebo předmět 
}{Z}
\btbinfo{Loutky}{loutky}{3 minuty}{sudý počet, 4 a více 
}{Téma 
}{Z}
\btbinfo{Love song}{love song}{neomezený, 3 - 5 minut (podle počtu hráčů)}{2 
}{Dobrovolník z publika, židle 
}{Z}
\btbinfo{Metafory}{metafory}{volný, dokud nedojdou nápady}{libovolný, obvykle se zapojí všichni 
}{Dva názvy předmětů 
}{Z}
\btbinfo{Mikrofon}{mikrofon}{neomezený,volný}{libovolný 
}{Téma 
}{Z}
\btbinfo{Monolog}{monolog}{neomezený,volný}{libovolný 
}{Téma 
}{Z}
\btbinfo{Muzikál}{muzikál}{5-15 minut}{2 - 5 
}{Slovní téma 
}{I}
\btbinfo{Nátlak (kategorie)}{nátlak (kategorie)}{3 minuty, či do splnění úkolu}{2 
}{úkol pro jednoho hráče 
}{Z}
\btbinfo{Občanky}{občanky}{volný, dokud nedojdou nápady}{3-4 
}{Občanské průkazy od diváků podle počtu hráčů 
}{I}
\btbinfo{Opilecká píseň}{opilecká píseň}{volný}{4 
}{Téma 
}{Z}
\btbinfo{Oprsklá}{oprsklá}{neomezen 
}{2 
}{téma/prostředí/vztah/situace 
}{I}
\btbinfo{Orákulum}{orákulum}{volný}{všichni 
}{otázka 
}{I}
\btbinfo{Počet slov}{počet slov}{3 minuty}{libovolný (3 a více) 
}{Téma,počet pro každého hráče (ten může být přidělený) 
}{Z}
\btbinfo{Poetická}{poetická}{3 minuty}{2 
}{Téma 
}{Z}
\btbinfo{Polovina času}{polovina času}{Např 60, 30, 15, 7.5, 3.75, 1.8   
}{2 a více 
}{téma}{Z}
\btbinfo{Poslední věta}{poslední věta}{do nalezení pointy, 5-10 minut}{dvojice (2 a více) 
}{Téma a prostředí 
}{Z}
\btbinfo{Prskavka}{prskavka}{30 vteřin 
}{libovolný, každý sám za sebe 
}{jednoslovné téma pro každého hráče 
}{Z}
\btbinfo{Prskavka ala šanson}{prskavka ala šanson}{cca 40 - 90 vteřin na každého hráče 
}{libovolný, každý jednotlivě 
}{Reálné předměty z publika 
}{Z}
\btbinfo{Překlad do znakové řeči}{překlad do znakové řeči}{3 min}{libovolný (3 a více) 
}{Téma,odbornost experta 
}{Z}
\btbinfo{Příběh po slovech}{příběh po slovech}{volný}{libovolný,minimálně 2 
}{Téma 
}{I}
\btbinfo{Pyramida}{pyramida}{volný}{4-6 
}{Prostředí, Téma nebo činnost 
}{Z}
\btbinfo{Reklamace}{reklamace}{neomezený, volný}{4 
}{Co se bude reklamovat 
}{Z}
\btbinfo{Režiséři}{režiséři}{neomezen}{neomezen}{Téma pro každého z režisérů}{I}
\btbinfo{Rozhlasová hra}{rozhlasová hra}{neomezený}{všichni 
}{Téma 
}{Z}
\btbinfo{Ruce}{ruce}{cca 8 minut}{4+ (nebo 4 a MC) 
}{Bláznivý vynález 
}{Z}
\btbinfo{Ryba (ve čtyřech)}{ryba (ve čtyřech)}{3 minuty 30 vteřin}{4 
}{Prostředí 
}{Z}
\btbinfo{Scénář / Souboj s knihou}{scénář / souboj s knihou}{3 minuty}{2 
}{Připravená kniha či scénář 
}{I}
\btbinfo{Sci-fi}{sci-fi}{cca 3 minuty}{2+libovolně 
}{téma  
}{Z}
\btbinfo{Shakespeare pozpátku}{shakespeare pozpátku}{cca 10 minut}{neomezený 
}{Téma 
}{I}
\btbinfo{Schizofrenie}{schizofrenie}{2 min.}{4}{téma + charakter pro každého hráče}{Z}
\btbinfo{Smrt v 1 minutě}{smrt v 1 minutě}{60 vteřin}{2 
}{Prostředí 
}{Z}
\btbinfo{Spoon river}{spoon river}{Neomezený, bývá 5 a více minut, typicky alespoň na 3 promluvy za každého hráče. 
}{4-6 
}{Dvojslovné téma, prostředí 
}{Z}
\btbinfo{Sportovní komentátor}{sportovní komentátor}{3 minuty}{3 
}{Vrcholový sport 
}{Z}
\btbinfo{S rekvizitou}{s rekvizitou}{2 minuty}{2 
}{Reálný předmět , Prostředí 
}{Z}
\btbinfo{Statusy}{statusy}{volný}{4 
}{Téma a lístky s čísly 
}{I}
\btbinfo{Stíhačka}{stíhačka}{1:20, 40s na tým}{2, plus naskakující 
}{Téma 
}{Z}
\btbinfo{Stojí, leží, sedí}{stojí, leží, sedí}{2 min}{3}{Téma}{I}
\btbinfo{Šumlovaná}{šumlovaná}{2 min 
}{2 a více 
}{Téma}{Z}
\btbinfo{Šumlovaná se simultánním překladem}{šumlovaná se simultánním překladem}{2 min 
}{2 a více 
}{Téma}{Z}
\btbinfo{Tam a zpět}{tam a zpět}{neomezený,volný , 3 -5 minut (podle počtu hráčů)}{libovolný (4 a více) 
}{Prostředí, slovo pro každého hráče 
}{Z}
\btbinfo{Televizní komentátor}{televizní komentátor}{3 minuty}{2x3 
}{Téma, pro každé studio 
}{Z}
\btbinfo{Toaster}{toaster}{cca 6 minut  
}{4-7}{Prostředí 
}{Z}
\btbinfo{Trojitá stíhačka}{trojitá stíhačka}{6x40s 
}{Dva týmy nebo dvě dvojice}{Téma 
}{Z}
\btbinfo{Tříhlavý song}{tříhlavý song}{cca 3 min}{3}{první verš básně}{I}
\btbinfo{Tři židle}{tři židle}{5-15 minut}{3 
}{není nutné 
}{I}
\btbinfo{Věty, které by neměly zaznít}{věty, které by neměly zaznít}{neomezený}{libovolný 
}{Situace/činnosti (3 - 5) 
}{I}
\btbinfo{Věty z kapsy}{věty z kapsy}{neomezený,volný}{2 
}{Téma, prostředí 
}{Z}
\btbinfo{Vnitřní hlasy}{vnitřní hlasy}{2-3 minuty}{4 
}{Téma 
}{Z}
\btbinfo{Volná}{volná}{cca 10 minut}{2+libovolně 
}{téma 
}{Z}
\btbinfo{Vtip}{vtip}{cca 1 min 
}{alespoň dva}{nemusí být 
}{I}
\btbinfo{Vyprávěná}{vyprávěná}{2 minuty}{neomezen 
}{téma 
}{Z}
\btbinfo{Vysílání FM}{vysílání fm}{volný, obvykle 5 - 10 minut}{5 a více, obvykle se zapojí všichni 
}{Téma 
}{Z}
\btbinfo{Western}{western}{cca 3 minuty}{2+libovolně 
}{téma  
}{Z}
\btbinfo{Zemři}{zemři}{do vyčerpání}{3 - 20 
}{Téma 
}{I}
\btbinfo{Změna}{změna}{2 min}{2 
}{Téma 
}{Z}
\btbinfo{Zpívaná}{zpívaná}{neomezený,volný , 3 -5 minut (podle počtu hráčů)}{libovolný (3 a více) 
}{Téma, prostředí 
}{Z}
\btbinfo{Zvuky}{zvuky}{2-3 minuty}{2 
}{Téma 
}{I}
\btbinfo{Žánrová vyprávěná}{žánrová vyprávěná}{3 minuty}{4 
}{Téma, 4 žánry 
}{Z}
\btbinfo{Ženy vs muži (kategorie)}{ženy vs muži (kategorie)}{2*2minuty}{skupina mužů, skupina žen 
}{dvouslovné téma/situace 
}{Z}
\btbinfo{Živé obrazy}{živé obrazy}{neomezený, obvykle 2-3 minuty}{neomezený 
}{Prostředí 
}{I}
\btbinfo{Živé rekvizity}{živé rekvizity}{2 minuty}{2 
}{Prostředí, dva dobrovolníci 
}{Z}
\btbinfo{1000 způsobů jak}{1000 způsobů jak}{volný, dokud nedojdou nápady}{libovolný, obvykle se zapojí všichni 
}{Činnosti (3-5) 
}{I}
\btbinfo{4, 2, 1}{4, 2, 1}{1 min}{4 
}{Téma nebo prostředí 
}{I}

\end{document}
\end {multicols}
\renewcommand{\btbinfo}[6]{
\ifx Z#6  \else 
\large{#1 }

  \fi
}
\hline

\begin{multicols}{3}
\large{\textbf{Veřejné rozcvičky}}

\subfile{rozcvickyshort}
\end {multicols}
\hline

\begin{multicols}{3}
\large{\textbf{Kategorie na improshow}}

\subfile{boxtable}
\end {multicols}


\pagebreak

\begin{tabular}[t]{|p{3cm}|p{4cm}|p{1cm}|p{6cm}    |}
\subfile{faultable}
\end{tabular}
\pagebreak
\setcounter{tocdepth}{1}
\tableofcontents
\end{document}