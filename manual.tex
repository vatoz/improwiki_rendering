\documentclass[main.tex]{subfiles}\begin{document}
\needspace{5cm} \section{Manuál pomocného rozhodčího} \label{manuál pomocného rozhodčího} Role \odkaz{pomocného rozhodčího}{pomocný rozhodčí} bývá při \odkaz{zápasech}{zápas} podceňována, a bývají do ní obsazováni nejméně zkušení \odkaz{hráči}{hráč}. Bývají dva pomocní rozhodčí. 
 
Obvykle jsou pomocňáci třeba již před představením  k přípravě sálu  - rozmisťování židliček, přípravě \odkaz{hlasovacích kartiček}{hlasovací kartička} a \odkaz{papučí}{papuče} na jednotlivá místa, 
rozvěšením bodovacích tabulek, pomocí \odkaz{technikovi}{osvětlovač} s ozkoušením mikrofonů apod. 
 
Často také prodávají lístky - pak je nutné diváky přivítat, prvodivákům vysvětlit lístečky s tématy a předat hlasovací karty a papuče (pokud již nejsou připravené na jednotlivých místech). 
 
Těsně před zápasem pak od diváků vybírají napsaná témata a dávají je do košíku na témata. 
 
Na scénu přichází na vyzvání konferenciéra těsně před \odkaz{hlavním rozhodčím}{rozhodčí} - krátce poté, co se hráči vrátili z \odkaz{veřejné rozcvičky}{veřejná rozcvička}. 
 
Jejich hlavní rolí je pomoci rozhodčímu a hráčům k hladkému průběhu zápasu. Proto je vhodné, když se ještě před zápasem domluví: 
\begin{itemize}
\item  kdo z nich bude \odkaz{ukazovat čas}{ukazování času}  (ten by se to měl rychle doučit)
\item  kdo bude nosit košíček s tématy
\item  a kdo bude kterému \odkaz{týmu}{tým} počítat hlasy diváků, a následně přidávat body či trestné body za \odkaz{fauly}{faul}.
\end{itemize}
 
Po celou dobu zápasu pomocňák setrváva v charakteru - v postavě, která nadevše podporuje hlavního rozhodčího (HR) - takže tedy vrhá se do cesty papučí vrhaných na HR, mračí se na hráče, kteří s HR nesouhlasí apod. To mu ovšem nebrání hrát drobné \odkaz{statusové}{status} hry s druhým pomocňákem, a v reakcích na vše, co se děje, si budovat vlastní osobnost. Pokud ovšem nejsou zapotřebí, měly by se odebrat někam stranou, a nebrat \odkaz{fokus}{fokus} ostatním na scéně. 
 
Pokud HR určí čas na kategorii ( nebo u kategorií, kde je čas vždy např. - \odkaz{Smrt v jedné minutě}{smrt v jedné minutě},\odkaz{Polovina času}{polovina času}, \odkaz{Prskavka}{prskavka},  \odkaz{Ryba (ve čtyřech)}{ryba (ve čtyřech)}, \odkaz{Stíhačka}{stíhačka}, \odkaz{Dvojitá stíhačka}{dvojitá stíhačka}, \odkaz{Trojitá stíhačka}{trojitá stíhačka} (u stíhaček je ukazováni času poněkud speciální, doporučujeme prostudovat)), jeden pomocňák ukazuje čas, ovšem toto ukazování je informací pro hráče a HR, teprve hvizd píšťalky HR kategorii ukončuje. 
 
Ve chvíli kdy kategorie skončila, HR obvykle přiděluje trestné body za fauly, pomocňáci je promptně přidávají na skórovací tabuli a v případě, že jeden tým nasbírá tři trestné body, přidělí místo nich  druhému týmu jeden kladný bod, a informují hlavního rozhodčího, který vše může vysvětlit divákům. 
 
Následuje divácké hlasováni, kdy každý pomocňák sčítá jednu barvu a pak oznamuje výsledek HR, ten pak divákům (i pomocňákům) oznamuje, který tým získal kladný bod. Pomocňáci promptně bod týmu přidělí. 
 
Ve chvíli, kdy HR požádá o hlasovací témata (často žádá jen nataženou paží), pomocňák pověřený nošením košíčku měl by témata ihned HR přinésti. 

\end{document}