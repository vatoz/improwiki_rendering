\documentclass[main.tex]{subfiles}\begin{document}
\needspace{5cm} \section{Hráč} \label{hráč} Označení pro jedince na scéně, kteří jsou zodpovědní za tvorbu obsahu improvizace. Mezi hráče nepatří \odkaz{MC}{mc} a \odkaz{hudebníci}{hudebník}. V rámci improvizačních představení se preferuje výraz "\textbf{hráč}{}"{} nad výrazem "\textbf{herec}{}"{} i když mají téměř stejný význam. Všichni jedinci na scéně jsou herci, ale hráči jsou jedinci zastupující vždy svůj tým (zápas) či jsou vyčleněni od \odkaz{MC}{mc} (improshow).  
 
\subsection{ Zápasová forma } Na \odkaz{zápasech}{zápas} jsou hráči rozděleni do \odkaz{týmů}{tým}. 
 
\subsection{ Zápasový oděv } Na veřejnou rozcvičku jsou hráči oděni čistě v černém (bez výšivek, potisků apod),  
na samotný zápas si pak berou trika v barvách svých týmů. 
Trikot tedy zahrnuje 
\begin{itemize}
\item ponožky a kalhoty   případně leginy/punčocháče (40 DEN a víc) a sukni v barvě černé
\item triko v barvě černé
\item triko v barvě týmu.
\end{itemize}
V případě dlouhých vlasů je takřka nutnost gumička do vlasů, jinak nebývá skrz bujnou hřívu vidět do obličeje. 
 
 
 
 
 
\needspace{5cm} \section{Hudebník} \label{hudebník} Hudebník je skvělým oživením každého zápasu či improshow. 
Jeho role je proměnlivá, při \odkaz{hlasování}{hlasování} na zápasech hraje výrazně až freneticky. 
Ve všech \odkaz{zpívaných kategoriích}{:kategorie:zpívané kategorie} je určovatelem/hybatelem charakteru písní. 
A v dalších kategoriích může podkreslovat. 
 
Skvělé je mít několik nástrojů (či klávesy s přepínáním rejstříků) a volit dle charakteru děje/ hlasu hráčů, či jen tak pro zpestření. 
Je vhodné, když, zejména pro méně hudebně zkušené hráče, zvýrazní první dobu a případně i ukáže kývnutím. 
 
\subsection{Role hudebníka} Na hudebníka může být mnoho různých požadavků. Záleží na oraganizátorovi představení, formátu představení, zkušenostech hudebníka,  zkušenostech herců a pod. Jeho role může být: 
\begin{itemize}
\item \textbf{Hudební podkres mimo improvizaci}{} - Před začátkem představení, během úvodu, vyhlášení pauzy, hlasování, ukončení přestavení. Žánr záleží na cílené atmosféře celého představení, ale obvykle by měla být hudba pozitivní, příjemná, veselá až radostná. Pokud jde o jasné oznámení začátku nebo konce, je vhodná i fanfára.
\item \textbf{Podkresová hudba}{} - Slouží pro doplnění nálady a atmosféry během improvizace. Neměla by brát \odkaz{fokus}{fokus} hercům.
\item \textbf{Krátké znělky (jingly)}{} - Mezi větami či souslovími, například v kategoriích \odkaz{1000 způsobů jak}{1000 způsobů jak}, \odkaz{Věty, které by neměly zaznít}{věty, které by neměly zaznít} nebo v rozcvičce \odkaz{Jsem, beru}{jsem, beru}.
\item \textbf{Doprovod písně}{} - Píseň by měla začít jasnou předehrou, ve které hudebník deklaruje tóninu, (jednoduchý) rytmus, tempo a ideálně i (jednoduchý) akordový postup, kterého se po zbytek písně více méně drží. Možnosti variací jsou velmi závislé na schopnostech herce a na jeho souhře s hudebníkem. Experimentování se nedoporučuje, pokud si hudebník není opravdu jistý hercovými schopnostmi.
\item \textbf{Zvukové efekty}{} - Klepání na dveře, vyzvánení telefonu nebo třeba pípající přístroj v nemocnici improvizaci okoření.
\item \textbf{Ukončení improvizace}{} - Improvizaci může ukončit i hudebník. Zvláště, pokud byla právě řečena skvělá pointa na závěr, je to chvíle, kdy dát hercům případně \odkaz{MC}{mc} hudbou najevo, že je vhodný okamžik improvizaci ukončit. V písních má hudebník na starosti ukončování ve většině případů.
\item \textbf{Nový impulz do improvizace}{} - Ve vhodné chvíli, může hudebník výrazně změnit styl doprovodu a přinést tak do improvizace impulz, kterého se herci chytí. Je důležité tuto změnu udělat tak výrazně, aby si toho herci všimli. Vhodná situace je například ve chvíli, kdy se v příběhu dlouho nic neděje.
\item \textbf{Umět nehrát}{} - Do některých kategorií se hudba hodí více, do některých méně. Vždy je ale důležité nezapomínat na využití ticha a na přehlcení hudbou.
\end{itemize}
 
\subsection{Schopnosti a dovednosti} \begin{itemize}
\item \textbf{Poznat, kdy hrát a kdy nehrát}{}
\item '''Hrát v \odkaz{náladě}{hudební nálady}, která se hodí''' nebo aspoň neodporuje
\item \textbf{Umět improvizovat.}{} Naučený repertoir má své limity.
\item \textbf{Umět zahrát v mnoha náladách}{}, nejen \textif{veselý - smutný}{}
\item '''Umět zahrát v mnoha hudebních žánrech''' a doprovodit mnoho filmových a literárních žánrů
\item \textbf{Umět zahrát improvizovanou píseň}{}, která je čitelná pro herce
\item '''Umět pomoci hercům dodržet rytmus, tóninu''', vrátit se zpátky do rytmu, tóniny, nebo se podle herců přizpůsobit.
\item \textbf{Neplést herce složitostmi}{}
\end{itemize}
 
\subsection{Kdy hrát a kdy nehrát} Některé kategorie hudbu přímo vyžadují, v některých by hudba překážela. Je dobré si hudební doprovod dvakrát rozmyslet v kategoriích, které jsou založené na zvucích odjinud, jako \odkaz{Zvuky}{zvuky} nebo \odkaz{Dabing filmu}{dabing filmu}. Dále se hudba příliš nehodí do kategorií s rychlým spádem (\odkaz{Schizofrenie}{schizofrenie}) a obecně kategorií náročných na pozornost (\odkaz{Sportovní komentátor}{sportovní komentátor}). Do kategorií založených na gagování (\odkaz{1000 způsobů jak}{1000 způsobů jak}, \odkaz{Metafory}{metafory}) se hodí pouze oddělování znělkou (jinglem), do zpívaných kategorií je zase vhodné nehrát mezi písničkami, aby nebylo hudby příliš. Hudba se naopak hodí pro kategorie, které vybízejí ke dlouhému příběhu, ke kategoriím, kde se střídají prostředí (\odkaz{Poslední věta}{poslední věta}) nebo úhly pohledů (\odkaz{Spoon river}{spoon river}) pro jasné odlišení. Hudební podkres se dá použít ke gradaci v kategoriích, které jinak hudbu příliš nevyžadují (\odkaz{Polovina času}{polovina času}, \odkaz{Smrt v 1 minutě}{smrt v 1 minutě}). Hudbou se dá vyplnit prázdný prostor nebo naopak podtrhnout silný okamžik, je ale dobré rozvrhnout si množství hudby do celého představení, aby hudba nehrála pořád. V \odkaz{Longformách}{longforma} má hudba na starosti úvod, závěr, předěly a má podkreslovat důležité okamžiky, ale zároveň má nechat prostor na většinu dialogů pouze hercům.  
 
{{Todo| 
\begin{itemize}
\item  Doplnit navyklé způsoby komunikace mezi hudebníkem a hráčem
\end{itemize}
}} 
 
\needspace{5cm} \section{Konferenciér} \label{konferenciér} Konferenciér řídí průběh \odkaz{zápasu}{zápas} či \odkaz{improshow}{improshow}. 
 
\subsection{Běžné chyby} \begin{itemize}
\item rozvleklý úvod
\item chybný či nepřehledný popis kategorií
\end{itemize}
 
 
 
\needspace{5cm} \section{MC} \label{mc} \textbf{MC}{} z anglického \textbf{Master of Ceremony}{}. \odkaz{Konferenciér}{konferenciér}/moderátor/\odkaz{rozhodčí}{rozhodčí}. Prostě ten, kdo má pod pantoflem celý průběh \odkaz{zápasu}{zápas}/\odkaz{improshow}{improshow}. 
Na improshow někdy bývá "kolující MC"{} - \odkaz{hráči}{hráč} se v řízení průběhu střídají. 
 
 
\needspace{5cm} \section{Pomocný rozhodčí} \label{pomocný rozhodčí} \textbf{Pomocný rozhodčí}{}, též \textbf{pomocňák}{} je typicky k vidění na \odkaz{zápasech v improvizaci}{zápas}. 
Bývají dva, jejich úkolem je \odkaz{měřit čas}{ukazování času} kategorií, sčítání \odkaz{hlasů}{hlasování} (\odkaz{kartičky}{hlasovací kartička}), aktualizace skóre a faulů, podávání košíčku s tématy. Jejich pomoc \odkaz{hlavnímu rozhodčímu}{rozhodčí} někdy může zahrnovat zapisování a sledování \odkaz{faulů}{faul}, které rozhodčí pískne.  
 
Pomoci vám může i \odkaz{Manuál pomocného rozhodčího}{manuál pomocného rozhodčího}. 
\subsection{Běžné chyby} \begin{itemize}
\item Pomocňáci si nedohodnou, který sčítá body kterému týmu a sčítání (typicky první na zápase, či první po přestávce) se musí opakovat.
\end{itemize}
 
 
 
\needspace{5cm} \section{Rozhodčí} \label{rozhodčí} \textbf{Rozhodčí}{} je klíčovou postavou improvizačních \odkaz{zápasů}{zápas}, obvykle tou s nejvyším \odkaz{statusem}{status} na scéně. 
Během zápasu zejména vybírá \odkaz{kategorie}{kategorie}, píská \odkaz{fauly}{faul}, uděluje  \odkaz{trestné body}{trestný bod} a  
dále vede průběh zápasu ve spolupráci s \odkaz{konferenciérem}{konferenciér}. 
Pro některá svá nepopulární rozhodnutí bývá terčem hodu \odkaz{papučí}{papuče} či míčků od diváků. 
 
 
Způsoby, kterými může rozhodčí ovlivňovat běh zápasu 
\begin{itemize}
\item Pískání \odkaz{faulů}{faul} a přidělování \odkaz{trestných bodů}{trestný bod}
\item Vyloučení z následující kategorie/kategorií (např. za opakovanou chybu)
\item Výběr témat (\odkaz{Askfor}{askfor})
\item Výběr \odkaz{kategorií}{kategorie}
\item Zadávání dodatečných úkolů \odkaz{hráčům}{hráč}
\item Zastavení improvizace a např. hlasování diváků o dalším postupu
\item Úkolování \odkaz{hudebníka}{hudebník}, \odkaz{pomocných rozhodčí}{pomocný rozhodčí} a \odkaz{konferenciéra}{konferenciér}
\item Další verbální a nonverbální výlevy
\end{itemize}
 
 
\needspace{5cm} \section{Trenér} \label{trenér} Trenérem je myšlen jedinec, soustavně pracující s \odkaz{týmem}{tým} na jeho zdokonalování v improvizaci na pravidelných 
\odkaz{tréninzích}{trénink}. 
Ve starším pojetí \odkaz{zápasů}{zápas} byl rovněž v průběhu zápasu přítomen na scéně a koučoval hráče. 
  

\end{document}