\documentclass[main.tex]{subfiles}
 
\begin{document}
\needspace{5cm} \section{Hráč} \label{hráč} Označení pro jedince na scéně, kteří jsou zodpovědní za tvorbu obsahu improvizace. Mezi hráče nepatří \odkaz{MC}{mc} a \odkaz{hudebníci}{hudebník}. V rámci improvizačních představení se preferuje výraz "\textbf{hráč}{}"{} nad výrazem "\textbf{herec}{}"{} i když mají téměř stejný význam. Všichni jedinci na scéně jsou herci, ale hráči jsou jedinci zastupující vždy svůj tým (zápas) či jsou vyčleněni od \odkaz{MC}{mc} (improshow).  
 
\subsection{ Zápasová forma } Na \odkaz{zápasech}{zápas} jsou hráči rozděleni do \odkaz{týmů}{tým}. 
 
\subsection{ Zápasový oděv } Na veřejnou rozcvičku jsou hráči oděni čistě v černém (bez výšivek, potisků apod),  
na samotný zápas si pak berou trika v barvách svých týmů. 
Trikot tedy zahrnuje 
\begin{itemize}
\item ponožky a kalhoty   případně leginy/punčocháče (40 DEN a víc) a sukni v barvě černé
\item triko v barvě černé
\item triko v barvě týmu.
\end{itemize}
V případě dlouhých vlasů je takřka nutnost gumička do vlasů, jinak nebývá skrz bujnou hřívu vidět do obličeje. 
 
 
 
 
 
\needspace{5cm} \section{Hudebník} \label{hudebník} Hudebník je skvělým oživením každého zápasu či improshow. 
Jeho role je proměnlivá, při \odkaz{hlasování}{hlasování} na zápasech hraje výrazně až freneticky. 
Ve všech \odkaz{zpívaných kategoriích}{:kategorie:zpívané kategorie} je určovatelem/hybatelem charakteru písní. 
A v dalších kategoriích může podkreslovat. 
 
Skvělé je mít několik nástrojů (či klávesy s přepínáním rejstříků) a volit dle charakteru děje/ hlasu hráčů, či jen tak pro zpestření. 
Je vhodné, když, zejména pro méně hudebně zkušené hráče, zvýrazní první dobu a případně i ukáže kývnutím. 
 
{{Todo| 
\begin{itemize}
\item  Rozvést příklady, kdy je žádoucí proložit \odkaz{formu}{forma} hudbou
\item  Doplnit navyklé způsoby komunikace mezi hudebníkem a hráčem
\end{itemize}
}} 
 
\needspace{5cm} \section{Konferenciér} \label{konferenciér} Konferenciér řídí průběh \odkaz{zápasu}{zápas} či \odkaz{improshow}{improshow}. 
 
\subsection{Běžné chyby} \begin{itemize}
\item rozvleklý úvod
\item chybný či nepřehledný popis kategorií
\end{itemize}
 
 
 
\needspace{5cm} \section{MC} \label{mc} \textbf{MC}{} z anglického \textbf{Master of Ceremony}{}. \odkaz{Konferenciér}{konferenciér}/moderátor/\odkaz{rozhodčí}{rozhodčí}. Prostě ten, kdo má pod pantoflem celý průběh \odkaz{zápasu}{zápas}/\odkaz{improshow}{improshow}. 
Na improshow někdy bývá "kolující MC"{} - \odkaz{hráči}{hráč} se v řízení průběhu střídají. 
 
 
\needspace{5cm} \section{Pomocný rozhodčí} \label{pomocný rozhodčí} \textbf{Pomocný rozhodčí}{}, též \textbf{pomocňák}{} je typicky k vidění na \odkaz{zápasech v improvizaci}{zápas}. 
Bývají dva, jejich úkolem je \odkaz{měřit čas}{ukazování času} kategorií, sčítání \odkaz{hlasů}{hlasování} (\odkaz{kartičky}{hlasovací kartička}), aktualizace skóre a faulů, podávání košíčku s tématy. Jejich pomoc \odkaz{hlavnímu rozhodčímu}{rozhodčí} někdy může zahrnovat zapisování a sledování \odkaz{faulů}{faul}, které rozhodčí pískne. 
\subsection{Běžné chyby} \begin{itemize}
\item Pomocňáci si nedohodnou, který sčítá body kterému týmu a sčítání (typicky první na zápase, či první po přestávce) se musí opakovat.
\end{itemize}
 
 
 
\needspace{5cm} \section{Rozhodčí} \label{rozhodčí} \textbf{Rozhodčí}{} je klíčovou postavou improvizačních \odkaz{zápasů}{zápas}, obvykle tou s nejvyším \odkaz{statusem}{status} na scéně. 
Během zápasu zejména vybírá \odkaz{kategorie}{kategorie}, píská \odkaz{fauly}{faul}, uděluje  \odkaz{trestné body}{trestný bod} a  
dále vede průběh zápasu ve spolupráci s \odkaz{konferenciérem}{konferenciér}. 
Pro některá svá nepopulární rozhodnutí bývá terčem hodu \odkaz{papučí}{papuče} či míčků od diváků. 
 
 
Způsoby, kterými může rozhodčí ovlivňovat běh zápasu 
\begin{itemize}
\item Pískání \odkaz{faulů}{faul} a přidělování \odkaz{trestných bodů}{trestný bod}
\item Vyloučení z následující kategorie/kategorií (např. za opakovanou chybu)
\item Výběr témat (\odkaz{Askfor}{askfor})
\item Výběr \odkaz{kategorií}{kategorie}
\item Zadávání dodatečných úkolů \odkaz{hráčům}{hráč}
\item Zastavení improvizace a např. hlasování diváků o dalším postupu
\item Úkolování \odkaz{hudebníka}{hudebník}, \odkaz{pomocných rozhodčí}{pomocný rozhodčí} a \odkaz{konferenciéra}{konferenciér}
\item Další verbální a nonverbální výlevy
\end{itemize}
 
 
\needspace{5cm} \section{Trenér} \label{trenér} Trenérem je myšlen jedinec, soustavně pracující s \odkaz{týmem}{tým} na jeho zdokonalování v improvizaci na pravidelných 
\odkaz{tréninzích}{trénink}. 
Ve starším pojetí \odkaz{zápasů}{zápas} byl rovněž v průběhu zápasu přítomen na scéně a koučoval hráče. 
  
\end{document}