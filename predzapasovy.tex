\documentclass[main.tex]{subfiles}\begin{document}
\needspace{5cm} \label{předzápasový trénink} Týmy se scházejí zpravidla 2 hodinky před zápasem, aby se na sebe mohli hráči, hudebník, rozhodčí a konferenciér naladit na předzápasovém tréninku. Je to také čas, kdy si týmy projdou kategorie, protože některé kategorie mají více variant. 
 
Předzápasový trénink by měl vést jiný než hrající člověk, protože pak on sám není rozehřátý jako ostatní a může se cítit mimo tým. 
 
Na tréninku by se měl \odkaz{trenér}{trenér} zaměřit na témata jako je společné naladění, důvěra mezi hráči, energii týmu, rozdýchání a rozmluvení, rozjetí fantazie a nápadů. Nemá smysl se zaobírat dovednostmi, které hráčům chybí, protože na to není před zápasem čas a hráči se mohou zablokovat, protože vidí, že jim nějaká činnost nejde. 
Některé týmy ani nezkouší kategorie kromě zpívané, protože, když se kategorie nepovede, může to vést ke špatně náladě týmu. Nevadí však, když si poslední cvičení vezme na konci tréninku samotný rozhodčí. 
 
Je vhodné, když cvičení na tréninku jsou hráčům známá, nemusíme tolik času trávit vysvětlování pravidel. 
Hráči i zpravidla vědí na co je cvičení zaměřené a mohou se věnovat jejich naplňování. 
 
 
\section{ Ukázka cvičení předzápasového tréninku: }  
 
\begin{itemize}
\item Honička se speciálními pravidly (na rozehřátí těla, hráči se dostávají do přítomnosti)
\item Dýchání ve dvojících a uvědomování si těžiště, kdy hráči držením těla mají společné těžiště (uvědomování si svého těla, těla partnera a těžiště pomáhá odstraňovat nervozitu, dýchání je důležité pro hlas, ve dvojicích také hráči získávají důvěru mezi sebou)
\item \odkaz{Gordický uzel}{gordický uzel} (cvičení zaměřený na kontakt, důvěru ve skupinu, že rozpletení společně vyřeší)
\item Společná masáž, vlastní masáž obličeje (společná masáž pomáhá zvyknout si na dotyk ostatních, uvolňuje svalové napětí, dělá dobře)
\item Hlasové cvičení (zajistí sílu hlasu a správné používání hlasivek)
\item \odkaz{Asociace}{asociace (cvičení)} (rozehřívá myšlení, cvičení může být obohaceno o pravidlo např., že všechna slova začínají na jedno písmeno apod.)
\item \odkaz{Jsem a beru}{jsem, beru} (do slov zapojujeme pohyb, pomáhá zapojovat obě hemisféry)
\item Počítání do dvaceti (pomáhá skupinu naladit se na společnou vlnu, zvládnout společně úkol a soustředění se)
\item Společná báseň (hráči si zkouší spolupráci, brát nápady od ostatních a posouvat je dál)
\item \odkaz{Ready Go}{ready go} (hráči si zkouší přejímat emoce a posouvat příběh dál)
\item Jamování (hráč se zaměřují na společný rytmus, doplňování hudby pomocí svého těla)
\item Společná píseň (naladění se na hudebníka, který si také může ověřit, jak jsou na tom hráči se zpěvem)
\item \odkaz{Kdo ukradl sušenky a ze stolu je vzal}{kdo ukradl sušenky a ze stolu je vzal} (rytmické a energetické cvičení na závěr)
\end{itemize}
 
Na závěr předzápasového tréninku hráči s \odkaz{rozhodčím}{rozhodčí} vybírají ze seznamu kategorií ty, které umí oba týmy, případně se vybírají varianty těchto kategoríí. 
 
 

\end{document}