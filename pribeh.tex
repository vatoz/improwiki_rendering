\documentclass[main.tex]{subfiles}
 
\begin{document}\needspace{5cm} \section{Expozice} \label{expozice} {{Fáze příběhu}} 
 
\textif{Expozice}{} je úvod \odkaz{příběhu}{příběh}, který většinou bývá na začátku\footnote{Některé příběhy mohou mít opožděnou expozici a nebo ji naleznete až na konci příběhu. viz kategorie \odkaz{Shakespeare pozpátku}{shakespeare pozpátku}}. Obecně se expozicí vytváří předpoklady k pochopení příběhu. Zpravidla to znamená nastavení \odkaz{prostředí}{prostředí} představení hlavních \odkaz{postav}{postava} a nastavení vztahů mezi postavami. 
 
\subsection{ Běžné chyby }  
\subsubsection{ Prolínání úvodních fází příběhu } Velice běžná chyba, se kterou se potýká každá třetí improvizace. Například pokud se stane, že při expozici definujeme nějaký příběhový problém. Expozice není dokončena, vyprávění poskočí do další fáze a vzniká zmatek na pódiu. 
 
\subsubsection{ Příliš mnoho detailů } U \odkaz{krátkých forem}{:kategorie:krátké formy} improvizace by expozice příběhu neměla být zdlouhavá. Vypravěč nebo vypravěči by se měli držet \odkaz{improvizačního trojzubce}{improvizační trojzubec}. Často se stává, že vypravěči nadefinují více \odkaz{rekvizit}{rekvizita} nebo \odkaz{postav}{postava}, než které je možné v příběhu zahrnout. Výsledkem je zmatený divák, který nechápe, kam některé rekvizity nebo postavy zmizely. 
 
  
 
 
 
\needspace{5cm} \section{Kolize} \label{kolize} {{Fáze příběhu}} 
 
\textif{Kolize}{} je fáze \odkaz{příběhu}{příběh}, při které dochází k zařazení problému, konfliktu nebo rozporu do hlavní příběhové linky. Cílem kolize je vyvolání napětí a určení, kterým směrem se celý \odkaz{příběh}{příběh} bude dále vyvíjet. Předpokladem pro funkční kolizi je definování \odkaz{postav}{postava}.\footnote{\odkaz{Postavy}{postava} se definují obvykle v \odkaz{expozici}{expozice}.} 
 
\subsection{ Běžné chyby }  
\subsubsection{ Okamžité řešení } V \odkaz{krátkých formách}{:kategorie:krátké formy} by se nemělo stát, že se v \odkaz{příběhu}{příběh} objeví problém a okamžitě se vyřeší, protože takový problém nemá žádnou příběhovou hodnotu. Výsledkem okamžitého řešení je ztráta 20 vteřin, protože se v \textif{kolizi}{} musí pokračovat, dokud se nějaký konflikt nebo problém nenajde. V \odkaz{delších improvizačních formách}{:kategorie:delší formy} to smysl má, protože je zajímavá a příběhově hodnotná reakce postav na problém, který se řeší sám. 
 
\subsubsection{ Povrchní konflikt } Po \textif{kolizi}{} zpravidla přichází další fáze \odkaz{příběhu}{příběh}, kde se s problémem, což je výstup \textif{kolize}{}, nějakým způsobem pracuje. Pokud nemá problém nalezený v této fázi nějakou hloubku, může se stát, že další fáze \odkaz{příběhu}{příběh} budou působit povrchně. Způsobů, jak nastavit "hloubku"{} je více a mohou se kombinovat: 
 
\begin{itemize}
\item  Problém by neměl nepřesahovat \odkaz{žánr}{:kategorie:žánry} \odkaz{příběhu}{příběh}
\item  Postavy by měly mít citovou vazbu k problému, "stává se to osobním"
\item  Problém se dotýká přímo publika - například situace, kterou divák řeší nebo řešil
\end{itemize}
 
{{todo|Pomozte prosím doplnit další způsoby, jak postavám pořádně okořenit problém}} 
 
  
 
 
 
\needspace{5cm} \section{Krize} \label{krize} {{Fáze příběhu}} 
 
Zjednodušeně je \textif{krize}{} prostor v \odkaz{příběhu}{příběh} pro "pořádné pohrabání se v problému", který byl vytvořen v jiné fázi \odkaz{příběhu}{příběh}. Divácky je to nejdůležitější fáze, protože už jsou nastaveny všechny předpoklady pro vžití se do děje. Zároveň je to fáze, ve které dostávaji hlavní postavy hodně prostoru. 
 
 
\subsection{ Průběh } Příběh v této fázi popisuje a rozvádí, čím vším musí hlavní hrdinové projít, než vyřeší problém, tedy výstup \odkaz{kolize}{kolize}. Zpravidla zde dochází k zhoršování problému a oddalování řešení. V různých žánrech funguje prohlubování problému jinak, ale obecně to v divácích vyvolává nějakou empatii k postavě, která se buď stává bezmocnou a nebo jí někdo jen komicky zlomyslně ubližuje. Pro diváky '''jsou nejdůležitější reakce hlavní postavy''' na vývoj problému a každé rozhodnutí, které udělá.  
 
 
\subsection{ Běžné chyby }  
\subsubsection{ Úplné přeskočení fáze } \odkaz{Krátké formy}{:kategorie:krátké formy} vedou improvizátory ke spěchu a tak se občas stane, že improvizátor skočí od zadání problému ihned k jeho řešení a příběh ztratí hodnotu, která by byla vytvořena \textif{krizí}{}. Tato chyba se dá opravit \odkaz{střihem}{střih} nebo posunutím času \odkaz{příběhu}{příběh}. V \odkaz{delších formách}{:kategorie:delší formy} k této chybě zpravidla nedojde, protože \textif{krize}{} tvoří více jak polovinu \odkaz{příběhu}{příběh}. 
 
\subsubsection{ Neschopnost práce s výstupem kolize } Kolikrát už jste viděli, že problém který byl nastaven \odkaz{kolizí}{kolize}, se nakonec v \odkaz{příběhu}{příběh} vůbec neobjevil? A kolikrát jste na něj, improvizátoři, úplně zapomněli? Problém se v této fázi nesmí ani zapomenout, ztratit, vyřešit, nic. Pouze prohloubit a rozvíjet. 
 
\subsubsection{ Zbytečná expozice } Velice často potkáváme improvizované příběhy, ve kterých se při krizi začnou objevovat další \odkaz{postavy}{postava} nebo \odkaz{rekvizity}{rekvizita}, které nebyly v expozici vůbec zmíněny. Je to přirozený proces - improvizátor si řekne, že by se v \odkaz{příběhu}{příběh} hodilo letadlo, tak ho tam doplní. Chyba, která se velice často opakuje je \textbf{přehnání}{} této vnořené \odkaz{expozice}{expozice}, kdy improvizátoři stráví polovinu času \textif{krize}{} definováním nových \odkaz{postav}{postava} nebo \odkaz{rekvizit}{rekvizita}. 
 
\subsubsection{ Ztráta na váze } V této fázi je opravdu nejdůležitější prohlubování problému. Pokud \odkaz{hráč}{hráč} hrající hlavní postavu nedokáže adekvátně zareagovat na situaci, problém najednou ztratí na váze a celý \odkaz{příběh}{příběh} se stane povrchním. 
 
\begin{quote} 
"Jsem tvůj otec!" 
"Tak mi to podrž." 
\end{quote} 
 
 
 
 
\needspace{5cm} \section{Peripetie} \label{peripetie} {{Fáze příběhu}} 
 
\textif{Peripetie}{} je fáze příběhu, ve které hlavní \odkaz{postavy}{postava} po dlouhé \odkaz{krizi}{krize} nacházejí možnosti řešení. V této fázi může dojít u \odkaz{postav}{postava} k takzvanému "change of heart"\footnote{Anglická fráze přeložitelná jako fundamentální změna názoru}. 
 
\subsection{ Průběh } V této fázi příběh nabízí hlavním \odkaz{postavám}{postava} způsoby, jak se dostat z \odkaz{krize}{krize}, čímž je zároveň definována podoba \odkaz{katarze}{katarze} celého \odkaz{příběhu}{příběh}. Mělo by dojít k urovnání všech příběhových referencí a přípravě k rozuzlení celého děje. Hlavní postava se dostává informace, rekvizity nebo které chyběly k vyřešení problému. 
 
Pro diváka je v této fázi důležitá \textbf{změna}{} v myšlení nebo chování hlavních postav, které k nabídnutým řešením zaujmou nějaký postoj. 
 
\subsection{ Běžné chyby }  
Klasické chyby postav, jako nedůvěryhodnost role nebo nespojení informací dohromady, dokáží anulovat tuto fázi. 
 
  
 
 
\needspace{5cm} \section{Katarze} \label{katarze} {{Fáze příběhu}} 
 
\textif{Katarze}{} je zakončení příběhu. Divák se dozvídá, jak \odkaz{příběh}{příběh} končí. 
 
\subsection{ Průběh }  
V ideálním případě by se všechny \odkaz{příběhové}{příběh} reference (výstup \odkaz{krize}{krize} a \odkaz{peripetie}{peripetie}) v této fázi měly spojit dohromady a mělo by dojít k nějakému řešení výstupu \odkaz{kolize}{kolize}. Je více přístupů, jak nahlížet na závěr \odkaz{příběhu}{příběh}. 
 
\subsubsection{ Dobro a zlo } Dopadne to dobře nebo špatně ve vztahu k hlavním \odkaz{postavám}{postava}? Skončí to katastrofou a nebo rozuzlením? Dokáže se hrdina poprat s problémem a nebo spáchá sebevraždu? Obojí je řešení problému. 
 
\subsubsection{ Monolog } Hlavní \odkaz{postava}{postava} nebo \odkaz{vypravěč}{vypravěč} vypráví, co bylo dál. Typické pro \odkaz{vyprávěnou}{vyprávěná} a \odkaz{noir}{noir}. 
 
\subsubsection{ Otevřený konec } Pokud hrdinové získají všechny informace nebo \odkaz{rekvizity}{rekvizita}, které potřebovali a dá se logicky vydedukovat přímé pokračování příběhu, dá se skončit otevřeně. 
 
\subsubsection{ Láska za každou cenu } Hlavní postava zjistí, že konflikt nebo problém se dá vyřešit tím, že se změní. 
 
\subsubsection{ Level up } Hlavní postavy obrátí celou krizi v osobní prospěch. 
 
\begin{itemize}
\item  \begin{quote}"Sice je válka, ale já na tom vydělám."\end{quote}
\item  \begin{quote}"Moje žena stejně byla proti polygamii."\end{quote}
\end{itemize}
 
\subsubsection{ Krveprolití } Brutální odstranění všech problémových postav, typické pro \odkaz{western}{western} a \odkaz{červenou knihovnu}{červená knihovna}. 
 
 
\subsection{ Běžné chyby }  
\subsubsection{ Není čas! } V \odkaz{krátkých formách}{:kategorie:krátké formy} se může stát, že na závěr příběhu nezbyde čas. V takovém případě může buď \odkaz{MC}{mc} čas přidat a nebo se dá volně přejít do \odkaz{monologu}{katarze}. 
 
\subsubsection{ Neukončené příběhové linky } Pokud příběh obsahuje více linek, je mnohem zajímavější dohrát postupně všechny, než ukončit jenom jednu. Po konci jedné linky by se mělo přepnout do prostředí, kde se odehrává další a zahrát její \textif{katarzi}{}. 
 
 
 
\end{document}