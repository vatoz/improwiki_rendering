\documentclass[main.tex]{subfiles}\begin{document}
\needspace{5cm} \label{příprava zápasu} Je li z vás \textbf{Garant akce}{} - organizátor, zde se dozvíte pár věcí, na které je dobré myslet. 
\section{Rozdělení rolí účinkujících} \section{Garant akce} Garant akce je člověk, který zašťiťuje celkovou organizaci. Má veškeré informace. Rozděluje role ostatním a vykoná všechno, co nemůže rozdělit. Typicky na veřejných akcích zastává roli konferenciéra, a nebo uvaděče. Pokud je člověk přizván pořadatelem většího celku, jako jsou festivaly, zápas druhého týmu, je výhodou pokud přímo garant akce sestavuje tým hráčů. Dostává tak vedle velké zodpovědnosti i dostatečné kompetence. 
 
\section{Trenér} Člověk v roli \odkaz{trenéra}{trenér} má za úkol připravit trénink před akcí. Tato role se často slučuje s rolí rozhodčího. 
\section{Hudebník} Pokud není formátem akce stanoveno jinak, tak tato role hudebně doprovází improvizaci po celou dobu představení. 
Je důležité, aby byl \odkaz{hudebník}{hudebník} dostupný už na tréninku před akcí. 
Pozor na dopravu nástroje na místo. Může se stát, že hudebník nebude schopen svůj nástroj sám dopravit na místo. 
 
\section{Konferenciér nebo rozhodčí} Tyto dvě role se řídí \odkaz{manuálem konferenciéra}{manuál konferenciéra} a \odkaz{manuálem rozhodčího}{manuál rozhodčího}, další informace jsou v heslu \odkaz{Zápas}{zápas}. 
Je dobré, aby věděli 
\begin{itemize}
\item Kdo hraje (seznam \odkaz{hráčů}{hráč})
\item Kdo je \odkaz{hudebník}{hudebník}
\item Od kdy do kdy se hraje
\item Kategorie, které se mohou nebo budou hrát
\item Téma představení, pokud nějaké je
\end{itemize}
 
\section{Zvukař} Zvukař má na starosti ovládání hlasitosti hudebních nástrojů a mikrofonů po celou dobu představení. Jeho role se obvykle spojuje s osvětlovačem. Je důležité, aby věděl všechny technické parametry prostoru. 
V případě nedostupnosti některých pomůcek, jako například mikrofonů, má zvukař na starosti zařídit jejich dostupnost. 
Pokud bude potřeba přehrávat nějaká média (například empétrojky a nebo pouštění videí na plátno), řeší technické problémy s tím spojené. 
Může požadovat dopravu techniky. 
\section{Osvětlovač} Osvětlovač má na starosti ovládání světel a vedlejšího osvětlení prostoru po celou dobu představení. Jeho role se obvykle spojuje se zvukařem. Je důležité, aby věděl všechny technické parametry prostoru. 
Může požadovat dopravu techniky. 
\section{Pomocňák nebo pomocný rozhodčí} \odkaz{Pomocní rozhodčí}{pomocný rozhodčí}  mají na starosti přípravné práce před začátkem představení a obvykle během tréninku před představením. Garant by se s nimi měl dohodnout, aby přijeli před začátkem. 
Samozřejmě pomůže jejich proškolení pomocí \odkaz{Manuálu pomocného rozhodčího}{manuál pomocného rozhodčího}. 
 
\section{Hráč} Hráč by měl mít na starost pouze dopravit sám sebe včas spolu s oděvem vhodným pro konkrétní akci. 
\section{Prostory k hraní} Prostor k hraní musí být zajištěn předem. K prostoru je potřeba vědět všechny technické věci 
\begin{itemize}
\item Velikost pódia včetně nepřemistitelných překážek
\item Dostupnost dvou mikrofonů
\item Jaké mají osvětlení
\item Jestli se dá úplně zhasnout v sále (\odkaz{vysílání FM}{vysílání fm}, \odkaz{Rozhlasová hra}{rozhlasová hra})
\item Kapacita hlediště
\item Časové možnosti – od kdy do kdy se může hrát
\item Dostupnost odposlechů
\item Sezení pro diváky
\item Plátno a projektor (\odkaz{Dabing filmu}{dabing filmu})
\end{itemize}
 
\section{Propagační nástroje} \begin{itemize}
\item Plakát
\item Akce na Facebooku a Google Plus, Twitter
\item Vstupenky
\item Newsletter
\end{itemize}
 
\section{Časová omezení} Organizace akce by včetně tisku a rozvěšení plakátů měla být dokončena nejméně týden před představením. 
\subsection{Sumarizace pro účinkující} Uvést v bodech 
\begin{itemize}
\item Kde se hraje
\item Jak se tam jede
\item Kdo hraje
\item Co si má kdo vzít sebou
\item Individuální úkoly
\item After party
\end{itemize}
\section{Improzápas} \subsection{Pomůcky} \begin{itemize}
\item Tabule skóre pro každý tým (čísla 0-9) a kolíčky na trestné body
\item Píšťalka a kazoo
\item Hlasovací kartičky v barvách týmových dresů (Je nutné domluvit dostatečně předem)
\item Papuče nebo míčky na házení
\item Papírky na témata
\item Půjčovací tužky na témata
\item Košík na témata
\item Kostým pro rozhodčího a pomocňáky
\end{itemize}

\end{document}