\documentclass[main.tex]{subfiles}
 
\begin{document}\needspace{5cm} \section{Běh přes překážky} \label{běh přes překážky} Hráči stojí na zadní hraně jeviště, čekají na start závodu. Na pokyn \odkaz{MC}{mc} Připravit se, ke startu, pozor, teď   zakleknou do bloků a odstartují běh na místě. V průběhu běhu překonají (ideálně všichni najednou)přeskokem tři překážky, po té třetí se okamžitě přepínají do zpomaleného pohybu, během kterého se snaží dosáhnout přední hrany jeviště a zároveň zabránit ostatním hráčům tohoto cíle dosáhnout pomocí hraných útoků, úderů apod. 
 
\subsection{Varianta - štafeta} Tato varianta se obzvlášť hodí při malé scéně/velkém počtu hráčů. 
Na zadní hraně jeviště jsou dvě řady hráčů, přední zakleknutá na zemi, druhá stojící. Po odstartování běží na místě pouze zadní řada, po třetí překážce předávají štafetový kolík přední řadě která pokračuje dál, již ve zpomaleném režimu. 
 
 
 
 
\needspace{5cm} \section{Emocionální autobus} \label{emocionální autobus} \katabox{nic 
}{1+3-6 
}{neomezen} 
 
V našem \textbf{Emo autobusu}{} přebírá řidič a ostatní hráči (cestující) emoci/charakter toho, kdo právě nastoupil jako poslední.  
 
\subsection{Průběh} První na scéně je řidič, přichází k autobusu v nějaké \odkaz{emoci}{emoce}/charakterové vlastnosti/charakterové vadě a vydává se na cestu. Ve chvíli kdy zastaví a nabere pasažéra, přebírá od pasažéra jeho emoci/charakter. Při nabírání dalšího pasažéra přebírá jeho emoci i ostatní cestující. Když jsou v autobusu všichni hráči, začnou postupně vystupovat v opačném pořadí než nastupovali s tím, že autobus přebírá původní charakter toho, kdo bude vystupovat. 
 
\subsection{Varianta} \begin{itemize}
\item Když je autobus plný, všichni hráči vystoupí na další zastávce. Postupně, zachovává se stejný princip, jen se zrychlí konec kategorie.
\end{itemize}
 
\subsection{Scéna} Bývají připravené židličky. 
 
 
\needspace{5cm} \section{Horská dráha} \label{horská dráha} Hráči na pokyn nastoupí do vozíku horské dráhy - např. 6 lidí ve třech řadách, 
již při nastupování dbají na prostorové rozmístění sedadel a vrátek. 
Na hráče se zaklapnou zábrany, a začíná jízda, kterou hráči ztvárňují nakláněním do stran či dopředu/dozadu. 
 
\needspace{5cm} \section{Hospodská rvačka} \label{hospodská rvačka} Hráči rozehrají jednoducho scénu z baru/hospody. V jednu chvíli se scéna zvrtne ve rvanici, v tu chvíli všichni pokračují ve \odkaz{zpomaleném pohybu}{zpomalený pohyb}, dokud nezůstane poslední přeživší (stojící) hráč. 
 
 
 
\needspace{5cm} \section{Jsem, beru} \label{jsem, beru} Klasické cvičení, používané často na tréninzích i ve veřejných rozcvičkách. 
\subsection{Průběh} První hráč nabíhá, zaujme nějakou pózu, \odkaz{štronzo}{štronzo} , a pojmenuje se. 
 
Nabíhá druhý hráč, a zapojí se do obrazu, jak pozicí tak ztvárněným předmětem.  
 
Dtto třetí hráč. 
První hráč slovy Beru ... odvolá sebe a jednoho z dalších hráčů, ten následně začíná další kolo. 
\subsection{Varianty} \begin{itemize}
\item Odvolávání tlesknutím jednoho z čekajících hráčů, který následně tvoří základ dalšího obrazu
\item Veršovaně
\item Předmět, problém, řešení
\item Zapojí se všichni
\item Ze zadaného prostředí
\item Všichni se musí dotýkat
\item Beze slov
\item S opakováním  pohybu dokreslujícího danou věc
\end{itemize}
 
 
 
 
 
 
\needspace{5cm} \section{Krátká báseň} \label{krátká báseň} \katabox{téma či název básně 
}{4 
}{volný, max minuta} 
 
 
Hráči vám poví krátkou báseň na zadané téma 
 
 
\subsection{ Průběh } Hráči postupně přicházejí a rýmují na sebe \odkaz{ABAB}{abab}. 
Každý hráč řekne jeden verš.  
 
Rozcvička se obvykle opakuje dvakrát nebo třikrát. 
\subsection{ Varianty } \begin{itemize}
\item  \odkaz{MC}{mc} určí počet slok, po zaznění každé slok obvykle \odkaz{hudebník}{hudebník} proloží báseň krátkou vyhrávkou. Pořadí hráčů se v dalších slokách nemusí držet.
\end{itemize}
 
\subsection{ Běžné chyby } \begin{itemize}
\item  Běžné veršovací chyby
\end{itemize}
 
 
 
 
\needspace{5cm} \section{Orchestrion} \label{orchestrion} Na scénu přichází první hráč a začíná opakovat rytmický či melodický motiv. Připojí se druhý a následně i třetí hráč. Ve chvíli, kdy přijde čtvrtý hráč, první hráč odchází (ideálně po postupném utlumení vlastního motivu) a je připrave na scénu naběhnout znova. 
Po chvíli je možné jako vlastní motiv zpívat frázi z jednoho či dvou slov, držící se vytvořeného témata. 
 
Je vhodné, když alespoň jeden hráč drží nějaký základní rytmus, který pak všichni ostatní využívají. 
 
 
 
 
\needspace{5cm} \section{Roztočený brankář} \label{roztočený brankář} Základní improvizační cvičení na procvičení asociací, reakce a komunikace. 
 
Hráči stojí ve větším kruhu či v půlkruhu. Jeden hráč vstoupí do středu či před stojící hráče a začne se pohybovat. Pohyb není nijak omezen, může zaujímat rozličné pozice (sedat si, lehat si, postavit se na hlavu, ...), ale měl by být stále v pohybu. Kdokoli z ostatních hráčů může kdykoli během jeho pohybu tlesknout, což je pro pohybujícího pokyn \odkaz{štronzo}{štronzo}. Hráč, který tleskl zaujme vůči svému kolegovi též nějakou pozici, kterou mu pozice \odkaz{asociuje}{asociace} a rozehraje situaci. Hráč opouští štronzo a pokračuje také v rozehrané situaci. Hráči by měli co nejdříve objasnit kdo jsou, kde jsou a co tam dělají. Jakmile je toto objasněno, může opět tlesknout kdokoli z přihlížejících hráčů, což je opět povel tentokrát pro oba hrající hráče \odkaz{štronzo}{štronzo}. Hráč, který tentokrát tleskl přichází ke dvojici, dotykem určí hráče, který odejde, zaujme opět pozici k hráči, který zůstává a opět rozehrají jinou situaci. Toto cvičení se dá libovolně dlouho opakovat. 
 
 
 
 
\needspace{5cm} \section{Samuraj} \label{samuraj} Ze skupiny hráčů je vybrán samuraj, může se sám přihlásit nebo je určen cvičícím, který hráči předává imaginární samurajský meč. Ostatní hráči jsou samurajem ohroženi a snaží se před ním uniknout, ale současně se drží jako semknutá skupina. Zpravidla je skupina v jednom rohu prostoru a samuraj v protilehlém. Samuraj se snaží teatrálně a relativně pomalými pohyby meče ostatní hráče zasáhnout. Může mířit na hlavy a hráči se skrčí, může mířit na nohy a hráči nadskočí, případně sekne svisle do skupiny a skupina se rozestoupí, aby zásahu unikla. Pozice samuraje a skupiny se mohou během cvičení také prohodit, když po nezdařeném zásahu skupina proběhne na místo samuraje a samuraj zaujme původní místo skupiny. Závěr cvičení může být rozsekání skupiny na kousíčky zdařenými zásahy samuraje, nebo tím, že samuraj pod tíhou svého neúspěchu (žádný zasažený hráč) spáchá harakiri. 
 
 
 
 
 
\needspace{5cm} \section{Souboj otázek} \label{souboj otázek} V tomto cvičení se hráči rozdělí na dvě skupiny, které stojí proti sobě v zástupu. Cvičící zadá prvním dvěma hráčům prostředí, ve kterém jsou a ti spolu pak musí komunikovat pouze otázkami. První, kdo nepoloží otázku vypadává, jde na konec řady a na jeho místo nastupuje další hráč. Prostředí může cvičící střídat podle svého uvážení. 
 
\subsection{ Běžné chyby } \begin{itemize}
\item Hráči vrství stejné otázky na sebe - "Ty si myslíš, že já si myslím, že ty si myslíš, že jsi přišel pozdě?"
\item Hráči ve svých otázkách nepřináší nic nového.
\end{itemize}
 
  
 
 
 
\needspace{5cm} \section{Stroj} \label{stroj} Skupině je zadáno co bude daný stroj vyrábět nebo k čemu slouží. První hráč poté ztvární pohybující se součástku a doplní ji nějakým ozvučením. Ostatní hráči se postupně přidávají jako další součástky až vznikne celý stroj. Součástky by se měly dotýkat nebo navazovat na akci součástky, která je vedle ní. Když je stroj hotov cvičící zastaví jeho pohyb, ale hráči zůstávají v pozicích a pojmenují své součástky. Poté cvičící může stroj opět spustit a zkoušet různé rychlosti chodu stroje 50\%{}, 100\%{}, 150\%{}, 200\%{} až se stroj porouchá či exploduje. 
 
\subsection{ Varianty } Při velkém prostoru se dá vyzkoušet nekonečný stroj. Hráči opět představují jednotlivé součástky stroje, ale při zapojení všech hráčů ten první přebíhá za posledního a vytváří další originální součástku, atd. V tomto případě odpadá pojmenovávání součástek. 
 
 
 
 
 
\needspace{5cm} \section{Thai Chi} \label{thai chi} Hráči se rozestaví do čtverce (hrany, vrcholy i vnitřek) otočeni všichni stejným směrem. 
Všichni se pohybují stejně, pohyby,cviky i natočení určuje ten, kdo je na vrcholu směrem, kterým se všichni dívají. 
Cílem je vypadat stejně jako lidé cvičící Thai Chi. 
Po chvíli se začnou hráči náhodně střídat ve vyprávění příběhu po krátkých částech vět. 
\subsection{Varianty} \begin{itemize}
\item Čtení dopisu
\item Báseň
\item Vyprávění příběhu
\end{itemize}
\subsection{Běžné chyby} \begin{itemize}
\item Vůdčí hráč se pohybuje příliš rychle/ zaujímá pózy, které ostatní nezvládnou.
\item Hráči nepozorně sledují a kopírují pohyby
\item Vyprávění není plynulé
\end{itemize}
  
\needspace{5cm} \section{Židle} \label{židle} Uprostřed scény stojí židle, ke které se na písknutí ze stran rozeběhnou dva hráči. Ve chvíli, kdy se židle dotknou, na okamžik \odkaz{strnou}{štronzo} a pak sehrají krátkou scénu vycházející z pozic, ve kterých skončili. 
 
\subsection{ Viz také } \begin{itemize}
\item  \odkaz{Roztočený brankář}{roztočený brankář}
\end{itemize}
 
\subsection{ Může se plést s } \begin{itemize}
\item  \odkaz{Dvě židle}{dvě židle}
\item  \odkaz{Tři židle}{tři židle}
\end{itemize}
 
 
\end{document}