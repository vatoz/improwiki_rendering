\documentclass[main.tex]{subfiles}\begin{document}
\needspace{5cm} \section{ABAB} \label{abab} Vzorec básně, ve kterém se rýmují sudé verše se sudými a liché verše s lichými. 
 
Příklad takové básně: 
	 
\begin{quote} 
Někdo se bojí vlastní ženy (A)

 
A někdo třeba upíra (B)

 
Někdo je v přítmí vyděšený (A)

 
Ve tmě už téměř umírá (B) 
 
Jan Burian, Zubař 
\end{quote} 
 
Tento vzorec se používá hlavně v kategoriích \odkaz{Krátká báseň}{krátká báseň} a \odkaz{Opilecká píseň}{opilecká píseň}. 
 
Dál se často používají střídání AABB a ABBA. 
 
 
 
\needspace{5cm} \section{Askfor} \label{askfor} \textbf{Askfor}{} - v překladu např. \textbf{zeptání se}{}. \odkaz{MC}{mc} si vyžádá od publika téma, na které se bude hrát. 
 
Může se jednat o: 
\begin{itemize}
\item \textbf{Košíček s tématy}{}, do kterého publikum před \odkaz{zápasem}{zápas} vhazuje napsaná dvojslovná témata
\item \textbf{Reálné předměty}{}, které diváci zapůjčí na \odkaz{Prskavku ala šanson}{prskavka ala šanson}
\item \textbf{Prostředí}{} udávané např na \odkaz{Toaster}{toaster}, \odkaz{Smrt v 1 minutě}{smrt v 1 minutě} či \odkaz{Poslední větu}{poslední věta}
\item \textbf{Emoce}{} pro \odkaz{Schizofrenii}{schizofrenie}
\item \textbf{Problém}{} pro \odkaz{Barman song}{barman song}
\item \textbf{Jazyk}{} pro \odkaz{šumlovanou}{šumlovaná}, \odkaz{šumlovanou se simultánním překladem}{šumlovaná se simultánním překladem}, \odkaz{duál}{duál} a \odkaz{šumlovaný sbor}{šumlovaný sbor}. Víz také \odkaz{Gibberish}{gibberish}.
\end{itemize}
 
Dobrovolník např. do \odkaz{Love songu}{love song} je svým způsobem také Askfor 
 
 
\needspace{5cm} \section{Asociace (terminologie)} \label{asociace (terminologie)} Asociace je termín, kterým označujeme okamžitý nápad, který se nám vybaví bez přemýšlení. Asociace tak může s prvním zadáním zdánlivě nesouviset např. strom -> ovoce -> počítač (Apple). Improvizátoři trénují svou mysl a paměť, aby byli schopní vždy okamžitě reagovat i když může být zadáním slovo, jehož význam neznají. 
 
 
\needspace{5cm} \section{Báseň} \label{báseň} Báseň v širším slova významu představuje literární dílo tvořené ve verších. 
V improvizaci se s tímto pojmem diváci mohou potkat jako s kategorií \odkaz{Poetická}{poetická} nebo rozcvičkovým cvičením \odkaz{Báseň}{báseň (cvičení)}. 
 
\subsection{Viz. také} \begin{itemize}
\item \odkaz{ABAB}{abab}
\item \odkaz{Divadlo poezie}{divadlo poezie}
\end{itemize}
 
 
\needspace{5cm} \section{Forma} \label{forma} \textbf{Forma}{} je zkrácené označení pro formát představení. Určuje postavy a způsob, kterým budou vystupovat po dobu trvání celého představení. Seznam forem naleznete v \odkaz{kategorii Formy}{:kategorie:formy}. 
 
 
\needspace{5cm} \section{Gibberish} \label{gibberish} \textbf{Giberish}{} je označení pro vymyšlený jazyk. Setkat se můžeme také s českým označením \textbf{šumlovaná}{} nebo také \textbf{svojština}{}. Improvizátor vytváří vlastní řeč, hledá pro něj pravidla. Inspirace pro znění jazyka by mělo vycházet z kontextu situace ve kterém se hráč nachází, jako postavu hraje, jaké jsou jeho emoce apod. 
 
Hráči předstírájí  znalost neznámého cizího jazyka, někdy \odkaz{zadaného publikem}{askfor}. 
 
Giberish může být vhodný nástroj pro rýmování, kdy hráč není zatěžován významem slov, a umožňuje mu soustředit se na rytmiku a výrazovost projevu. 
 
\subsection{Viz také} \begin{itemize}
\item \odkaz{kvaziznaková řeč}{kvaziznaková řeč}
\end{itemize}
 
 
\needspace{5cm} \section{Hlasovací kartička} \label{hlasovací kartička}  
 
Pomůcka, na kterou by se nemělo zapomenout při organizování klasického \odkaz{zápasu}{zápas}. Divák při vstupu do sálu dostane od pomocného rozhodčího či jiné pověřené osoby dvě hlasovací kartičky různých barev nebo jednu s kartičku s každou stranou jiné barvy. Hlasovací kartička je zhruba velikosti A6 většinou z papírového kartonu či zalaminovaná. Barva závisí na hrajících týmech, každý \odkaz{tým}{tým} má svou specifickou barvu či hraje v dresech dané barvy. Divák prostřednictvím hlasovací kartičky dává najevo, kterému \odkaz{týmu}{tým} by přidělil bod v průběhu \odkaz{hlasování}{hlasování} o proběhlé improvizaci. 
 
 
\needspace{5cm} \section{Hlasování} \label{hlasování} Na \odkaz{zápasech}{zápas} se využívá \textbf{hlasování}{}, aby se určilo, který \odkaz{tým}{tým} dostane bod za právě odehranou \odkaz{kategorii}{kategorie}. Organizátor a nebo \odkaz{MC}{mc} určují způsob, jakým mohou diváci hlasovat. 
 
Hlasování se používá jako varianta \odkaz{náhodného rozhodování}{náhodné rozhodování}. 
 
\subsection{ Obvyklé způsoby hlasování } \begin{itemize}
\item  Pomocí \odkaz{hlasovacích kartiček}{hlasovací kartička}
\item  \odkaz{Vteřinovým potleskem}{vteřinový potlesk}
\end{itemize}
 
 
\needspace{5cm} \section{Impro} \label{impro} Zkratka používaná pro divadelní improvizaci. Velice často se také spojuje s jinými slovy, jako \odkaz{improtým}{tým}, \odkaz{improkategorie}{kategorie}, \odkaz{Improtřesk}{improtřesk} 
 
 
\needspace{5cm} \section{Improshow} \label{improshow} Improvizované představení, které je podobné \odkaz{zápasu v improvizaci}{zápas}. Narozdíl od zápasu ale hraje jenom jeden tým a představení nemá \odkaz{rozhodčího}{rozhodčí}, ani \odkaz{pomocňáky}{pomocný rozhodčí}. 
 
\subsection{ Průběh } Improshow není standardizovaný formát představení, na kterém se hrají \odkaz{:Kategorie:Krátké formy}{:kategorie:krátké formy} a cokoliv se může změnit. Obvykle se hrají \odkaz{kategorie}{kategorie} v nějakém pořadí a celým večerem provází \odkaz{konferenciér}{konferenciér}. 
 
 
 
 
 
\needspace{5cm} \section{Improvizační trojzubec} \label{improvizační trojzubec} Pojem označující seřazenou sadu tří základních otázek, na které by si měla umět každá postava v improvizaci odpovědět. Pokud to nedokáže, měla by ve scénce vytvořit věci, které k tomu chybí. 
 
\subsection{ Kdo jste? } Byly všechny postavy dobře popsány? Ví všichni o všech dost na to, aby z toho mohli vycházet? 
 
\subsection{ Kde jste? } Bylo definováno prostředí? Dá se s ním pracovat? 
 
\subsection{ Co tam děláte? } Proč jsme v této situaci? Proč moje postava dělá to, co dělá? Zachovala by se takhle pokaždé? 
 
 
 
\needspace{5cm} \section{Kvaziznaková řeč} \label{kvaziznaková řeč} Způsob komunikace založený na pohybu celým tělem. Každý projev v této řeči je originální, protože pohyby obvykle vycházejí z první asociace na významovou část překládané věty. Tlumočník se obvykle snaží v reálném čase překládat řeč nebo rozhovor. Při překladu z verbální komunikace se nepřekládá každé slovo, ale pouze významová část. 
 
Při překladu se vyjadřuje emoce a její intenzita výrazem v obličeji a rychlostí pohybu. 
 
Tato technika se obvykle používá v kategorii \odkaz{Překlad do znakové řeči}{překlad do znakové řeči} a \odkaz{Sportovní komentátor}{sportovní komentátor}. 
 
 
\needspace{5cm} \section{Motivovaný odchod} \label{motivovaný odchod} Odchod zdůvodněný logikou scény. Např. při scéně s \odkaz{výměnou žárovky}{výměna žárovky} to může být pro vypnutí proudu. Je potřebný zejména pro \odkaz{pyramidu}{pyramida} a \odkaz{Tam a zpět}{tam a zpět}. 
 
 
\needspace{5cm} \section{Náhodné rozhodování} \label{náhodné rozhodování} Na \odkaz{zápasech}{zápas} je potřeba občas náhodně rozhodnout, který tým dostane výhodu. Převážně se jedná o rozhodování, který tým bude začínat s pokřiky a nebo v porovnávacích kategoriích. 
 
\subsection{ Puk } V ideálním případě má rozhodčí k dispozici puk. Kapitání týmů si vyberou, která strana podle nich padne. Výherce si vybírá, jestli chce začít. Občas se jako náhražka používá mince. Hod probíhá pod důsledným dozorem \odkaz{pomocných rozhodčích}{pomocný rozhodčí}. 
 
\subsection{ Náhodný generátor čísel } \odkaz{Rozhodčí}{rozhodčí} se zeptá týmů kdo chce začít a během toho v hlavě vygeneruje náhodné číslo. Tým, jehož hlasitost v decibelech při křičení "Ano"{} je blíže začíná. 
 
\subsection{ Diváci } Vždycky je možnost zeptat se diváků, pro kterou možnost budou \odkaz{hlasovat}{hlasování}. 
 
 
 
\needspace{5cm} \section{Pantomima} \label{pantomima} Pantomima je divadelní způsob vyjádření pomocí mimiky a gestikulace s prvky tance bez použití hlasu.  
 
 
 
\needspace{5cm} \section{Papuče} \label{papuče} Molitanová papuče či lehký míček slouží divákům k vyjádření nesouhlasu s verdiktem hlavního \odkaz{rozhodčího}{rozhodčí}. Papuče dostávají diváci při vstupu do sálu, vždy jeden kus na osobu. Během začátku zápasu instruuje \odkaz{konferenciér}{konferenciér} diváky, jak mohou papuči využít. Vyjádření nesouhlasu zpravidla bývá hlavně při udílení trestných bodů po proběhlé improvizaci, kdy hlavní rozhodčí upozorňuje na fauly, kterých se hráči v průběhu kategorie dopustili. Pokud se divákům zdá faul nespravedlivý nebo hráčů argument dobrý, ale rozhodčím neuznaný, mají možnost hodit papuči či míček na jeviště a tím vyjádřit svůj nesouhlas. Hlavní rozhodčí může vzít tento nesouhlas v potaz a trestný bod neudělit. 
 
\odkaz{Pomocní rozhodčí}{pomocný rozhodčí} obvykle papuče/míčky vrací zpátky divákům jako  přísun další munice.  
Zřídkakdy diváci použijí papuče k vyjádření nesouhlasu s chováním některého hráče. 
 
 
\needspace{5cm} \section{Pískání} \label{pískání} \odkaz{Rozhodčí}{rozhodčí} dává písknutím do kazoo najevo, že došlo k chybě při improvizaci a pohybem těla signalizuje, o který \odkaz{faul}{faul} došlo. 
 
Pomocí obyčejné píšťalky začíná či končí kategorii. 
U některých kategorii, např. \odkaz{Pyramida}{pyramida},\odkaz{Změna}{změna},\odkaz{Čtverec}{čtverec},\odkaz{Poslední věta}{poslední věta} se v průběhu používá jedno hvízdnutí obyčejné píšťalky na přepnutí a až dvojité zahvízdání ukončuje kategorii. 
 
 
\needspace{5cm} \section{Porovnávací kategorie} \label{porovnávací kategorie} Při \odkaz{kategoriích}{kategorie} v tomto seznamu nehrají týmy společně, ale zvlášť. 
 
 
\label{:kategorie:porovnávací kategorie} 
\begin{multicols}{2}\begin{itemize} 
\item  \odkaz{Break-up song}{break-up song}  
\item  \odkaz{Báseň od publika}{báseň od publika}  
\item  \odkaz{Dvojitá stíhačka}{dvojitá stíhačka}  
\item  \odkaz{Love song}{love song}  
\item  \odkaz{Sportovní komentátor}{sportovní komentátor}  
\item  \odkaz{Stíhačka}{stíhačka}  
\item  \odkaz{Televizní komentátor}{televizní komentátor}  
\item  \odkaz{Trojitá stíhačka}{trojitá stíhačka}  
\item  \odkaz{Ženy vs muži (kategorie)}{ženy vs muži (kategorie)}  
\end{itemize} 
\end{multicols} 
\needspace{5cm} \section{Přebírání pozice} \label{přebírání pozice} \textbf{Přebírání pozice}{} je termín, který popisuje situaci kdy je jeden hráč ve \odkaz{štronzu}{štronzo} a jeho pozici se snaží co nejpřesněji zaujmout druhý hráč. Ten by měl zohlednit co nejvíce detailů, případně i mimiku, kterou zaujímá první hráč. Přebírání pozic se používá např. v kategoriích \odkaz{dvojitá stíhačka}{dvojitá stíhačka}, \odkaz{poslední věta}{poslední věta} ale i dalších a různých cvičeních. 
 
 
\needspace{5cm} \section{Předmět} \label{předmět} Předmět může být od diváků vybírán ve formě inspirace pro danou kategorii či cvičení a hráči jej musí během cvičení či improvizace zohlednit. Případně se jeden či více improvizátorů může předmětem stát a my tak máme možnost krom sledování práce s předmětem sledovat i jeho emoční pochody. Předmět v zastoupení hráče ve většině případů nemluví, ale není to podmínkou. 
 
Druhá forma vybírání předmětů od diváků je jejich skutečné vybrání. Diváci mají možnost zapůjčit předmět, který mají u sebe a ten se stává rekvizitou či inspirací pro improvizátory. Po kategorii se předměty divákům vracejí... většinou v pořádku, jídlo ale většinou nepřežije. 
 
Rekvizity od diváků se používají v kategoriích \odkaz{prskavka ala šanson}{prskavka ala šanson}, \odkaz{s rekvizitou}{s rekvizitou}, a další. 
 
 
\needspace{5cm} \section{Reflexe} \label{reflexe} Část po \odkaz{zápase}{zápas} či tréninku, kdy účastníci mají možnost vyjádřit své připomínky k průběhu. 
 
Mezi známé způsoby vedení reflexe patří 
\begin{itemize}
\item každý má půl minuty na to, aby řekl, co ho tíží a těší
\item všichni mají dohromady 3 minuty aby se vyjádřili, osobní zpětné vazby se pak podávají v hospodě
\item kladné kolečko - co se povedlo, záporné kolečko - rady pro příště
\end{itemize}
 
Je vhodné v rámci reflexe nezabředávat do vysvětlování a obskurních diskuzí.    
    
 
\needspace{5cm} \section{Rekvizita} \label{rekvizita} V drtivé většině divadelních her se používají rekvizity. V \odkaz{impru}{impro} se nejčastěji setkáte s imaginární rekvizitou. 
 
 
\subsection{ Imaginární rekvizita } \odkaz{Hráč}{hráč} předvádí \odkaz{pantomimou}{pantomima}, že má u sebe rekvizitu, která reálně neexistuje. Pohybem těla simuluje pohyby, které by dělal, kdyby rekvizitu opravdu měl nebo držel. V závislosti na povaze imaginární rekvizity je dobré ji na začátku pojmenovat nebo s ní provést přirozenou akci\footnote{Akci přirozenou pro rekvizitu, například vyhrožování pistolí}, aby bylo jasné, o jaký předmět se jedná. 
 
\begin{quote}"Improvizátor má u sebe vždycky všechno"\end{quote} 
 
\subsection{ Personifikovaná rekvizita } \odkaz{Hráč}{hráč} předvádí samotnou rekvizitu. Může se v očích diváka stát konkrétním předmětem. Tento druh rekvizity přináší své výhody.  
 
\begin{itemize}
\item  S takovou rekvizitou se dá reálně pohybovat, nedá skrz ni projít
\item  \odkaz{Hráč}{hráč} hrající rekvizitu může předmětu přidávat emoce a výrazy, pokud je předmět uveden v činnost
\end{itemize}
 
Často k vidění v kategorii \odkaz{duety}{duet}. 
 
Podobně v kategorii \odkaz{Živé rekvizity}{živé rekvizity} je divák používán jako rekvizita. 
 
\subsection{ Zástupná rekvizita } Reálný předmět, který v divadelní hře představuje jiný předmět z děje. Například žlutá čepice zastupující královskou korunu. 
 
Často k vidění v kategorii \odkaz{s rekvizitou}{s rekvizitou}. 
 
\subsection{ Reálná rekvizita } Reálný předmět v příběhu. Například "auto na pódiu"{} v příběhu odehrávajícím se v autě. V improvizaci se s touto rekvizitou prakticky nesetkáte. 
 
  
 
 
\needspace{5cm} \section{Replika} \label{replika} Replika je jednotlivá věta/odpověď/reakce. 
 
\needspace{5cm} \section{Rozehřívání diváků} \label{rozehřívání diváků} Na začátku \odkaz{zápasu}{zápas} či \odkaz{improshow}{improshow} se \odkaz{MC}{mc}  snaží vybudit v divácích aktivitu a odezvu. 
Používají se například: 
\begin{itemize}
\item odstupňované učení se tleskat
\item zkušební \odkaz{vyžádání si témat}{askfor}
\item společný zpěv písně
\item vzájemné představování se, objímání či líbání diváků
\end{itemize}
 
 
\needspace{5cm} \section{Status} \label{status} Při kontaktu dvou osob neustále mimovolně probíhají drobné statusové hry, 
jedná se o to, kdo bude mít "navrch". Status ale můžete mít i vůči prostředí, jinak vstoupí do  
dveří ředitel firmy a jinak žebrák. 
 
Jakýkoli příběh oživí drobné statusové hry, či dokonce statusové výměny. 
 
 
Vysoký status bývá rozpoznatelný skrze: 
\begin{itemize}
\item vysoký tonus těla
\item pomalejší pohyby a mluva
\item pevný pohled
\item ruce volně podél těla
\end{itemize}
zatímco pro  nízký status bývají charakteristické 
\begin{itemize}
\item vyhýbavý těkavý pohled
\item klopení hlavy
\item nervózní chichotání
\item neurotická práce s rukama - točení mlýnku palci, upravování si oděvu
\end{itemize}
 
\needspace{5cm} \section{Střih} \label{střih} Cítí li některý z hráčů, že je nutné děj někam více posunout, obvykle tak činí nahlas pronesenou scénickou poznámkou  - např. O hodinu později na letišti.  nebo Ukažme si, jak se kdysi seznámili. 
 
V \odkaz{longformách}{longforma} je množství používaných uznávaných metod, kterými  lze ukončit scénu, podstatně širší. 
Jedná se např. o: 
 
\begin{itemize}
\item scénická poznámka
\item monolog postavy
\item mazání scény rychlým průchodem před ostatními hráči
\item opakování pohybu
\item opakování \odkaz{repliky}{replika}
\item rozehrátí nové scény bez kontaktování ostatních hráčů
\end{itemize}
 
Další změny scény může iniciovat osvětlovač či \odkaz{hudebník}{hudebník}. 
 
 
 
\needspace{5cm} \section{Štronzo} \label{štronzo} Slovní pokyn, občas nahrazovaný písknutím na píšťalku. Hráči se ve chvíli, kdy uslyší slovo \textif{stronzo}{}, zastaví a nemohou (ano, jsou to hráči, takže vlastně nechtějí, i když po chvíli hodně chtějí) se hýbat. Cvičení používané jak na fyzickou zdatnost, tak na zklidnění hráčů a jejich koncentraci. Obvyklým výstupem je následný popis \textif{živých obrazů}{}. Vydržení ve \textif{stronzu}{} je základním znakem síly týmu. A sadismu rozhodčího či trenéra. 
 
Na zápase uvidíte ztuhnutí nejčastěji při rozcvičce, jako funkční prvek se používá  v \odkaz{živých obrazech}{živé obrazy}, \odkaz{pyramidě}{pyramida}, \odkaz{poslední větě}{poslední věta} a \odkaz{čtverci}{čtverec} . 
 
Opakem \textif{štronza}{} je \textif{portamento}{}, tedy rozpohybování se.  
 
 
 
\subsection{ Poznámky }  
Pozor při překladech a komunikaci se zahraničními skupinami. V některých jazycích může být výraz \textif{stronzo}{} \textbf{urážlivý}. 
 
 
\needspace{5cm} \section{Trénink} \label{trénink} O improvizaci říkáme, že ji trénujeme, nikoli zkoušíme, protože se, narozdíl od klasického divadla 
nejedná o neustálé zkoušení/vylepšování konkrétního obsahu, ale 
o soustavnou práci na vylepšování vlastních schopností.  
 
\needspace{5cm} \section{Trestný bod} \label{trestný bod} Trestný bod je udělován na \odkaz{zápasech}{zápas} jako forma trestu za \odkaz{fauly}{faul}. Typicky se vizualizuje připnutím kolíčku k tabulce s počtem bodů daného \odkaz{týmu}{tým}. Za každé tři trestné body je připočítán jeden bod do celkového skóre soupeřícího týmu. 
 
 
\needspace{5cm} \section{Ukazování času} \label{ukazování času} Při \odkaz{zápasech}{zápas} je jednou z hlavních úloh \odkaz{pomocných rozhodčí}{pomocný rozhodčí} měření času a jeho ukazování \odkaz{hráčům}{hráč} a \odkaz{hlavnímu rozhodčímu}{rozhodčí}. 
Pro toto ukazování je několik ustálených gest.  
 
\subsection{Polovina času} Ruce zkřížené v zápěstí. 
 
\subsection{10 vteřin do konce} Kroucení dlaní s roztaženými prsty. 
 
\subsection{5 (4,-1) vteřin do konce} Ukazuje se rukou se zvednutým patřičným počtem prstů. 
 
\subsection{Konec} Kroucení rukou zatnutou v pěst 
 
 
 
 
\needspace{5cm} \section{Veřejná rozcvička} \label{veřejná rozcvička} Na začátku zápasu se hráči rozehřívají před diváky několika jednoduššími cvičeními. Cvičení pomáhá seznámit se s prostorem, světlem, jevištěm i diváky. 
Diváci si naopak také zvykají na hráče, kdy mohou pocítit již první prvky improvizace. 
Existuje několik známějších cvičení, se kterými se může divák i hráč na zápasech potkat. 
 
Moderátor by měl divákům vždy představit pravidla cvičení, aby diváci nezískali pocit, že je vše připravené a mohli se soustředit na obsah, ne na hledání porozumění pravidel. 
 
Seznam cvičení na veřejnou rozcvičku naleznete v kategorii \odkaz{Rozcvičky}{:kategorie:rozcvičky}. 
 
 
\needspace{5cm} \section{Vteřinový potlesk} \label{vteřinový potlesk} \textbf{Vteřinový potlesk}{} je zvláštní formou \odkaz{hlasování}{hlasování}. Diváci v tomto případně nemají hlasovací kartičky a k hlasování používají jen svůj hlas či jiné hlasité projevy. Průběh hlasování je v tomto případě následující: \odkaz{MC}{mc} určí, pro co či jaký tým se bude právě hlasovat, poté odstartuje vteřinový potlesk, během něhož mohou diváci křičet, tleskat, dupat a jinak se projevovat následně MC potlesk ukončí zpravidla gestem a hlasuje se pro druhou variantu či tým. \odkaz{MC}{mc} poté rozhodne, který potlesk byl hlasitější a na jeho základě je rozhodnuto. Toto hlasování je dobré při velmi krátkém čase na hlasování, velkém počtu diváků nebo v případě zapomenutí hlasovacích kartiček. 
 
\subsection{ Varianta } \begin{itemize}
\item Potlesk je měřen decibelometrem a rozhoduje vyšší hodnota.
\end{itemize}
 
 
 

\end{document}