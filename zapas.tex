\documentclass[main.tex]{subfiles}
 
\begin{document}\needspace{5cm} \label{zápas} Zápas v divadelní improvizaci je unikátní divadelní forma, ve které se dva \odkaz{týmy}{tým} utkají v improvizačních \odkaz{kategoriích}{kategorie}.  
Forma je částečně inspirována hokejem. 
 
\section{ Celkový průběh z pohledu diváka }  
\subsection{ Před zahájením } \begin{itemize}
\item  Diváci při příchodu mají možnost napsat na lístečky \odkaz{témata}{askfor}
\item  Pokud není zařízeno jinak, obvykle diváci dostávají \odkaz{hlasovací kartičky}{hlasovací kartička}
\item  Diváci dostávají \odkaz{papuče}{papuče} nebo \odkaz{míčky}{papuče} na vyjádření nesouhlasu s rozhodčím
\end{itemize}
 
\subsection{ Úvod } \odkaz{Konferenciér}{konferenciér}  
\begin{enumerate}
\item  přivítá diváky, trochu je \odkaz{rozehřeje}{rozehřívání diváků} a vysvětlí jim na co vlastně přišli a princip improvizačního zápasu.
\item  zve \odkaz{hráče}{hráč} na \odkaz{veřejnou rozcvičku}{veřejná rozcvička}, hráči jsou ještě v nediferencovaných trikotech zpravidla černých, proběhne několik cvičení, hráči následně odbíhají.
\item  dál informuje diváky, např. o \odkaz{hlasování}{hlasování}, možnosti vyjádření nesouhlasu papučí či míčkem a \odkaz{faulech}{faul}
\end{enumerate}
 
\subsection{ Začátek zápasu } \odkaz{Konferenciér}{konferenciér}  
\begin{enumerate}
\item  zve jednotlivé \odkaz{hráče}{hráč} v \odkaz{týmech}{tým} a představí je, případné zmíní, že některý improvizátor hraje poprvé zápas
\item  zve \odkaz{pomocňáky}{pomocný rozhodčí} (pomocné rozhodčí)
\item  zve \odkaz{hlavního rozhodčího}{rozhodčí} a předává mu hlavní slovo
\end{enumerate}
 
\subsection{ První polovina } \begin{enumerate}
\item  Pokřiky týmů (jediná nacvičená věc)
\item  \odkaz{Odehraje}{zápas}  se několik \odkaz{kategorií}{kategorie}
\item  Přestávka
\end{enumerate}
 
\subsection{ Druhá polovina } \begin{enumerate}
\item  Výměna stran týmů
\item  Voliteně pokřiky a veřejná rozcvička na druhou půli
\item  \odkaz{Hraje}{zápas} se několik dalších \odkaz{kategorií}{kategorie}
\end{enumerate}
 
\subsection{ Závěr } \begin{enumerate}
\item  Ukončení zápasu a vyhlášení vítěze podle skóre
\item  Představení \odkaz{hráčů}{hráč}, \odkaz{pomocných rozhodčích}{pomocný rozhodčí}, \odkaz{hlavního rozhodčího}{rozhodčí}, \odkaz{hudebníka}{hudebník}, \odkaz{konferenciéra}{konferenciér} a ostatních
\item  Pozvánka na další zápasy
\item  Klanění
\item  Volitelně přídavek
\end{enumerate}
 
\section{ Průběh z pohledu účastníků } \begin{itemize}
\item  \odkaz{Předzápasový trénink}{předzápasový trénink}
\item  Domluva s \odkaz{rozhodčím}{rozhodčí}, které \odkaz{kategorie}{kategorie} se budou hrát, případně která varianta.
\item  \odkaz{Zápas (divácká část)}{zápas}
\item  Volitelně loučení s odcházejícími diváky
\item  \odkaz{pozápasová reflexe}{reflexe}
\item  Volitelně osobní neřízená reflexe v restauračním zřízení
\end{itemize}
 
\section{ Průběh kategorie } \begin{enumerate}
\item  \odkaz{Rozhodčí}{rozhodčí} si vyžádá \odkaz{askfor}{askfor} ke kategorii a určí která kategorie se bude hrát, určí čas na přípravu a čas trvání kategorie
\item  \odkaz{Konferenciér}{konferenciér} stručně vysvětlí pravidla dané kategorie
\item  Hraje se \odkaz{kategorie}{kategorie}. \odkaz{Rozhodčí}{rozhodčí} má možnost v průběhu pískat \odkaz{fauly}{faul}. \odkaz{Pomocný rozhodčí}{pomocný rozhodčí} ukazuje hráčům gesty zbývající čas kategorie.
\item  Kategorie je ukončena. \odkaz{Rozhodčí}{rozhodčí} přiděluje \odkaz{trestné body}{trestný bod} za \odkaz{fauly}{faul}. Může si pozvat kapitány a nebo přímo \odkaz{hráče}{hráč}, kterým byl \odkaz{faul}{faul} písknut, aby odůvodnili svůj faul odůvodnili, případně se z něj vykecali. Diváci mají možnost vyjádřit nesouhlas s rozhodčím, např hodem \odkaz{papučí}{papuče}, míčkem či bučením.
\item  Probíhá \odkaz{hlasování}{hlasování} diváků a přidělení vítězných bodů.
\end{enumerate}
 
\section{ Kam dál } \begin{itemize}
\item  \odkaz{Zápasové kategorie}{:kategorie:zápasové kategorie}
\item  \odkaz{Příprava zápasu}{příprava zápasu}
\end{itemize}
 
 
 
 
\end{document}