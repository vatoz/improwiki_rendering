\needspace{5cm} \section{Automatické psaní} \label{automatické psaní} Automatické psaní je proces psaní bez vědomé kontroly autora. V google se vám sice po zadání tohoto hesla objeví spoustu spirituálních článku, tato metoda je však velice uvolňující a pomáhá rozvíjet vaše myšlenky a nápady. 
 
Pro automatické psaní je důležité si najít čas, příjemnou polohu sedu, tužku (nejlépe měkkou normální tužku) a čistý bílý papír.  
Tužku nechat lehce nad papírem, prostě vypnout a nechat ruku psát všechny slova, která vám začnou plynout myslí. 
V rámci cvičení na psaní je možné si jasně omezit časem, třeba na 5 minut. Někomu více vyhovuje trochu stresu. 
 
Opakovaným zkoušením se budete zlepšovat a texty půjdou snadněji. 
Nevadí, pokud budete zapisovat i citoslovce nebo nesmyslná slova. Důležité je, že všechno zaznamenáváte bez jakékoliv vaší cenzury a přemýšlení nad obsahem. 
 
S textem lze i dále pracovat. Například se jím můžete nechat inspirovat pro psaní povídek. 
Text po napsaní si přečtěte, podtrhněte si zajímavá spojení nebo vystřihněte a zkuste z nich složit nový text, který si už budete vědomě řídit. 
\needspace{5cm} \section{Emoce} \label{emoce} Emoce jsou procesy vykládané jako subjektivní zážitky libosti a nelibosti provázané fyziologickými změnami, motorickými projevy (gestikulace, mimika), stavy s různě velkou aktivností a vztahují se vůči něčemu konkrétnímu. Tyto prvky vidíme u ostatních na první pohled, známe je, protože sami pořád prožíváme různé stavy emocí. 
 
A stejně tak i každá postava na jevišti musí vyjadřovat nějakou emoci, aby byla uvěřitelná, pokud zrovna hlavním hrdinou není robot a i ten teoreticky může něco cítit. Často se stává, že hráč ukazuje na jevišti spíše svoje rozpoložení než psychický stav své postavy. Pokud se mu daří přijít v uvědomělém emočním stavu, je často omezený na základní emoce, jako je radost, smutek, vztek, strach, znechucení. Přitom škála našich emocí je daleko širší. 
 
Pokud pracujeme v improvizačních cvičení s emocemi, je dobré vědět: 
\begin{itemize}
\item  emoce má nějaký důvod
\item  u hraní extrémních emocí je důležité, aby hráči ovládali neutrální stav a uměli stavy přepínat
\item  k hraní emocí můžeme jít přes fyzické napětí, gesta, mimiku apod.
\item  sledovat pravidelné dýchání, aby nedocházelo u hráčů k hyperventilaci nebo hypoventilaci
\end{itemize}
 
Existují různě dlouhé seznamy, a je otázka co všechno můžeme považovat za emoci. 
Důležitým parametrem je však to, že emoce má nějakou intenzitu, je časově omezená a je vůči něčemu zaměřena. 
 
\subsubsection{Seznam emocí} Vybrané jsou převážně ty emoce, které se snadno interpretují na scéně: 
 
\begin{multicols}{3} 
\begin{itemize}
\item  Afektovaně
\item  Bázlivě
\item  Blouznivě
\item  Bojácně
\item  Bojovně
\item  Bolestivě
\item  Bouřlivě
\item  Citlivě
\item  Živě
\item  Děsivě
\item  Divoce
\item  Dotčeně
\item  Drsně
\item  Drze
\item  Důstojně
\item  Energicky
\item  Chladně
\item  Intimně
\item  Jásavě
\item  Láskyplně
\item  Lhostejně
\item  Mazlivě
\item  Milostně
\item  Nadšeně
\item  Napjatě
\item  Nedůvěřivě
\item  Nerozhodně
\item  Nedůvěřivě
\item  Pobouřeně
\item  Pokorně
\item  Procítěně
\item  Přísně
\item  Radostně
\item  Rezignovaně
\item  Rozčileně
\item  Roztomile
\item  Rozlobeně
\item  Sebejistě
\item  Sladce
\item  Snivě
\item  Starostlivě
\item  Stydlivě
\item  Svěže
\item  Štítivě
\item  Toužebně
\item  Trapně
\item  Unaveně
\item  Uraženě
\item  Vítězně
\item  Výbušně
\item  Vztekle
\item  Zahanbeně
\item  Zamilovaně
\item  Zatvrzele
\item  Zle
\item  Zlobně
\item  Znepokojeně
\end{itemize}
\end{multicols} 
 
\needspace{5cm} \section{Icebreaker} \label{icebreaker} viz \odkaz{Warm-up}{warm-up} 
 
Cvičení na rozehřátí, odhození nervozity, seznámení se ve skupině. Cvičení mají zpravidla jednoduché zadání, na které se hráči musí plně soustředit. Pomáhá jim tak odhodit myšlenky z minulosti a naopak se soustředit na přítomnost. 
\needspace{5cm} \section{ImproAktiv} \label{improaktiv} Tato společenská hra je inspirována deskovou hrou Aktivity a jí podobných. Vznikla na improvizovaném setkání, s nadsázkou pojmenovém “ImproNarcis” v A Maze in Tchaiovna. 
 
{| class="wikitable" 
|\textbf{Pro:}{} 4  - 12 
|- 
|\textbf{Čas:}{} cca 1h 
|- 
|\textbf{Pomůcky:}{} měřič času, počítání bodů, karty s oblastmi 
|} 
 
Cílem hry je nasbírat co nejvíce bodů pro váš tým po předem dohodnutém počtu kol. Hráči se losem rozdělí do 2 týmů, v případě většího množství lidí, doporučujeme vytvořit více skupin. 
 
\begin{enumerate}
\item  \textbf{krok:}{} První člověk z týmu A (předvádějící) si vylosuje z karet oblast, podle které dostane od protivníka téma. Tip na oblasti najdete níže. Na výrobu dalších oblastí se meze nekladou. Vybranou oblast odhazujeme a dobráním karet končí hra. Vylosovaná oblast je veřejná.
\item  \textbf{krok:}{} Jeden člověk z týmu B vymýšlí téma, které šeptá losujícímu hráči.
\item  \textbf{krok:}{} 2. hráč z týmu B naopak vymýšlí limity improvizace, viz příklady níže, aniž by ještě v danou chvíli znal téma.
\item  \textbf{krok:}{} Hráči se musí shodnout (předvádějící a zadávající téma) , zda je věc hratelná a zda si rozumí v domluvených pravidlech.
\item  \textbf{krok:}{} Ostatní hráči z týmu A mají přesně 1 minutu, aby uhádli téma. V případě porušení pravidel (limitů) získává tým trestné body. Za 3 trestné body, dávají protihráči 1 bod.
\end{enumerate}
 
Hra končí buď a) dobráním všech karet b) po domluveném počtu kol. Hru lze protáhnout o 2. level, kdy na 1 oblast a 1 téma hrají hráči z týmu A a B zároveň. Téma předvádějí společně a je hezké, aby scéna byla založena na spolupráci. Hráč, který zadával téma pro dané kolo nehraje. 
 
\textbf{Tipy na oblasti:}{} 
předmět denní potřeby, výrobek, povolání, vymyšlená postava, reálná postava, historická událost, přísloví, název knihy, název filmu, vtip 
 
\textbf{Tipy na limity:}{} pantomima, báseň, repliky mohou začínat od 1 písmene, zpěv, scéna zahraná pozpátku, repliky je možné pronášet jen ve chvíli, kdy se hráč celý hýbe, jako moderátor zpráv, ve scéně se mohou vyskytnout pouze zvířátka, hráč se za celou dobu nesmí pohnout, improvizace v \odkaz{gibberish}{gibberish}, předvádění pouze rukama a hlasem, na židli bez dotknutí se země apod.. 
 
 
\needspace{5cm} \section{Improtřesk} \label{improtřesk} Několikadenní akce pro improvizátory, sestávající se z několika workshopů, večerních představení a dalších souvisejících aktivit, networkingu improvizátorů, konference atd. 
 
\begin{itemize}
\item 2011 - Improtřesk 2011 - oslava 10 let Improligy - DDM Klamovka
\item 2012 - Improtřesk 2012 - SVČ Lužánky
\item 2013 - Improtřesk 2013 - Divadlo do improligy, improliga do divadla - SVČ Lužánky
\item 2014 - Improtřesk 2014 - Get inspired - SVČ Lužánky
\item 2015 - Improtřesk 2015 - Sladíme se.   - KD Milevsko
\item 2016 - Improtřesk 2016 - Na stejné vlnové délce - KD Milevsko
\end{itemize}
 
Improtřesk 2012 organizovalo SVČ Lužánky,  
Improtřesk 2013 a 2014 skupina Bafni, Improtřesk 2015 a 2016  ČILI a přátelé. 
 
\textbf{Web}{} - http://www.improtresk.cz/ 
 
{{todo|Kdo organizoval 2011 (Asi \odkaz{ČILI}{čili})?}} 
\needspace{5cm} \section{Jazykolam} \label{jazykolam} Jazykolamem se nazývá větné spojení, které je těžko vyslovitelné. Výhodou češtiny je, že jazykolam není zas tak těžké vytvořit. :) 
Pro improvizátory je i dobré rozvíjet cit k jazyku a vytvářet si vlastní věty a zároveň si je aspoň chvíli pamatovat tak, že je dokáží zopakovat vícekrát za sebou. 
 
 
\subsection{ České jazykolamy }  
Spoustu jazykolamů známe ze škol a jsou už trochu ohrané, tak pro inspiraci méně známé: 
 
\begin{itemize}
\item  Měla babka v kapse vrabce, vrabec babce v kapse píp. Zmáčkla babka vrabce v kapse, vrabec babce v kapse chcíp.
\item  Smrž pln skvrn zvlhl z mlh.
\item  Piksla sklapla, sklapla piksla. Kapsa splaskla, splaskla kapsa.
\item  Letěl jelen jetelem, jetelem letěl jelen.
\item  Patří rododendron do čeledi rododendronovitých či nerododendronovitých?
\item  Já rád játra, ty rád játra, ty rád játra, já rád játra, co nám brání dát si játra?
\end{itemize}
 
\subsection{ Anglické jazykolamy }  
\begin{itemize}
\item  Three witches watched three watches. Which witch watched which watch?
\item  I scream, you scream, we all scream for ice cream.
\item  Black back bat. (x3)
\item  Good blood, bad blood. (x3)
\item  Fuzzy Wuzzy was a bear. Fuzzy Wuzzy had no hair. Fuzzy Wuzzy wasn’t fuzzy, was he?
\item  Eleven elves licked eleven little liquorice lollipops.
\item  A box of biscuits, a box of mixed biscuits and a biscuit mixer!
\item  She sells sea-shells on the sea-shore.
\item  A proper copper coffee pot.
\end{itemize}
 
 
\needspace{5cm} \section{Kontaktní improvizace} \label{kontaktní improvizace} Jedná se o formu improvizace, kde se využívá tanec, pohyb a dotyk pro komunikaci mezi hrajícímí. Ve hře se pracuje s fyzickým vedením, využíváním tlaku, předáváním váhy těla, emocemi a důvěrou. V české republice nabízí  kurzy kontaktní improvizace např. sdružení Druna, kde lektoruje Zita Pavlištová a mnoho dalších center po celé české republice. 
 
Kontaktní improvizace rozvíjí citlivost vnímání sebe sama i druhého člověka. Pracuje s pozorností, koordinací pohybu, tlakem, uvolněním, gravitací apod. 
 
Základní rozdělení: 
 
\begin{itemize}
\item  jam (volná improvizace)
\item  akrobatická
\item  v plném kontaktu
\item  bez doteku s partnerem
\item  komunitní tvorba
\end{itemize}
 
\needspace{5cm} \section{Postava} \label{postava} Smyšlená osoba, identita nebo entita, kterou ztvárňují \odkaz{hráči}{hráč} tím, že ji hrají. Postava se projevuje pohybem, statusem, myšlením, důležitostí v příběhu dalšími projevy. 
 
{{todo|Doplnit projevy improvizované postavy}} 
\needspace{5cm} \section{Tým} \label{tým} \odkaz{Improligový}{čili} \textbf{tým}{} je skupina lidí věnující se společně improvizaci, většinou pod vedením \odkaz{trenéra}{trenér}, soustavně se připravují na \odkaz{zápasy}{zápas} i další formy. 
 
Aktuální verze seznamu týmů uznávaných  Českou Improvizační Ligou  \textbf{je dostupný online}. 
\needspace{5cm} \section{Výměna žárovky} \label{výměna žárovky} Primitivní činnost, často používaná jako příklad pro kategorii \odkaz{1000 způsobů jak}{1000 způsobů jak}. 
 
 
---- 
 
 
''Kolik je potřeba vývojářů Microsoftu na výměnu žárovky?'' 
 
''-- Ani jeden, Microsoft definuje tmu jako standard'' 
\needspace{5cm} \section{Výuka improvizace} \label{výuka improvizace} Jak si vyzkoušet improvizaci, potkat s novými lidmi, které mají podobný zájem, nabrat nové zkušenosti a inspiraci od lektorů? Nabízíme základní přehled, kam se obrátit. 
 
\subsection{ V České republice }  
Dále několik \odkaz{improligových}{čili} týmů kromě představení pořádá pravidelně víkendové či semestrální kurzy improvizace. 
Jsou to např: 
\begin{itemize}
\item  Brno, Praha - \textbf{Bafni}
\item  Pardubice, Hradec Králové - \textbf{Paleťáci}
\end{itemize}
 
Mimo Improligu existují také profesionální instituce, na které se můžete obrátit: 
 
\begin{itemize}
\item  Praha - \textbf{Impro Institut} a \textbf{Škola improvizace} - Martin Vasquez, Ondřej Nečas, Lukáš Venclík,  \odkaz{Vanda Gabrielová}{uživatel:vandagabi} , Pája Sedláčková, Táňa Jírová a další
\item  Pardubice - \textbf{Improvision} - Andrea Moličová a Tomáš Jireček
\item  Brno - \textbf{ImproŠpíl} - Džem Theatre (středoškolská improliga)
\item  Ostrava, Olomouc - \textbf{Improacademy} - Vladislav Kos a Alexandr Dvořák
\item  Praha - [http://www.blrtheatre.com/ Blood, Love & Rhetoric] - Dan Brown, Marc Cram, Jim High a John Poston -  vedeno v angličtině
\item  Brno - \textbf{Proraptor} - Bára Herucová, David Tchelidze, Deni Tchelidze, Jana Netíková a další
\end{itemize}
 
Dále je možné domluvit si workshop s konkrétními lektory. Ti využívají improvizační techniky zpravidla v kombinaci s jejich profesionální oblastí.  
 
 
Neposlední možností je návštěva festivalu \odkaz{Improtřesk}{improtřesk}, kde je v nabídce několik zajímavých workshopů.  
  
\subsection{ V zahraničí }  
Po celém světě probíhají improvizační festivaly. Vedle večerních představení se zpravidla setkáte s workshopy vedenými v angličtině. V Německu je to například \textbf{IMPRO} pořádané Die Gorillas, v Polsku můžete zavítat do Krakova na \textbf{IMPROFEST}. Aktuální seznam festivalů v Evropě najdete na stránkách \textbf{Improworld.com} 
 
 
Velkou výzvou může být i návštěva Improvizačních škol, které jsou nejvíce rozšířené v USA a Kanadě. Například v Chicagu, USA najdete hned dvě vedle sebe \textbf{Improv Olympic} a \textbf{Second City} pořádající mimo jiné i intenzivní letní kurzy. 
 
 
20 škol Improvizace  http://www.rantlifestyle.com/2014/01/04/20-schools-learn-improv-comedy/ 
\needspace{5cm} \section{Zpomalený pohyb} \label{zpomalený pohyb} Scény jako je rvačka, souboj, milostné scény se mohou hrát ve zpomaleném pohybu. Tato forma umožňuje hráčům zvýraznit a gradovat veškeré emoce, pohyb, reakci mezi sebou. 
Umět svůj pohyb a gesta zpomalit však znamená výraznou fyzickou náročnost, propracovanou techniku v ovládání svalů a fyziologické znalosti. 
 
Na youtube existují tutoriály v angličtině, které si můžete vyhledat a nebo se můžete inspirovat moderními filmy využívají střídání rychlých a pomalých záběrů. 
 
\needspace{5cm} \section{ČILI} \label{čili} Jedná se o zkratku názvu Česká improvizační liga z.s..   
 
\textbf{ČILI}{} zaštiťuje \odkaz{zápas}{zápas} v improvizaci a je držitelem ochranné známky  \textbf{Improliga.cz}{} 
 
Členem valné hromady se může stát kdokoliv, kdo chodí do improvizačního týmu uvedeného na stránkách improliga.cz 
 
Valná hromada je svolávána jednou za rok výborem a o akci jsou informováni aktuální členové a skrze intranet Improligy i členové jednotlivých týmů. 
 
Výbor se schází častěji. 
 
Ve výkonném výboru  Čili jsou v tuto chvíli  \odkaz{Vanda Gabrielová}{uživatel:vandagabi}, Martin Dočkal a \odkaz{Václav Černý}{uživatel:vatoz}. 
V revizní komisi jsou v tuto chvíli  Pavel Wieser, Jan Formánek a Jan Drahorád. 
 
ČILI pro propagaci a komunikaci provozuje \textbf{stránky improliga.cz} (včetně subdomén) a \textbf{facebookový kanál} . 
 
Podívat se můžete na oficiální kanál \textbf{Youtube}, kde naleznete sestřihy kategorií od různých týmů. 
\needspace{5cm} \section{Zpívané kategorie} \label{zpívané kategorie} \label{:kategorie:zpívané kategorie}Seznam \odkaz{kategorií}{kategorie}, ve kterých se zpívá, obvykle za doprovodu \odkaz{hudebníka}{hudebník}. 
 
\begin{multicols}{2}\begin{itemize} 
\item  \odkaz{Barman song}{barman song}  
\item  \odkaz{Break-up song}{break-up song}  
\item  \odkaz{Kam zmizel ten starý song}{kam zmizel ten starý song}  
\item  \odkaz{Kramářská píseň}{kramářská píseň}  
\item  \odkaz{Love song}{love song}  
\item  \odkaz{Mikrofon}{mikrofon}  
\item  \odkaz{Muzikál}{muzikál}  
\item  \odkaz{Opilecká píseň}{opilecká píseň}  
\item  \odkaz{Prskavka ala šanson}{prskavka ala šanson}  
\item  \odkaz{Tříhlavý song}{tříhlavý song}  
\item  \odkaz{Vysílání FM}{vysílání fm}  
\item  \odkaz{Zpívaná}{zpívaná}  
\end{itemize} 
\end{multicols} 
\needspace{5cm} \section{Warm-upy} \label{warm-upy} Warm-up cvičení, nebo také \textbf{Icebreaker}{} či \textbf{Rozehřívačky}{}, mají za úkol hráče zaktivovat, rozehřát, dostat do dobré nálady, najít důvěru k ostatním hráčům a zkoncentrovat se. Zpravidla se tyto cvičení nepoužívají pro zlepšování speciálních dovedností. Jsou vkládány na začátek workshopů nebo před vystoupením. Dobře zabaví i během výletů, při čekání na vlak. 
 
Mají zpravidla jednoduché zadání, na které se hráči musí plně soustředit. Pomáhá jim tak odhodit myšlenky z minulosti a naopak se soustředit na přítomnost. 
\begin{multicols}{2}\begin{itemize} 
\item  \odkaz{Asociace (cvičení)}{asociace (cvičení)}  
\item  \odkaz{Bunny bunny}{bunny bunny}  
\item  \odkaz{Gordický uzel}{gordický uzel}  
\item  \odkaz{Heja hají}{heja hají}  
\item  \odkaz{Hot spot}{hot spot}  
\item  \odkaz{Kdo ukradl sušenky a ze stolu je vzal}{kdo ukradl sušenky a ze stolu je vzal}  
\item  \odkaz{Kung-fu}{kung-fu}  
\item  \odkaz{Otázky (cvičení)}{otázky (cvičení)}  
\item  \odkaz{Popcorn}{popcorn}  
\item  \odkaz{Posílání signálu po kruhu}{posílání signálu po kruhu}  
\item  \odkaz{Samuraj}{samuraj}  
\item  \odkaz{Zabírání kopce}{zabírání kopce}  
\item  \odkaz{Živé obrazy}{živé obrazy}  
\end{itemize} 
\end{multicols} 
